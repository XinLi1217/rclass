\documentclass[8pt,ignorenonframetext,dvipsnames]{beamer}
\setbeamertemplate{caption}[numbered]
\setbeamertemplate{caption label separator}{: }
\setbeamercolor{caption name}{fg=normal text.fg}
\beamertemplatenavigationsymbolsempty
\usepackage{lmodern}
\usepackage{amssymb,amsmath}
\usepackage{ifxetex,ifluatex}
\usepackage{fixltx2e} % provides \textsubscript
\ifnum 0\ifxetex 1\fi\ifluatex 1\fi=0 % if pdftex
  \usepackage[T1]{fontenc}
  \usepackage[utf8]{inputenc}
\else % if luatex or xelatex
  \ifxetex
    \usepackage{mathspec}
  \else
    \usepackage{fontspec}
  \fi
  \defaultfontfeatures{Ligatures=TeX,Scale=MatchLowercase}
\fi
% use upquote if available, for straight quotes in verbatim environments
\IfFileExists{upquote.sty}{\usepackage{upquote}}{}
% use microtype if available
\IfFileExists{microtype.sty}{%
\usepackage{microtype}
\UseMicrotypeSet[protrusion]{basicmath} % disable protrusion for tt fonts
}{}
\newif\ifbibliography
\hypersetup{
            pdftitle={Lecture 5: Survey data and exploratory data analysis (for data quality)},
            pdfauthor={Ozan Jaquette},
            colorlinks=true,
            linkcolor=Maroon,
            citecolor=Blue,
            urlcolor=blue,
            breaklinks=true}
\urlstyle{same}  % don't use monospace font for urls
\usepackage{color}
\usepackage{fancyvrb}
\newcommand{\VerbBar}{|}
\newcommand{\VERB}{\Verb[commandchars=\\\{\}]}
\DefineVerbatimEnvironment{Highlighting}{Verbatim}{commandchars=\\\{\}}
% Add ',fontsize=\small' for more characters per line
\usepackage{framed}
\definecolor{shadecolor}{RGB}{248,248,248}
\newenvironment{Shaded}{\begin{snugshade}}{\end{snugshade}}
\newcommand{\KeywordTok}[1]{\textcolor[rgb]{0.13,0.29,0.53}{\textbf{#1}}}
\newcommand{\DataTypeTok}[1]{\textcolor[rgb]{0.13,0.29,0.53}{#1}}
\newcommand{\DecValTok}[1]{\textcolor[rgb]{0.00,0.00,0.81}{#1}}
\newcommand{\BaseNTok}[1]{\textcolor[rgb]{0.00,0.00,0.81}{#1}}
\newcommand{\FloatTok}[1]{\textcolor[rgb]{0.00,0.00,0.81}{#1}}
\newcommand{\ConstantTok}[1]{\textcolor[rgb]{0.00,0.00,0.00}{#1}}
\newcommand{\CharTok}[1]{\textcolor[rgb]{0.31,0.60,0.02}{#1}}
\newcommand{\SpecialCharTok}[1]{\textcolor[rgb]{0.00,0.00,0.00}{#1}}
\newcommand{\StringTok}[1]{\textcolor[rgb]{0.31,0.60,0.02}{#1}}
\newcommand{\VerbatimStringTok}[1]{\textcolor[rgb]{0.31,0.60,0.02}{#1}}
\newcommand{\SpecialStringTok}[1]{\textcolor[rgb]{0.31,0.60,0.02}{#1}}
\newcommand{\ImportTok}[1]{#1}
\newcommand{\CommentTok}[1]{\textcolor[rgb]{0.56,0.35,0.01}{\textit{#1}}}
\newcommand{\DocumentationTok}[1]{\textcolor[rgb]{0.56,0.35,0.01}{\textbf{\textit{#1}}}}
\newcommand{\AnnotationTok}[1]{\textcolor[rgb]{0.56,0.35,0.01}{\textbf{\textit{#1}}}}
\newcommand{\CommentVarTok}[1]{\textcolor[rgb]{0.56,0.35,0.01}{\textbf{\textit{#1}}}}
\newcommand{\OtherTok}[1]{\textcolor[rgb]{0.56,0.35,0.01}{#1}}
\newcommand{\FunctionTok}[1]{\textcolor[rgb]{0.00,0.00,0.00}{#1}}
\newcommand{\VariableTok}[1]{\textcolor[rgb]{0.00,0.00,0.00}{#1}}
\newcommand{\ControlFlowTok}[1]{\textcolor[rgb]{0.13,0.29,0.53}{\textbf{#1}}}
\newcommand{\OperatorTok}[1]{\textcolor[rgb]{0.81,0.36,0.00}{\textbf{#1}}}
\newcommand{\BuiltInTok}[1]{#1}
\newcommand{\ExtensionTok}[1]{#1}
\newcommand{\PreprocessorTok}[1]{\textcolor[rgb]{0.56,0.35,0.01}{\textit{#1}}}
\newcommand{\AttributeTok}[1]{\textcolor[rgb]{0.77,0.63,0.00}{#1}}
\newcommand{\RegionMarkerTok}[1]{#1}
\newcommand{\InformationTok}[1]{\textcolor[rgb]{0.56,0.35,0.01}{\textbf{\textit{#1}}}}
\newcommand{\WarningTok}[1]{\textcolor[rgb]{0.56,0.35,0.01}{\textbf{\textit{#1}}}}
\newcommand{\AlertTok}[1]{\textcolor[rgb]{0.94,0.16,0.16}{#1}}
\newcommand{\ErrorTok}[1]{\textcolor[rgb]{0.64,0.00,0.00}{\textbf{#1}}}
\newcommand{\NormalTok}[1]{#1}
\usepackage{longtable,booktabs}
\usepackage{caption}
% These lines are needed to make table captions work with longtable:
\makeatletter
\def\fnum@table{\tablename~\thetable}
\makeatother

% Prevent slide breaks in the middle of a paragraph:
\widowpenalties 1 10000
\raggedbottom

\AtBeginPart{
  \let\insertpartnumber\relax
  \let\partname\relax
  \frame{\partpage}
}
\AtBeginSection{
  \ifbibliography
  \else
    \let\insertsectionnumber\relax
    \let\sectionname\relax
    \frame{\sectionpage}
  \fi
}
\AtBeginSubsection{
  \let\insertsubsectionnumber\relax
  \let\subsectionname\relax
  \frame{\subsectionpage}
}

\setlength{\parindent}{0pt}
\setlength{\parskip}{6pt plus 2pt minus 1pt}
\setlength{\emergencystretch}{3em}  % prevent overfull lines
\providecommand{\tightlist}{%
  \setlength{\itemsep}{0pt}\setlength{\parskip}{0pt}}
\setcounter{secnumdepth}{0}

%packages
\usepackage{graphicx}
\usepackage{rotating}
\usepackage{hyperref}

\usepackage{tikz} % used for text highlighting, amongst others
%title slide stuff
%\institute{Department of Education}
%\title{Managing and Manipulating Data Using R}

%
\setbeamertemplate{navigation symbols}{} % get rid of navigation icons:

%\setbeamertemplate{frametitle}{\thesection \hspace{0.2cm} \insertframetitle}
\setbeamertemplate{section in toc}[sections numbered]
\setbeamertemplate{subsection in toc}[subsections numbered]

%define colors
%\definecolor{uva_orange}{RGB}{216,141,42} % UVa orange (Rotunda orange)
\definecolor{mygray}{rgb}{0.95, 0.95, 0.95} % for highlighted text
	% grey is equal parts red, green, blue. higher values >> lighter grey
	%\definecolor{lightgraybo}{rgb}{0.83, 0.83, 0.83}

% new commands

%highlight text with very light grey
\newcommand*{\hlg}[1]{%
	\tikz[baseline=(X.base)] \node[rectangle, fill=mygray] (X) {#1};%
}
%, inner sep=0.3mm
%highlight text with very light grey and use font associated with code
\newcommand*{\hlgc}[1]{\texttt{\hlg{#1}}}

% Font
\usepackage[defaultfam,light,tabular,lining]{montserrat}
\usepackage[T1]{fontenc}
\renewcommand*\oldstylenums[1]{{\fontfamily{Montserrat-TOsF}\selectfont #1}}

% Change color of boldface text to darkgray
\renewcommand{\textbf}[1]{{\color{darkgray}\bfseries\fontfamily{Montserrat-TOsF}#1}}

% Bullet points
\setbeamertemplate{itemize item}{\color{BlueViolet}$\circ$}
\setbeamertemplate{itemize subitem}{\color{BrickRed}$\triangleright$}
\setbeamertemplate{itemize subsubitem}{$-$}

% Reduce space before lists
\addtobeamertemplate{itemize/enumerate body begin}{}{\vspace*{-8pt}}

\title{Lecture 5: Survey data and exploratory data analysis (for data quality)}
\subtitle{EDUC 263: Managing and Manipulating Data Using R}
\author{Ozan Jaquette}
\date{}

\begin{document}
\frame{\titlepage}

\begin{frame}
\tableofcontents[hideallsubsections]
\end{frame}

\begin{frame}

\end{frame}

\section{Introduction}\label{introduction}

\begin{frame}[fragile]{Libraries we will use today {[}install if you
don't have them{]}}

\begin{Shaded}
\begin{Highlighting}[]
\KeywordTok{library}\NormalTok{(tidyverse)}
\CommentTok{#> -- Attaching packages ---------------------------------------------------------------------------------- tidyverse 1.2.1 --}
\CommentTok{#> v ggplot2 3.0.0     v purrr   0.2.5}
\CommentTok{#> v tibble  1.4.2     v dplyr   0.7.6}
\CommentTok{#> v tidyr   0.8.1     v stringr 1.3.1}
\CommentTok{#> v readr   1.1.1     v forcats 0.3.0}
\CommentTok{#> -- Conflicts ------------------------------------------------------------------------------------- tidyverse_conflicts() --}
\CommentTok{#> x dplyr::filter() masks stats::filter()}
\CommentTok{#> x dplyr::lag()    masks stats::lag()}
\KeywordTok{library}\NormalTok{(haven)}
\KeywordTok{library}\NormalTok{(labelled)}
\end{Highlighting}
\end{Shaded}

\end{frame}

\begin{frame}{Data we will use today}

High school longitudinal surveys from National Center for Education
Statistics (NCES)

\begin{itemize}
\tightlist
\item
  Follow U.S. students from high school through college, labor market
\end{itemize}

We will be working with
\href{https://nces.ed.gov/surveys/hsls09/index.asp}{High School
Longitudinal Study of 2009 (HSLS:09)}

\begin{itemize}
\tightlist
\item
  Follows 9th graders from 2009
\item
  Data collection waves

  \begin{itemize}
  \tightlist
  \item
    Base Year (2009)
  \item
    First Follow-up (2012)
  \item
    2013 Update (2013)
  \item
    High School Transcripts (2013-2014)
  \item
    Second Follow-up (2016)
  \end{itemize}
\end{itemize}

\end{frame}

\section{\texorpdfstring{\texttt{haven} and \texttt{labelled}
package}{haven and labelled package}}\label{haven-and-labelled-package}

\begin{frame}[fragile]{\texttt{haven} package}

\href{https://haven.tidyverse.org/}{\texttt{haven}}, which is part of
\textbf{tidyverse}, ``enables R to read and write various data formats''
from the following statistical packages:

\begin{itemize}
\tightlist
\item
  SAS
\item
  SPSS
\item
  Stata
\end{itemize}

When using \texttt{haven} to read data, resulting R objects have these
characteristics:

\begin{itemize}
\tightlist
\item
  Are \textbf{tibbles}, a particular type of data frame we discuss
  future weeks
\item
  Transform variables with ``value labels'' into the \texttt{labelled()}
  class {[}our focus today{]}

  \begin{itemize}
  \tightlist
  \item
    Helpful description
    \href{https://haven.tidyverse.org/articles/semantics.html}{HERE}
  \end{itemize}
\item
  Dates and times converted to R date/time classes
\item
  Character vectors not converted to factors
\end{itemize}

\end{frame}

\begin{frame}[fragile]{\texttt{haven} package}

Use \texttt{read\_dta()} function from \texttt{haven} to import Stata
dataset into R

\begin{Shaded}
\begin{Highlighting}[]
\NormalTok{hsls <-}\StringTok{ }\KeywordTok{read_dta}\NormalTok{(}\DataTypeTok{file=}\StringTok{"../../data/hsls/hsls_stu_small.dta"}\NormalTok{)}
\end{Highlighting}
\end{Shaded}

Let's examine the data

\begin{Shaded}
\begin{Highlighting}[]
\KeywordTok{names}\NormalTok{(hsls)}
\KeywordTok{names}\NormalTok{(hsls) <-}\StringTok{ }\KeywordTok{tolower}\NormalTok{(}\KeywordTok{names}\NormalTok{(hsls)) }\CommentTok{# convert names to lowercase}
\KeywordTok{str}\NormalTok{(hsls)}
\KeywordTok{str}\NormalTok{(hsls}\OperatorTok{$}\NormalTok{s3classes)}
\end{Highlighting}
\end{Shaded}

\end{frame}

\begin{frame}[fragile]{\texttt{labelled} package}

\texttt{labelled} package is to work with data imported from
SPSS/Stata/SAS using the \texttt{haven} package.

\begin{itemize}
\tightlist
\item
  Specifically, ``The purpose of the \texttt{labelled} package is to
  provide functions to manipulate \emph{metadata} as variable labels,
  value labels and defined missing values using the
  \texttt{labelled\ class} and the \texttt{label} attribute introduced
  in \texttt{haven} package.
  \href{https://cran.r-project.org/web/packages/labelled/vignettes/intro_labelled.html}{LINK}
\end{itemize}

Functions in \texttt{labelled} package

\begin{itemize}
\tightlist
\item
  \href{https://www.rdocumentation.org/packages/labelled/versions/1.1.0}{Full
  list}
\item
  A couple relevant functions

  \begin{itemize}
  \tightlist
  \item
    \texttt{val\_labels}: get or set variable \emph{value labels}
  \item
    \texttt{var\_label}: get or set a \emph{variable label}
  \end{itemize}
\end{itemize}

\begin{Shaded}
\begin{Highlighting}[]
\NormalTok{hsls }\OperatorTok\StringTok{ }\KeywordTok{select}\NormalTok{(s3classes) }\OperatorTok\StringTok{ }\NormalTok{var_label}
\NormalTok{hsls }\OperatorTok\StringTok{ }\KeywordTok{select}\NormalTok{(s3classes) }\OperatorTok\StringTok{ }\NormalTok{val_labels}
\end{Highlighting}
\end{Shaded}

\end{frame}

\begin{frame}[fragile]{Understanding \texttt{labelled} data}

First, let's review core concepts:

\textbf{atomic vectors (and lists)} the underlying data

\begin{itemize}
\tightlist
\item
  data structures: vector or list
\item
  data type: numeric (integer or double); character; logical
\end{itemize}

\begin{Shaded}
\begin{Highlighting}[]
\KeywordTok{typeof}\NormalTok{(hsls}\OperatorTok{$}\NormalTok{s3classes)}
\CommentTok{#> [1] "double"}
\end{Highlighting}
\end{Shaded}

\textbf{augmented vectors} are atomic vectors with \textbf{attributes}
attached

\textbf{attributes} are ``metadata'' attached to an object. Examples

\begin{itemize}
\tightlist
\item
  \textbf{names}: names of elements of a vector or list (e.g., variable
  names)
\item
  \textbf{levels}: display output associated with values of a factor
  variable
\item
  \textbf{class}: e.g., factor, labelled
\end{itemize}

\begin{Shaded}
\begin{Highlighting}[]
\KeywordTok{attributes}\NormalTok{(hsls}\OperatorTok{$}\NormalTok{s3classes)}
\end{Highlighting}
\end{Shaded}

\textbf{class} is an object oriented programming concept. The
\texttt{class} of an object determines which functions can be applied to
the object and what those functions do

\begin{itemize}
\tightlist
\item
  e.g., can't apply \texttt{sum()} to an object where
  \texttt{class=character}
\end{itemize}

\end{frame}

\begin{frame}[fragile]{Understanding \texttt{labelled} data}

Let's investigate the attributes of \texttt{hsls\$s3classes}

\begin{Shaded}
\begin{Highlighting}[]
\KeywordTok{typeof}\NormalTok{(hsls}\OperatorTok{$}\NormalTok{s3classes)}
\KeywordTok{class}\NormalTok{(hsls}\OperatorTok{$}\NormalTok{s3classes)}

\KeywordTok{str}\NormalTok{(hsls}\OperatorTok{$}\NormalTok{s3classes)}
\KeywordTok{attributes}\NormalTok{(hsls}\OperatorTok{$}\NormalTok{s3classes)}

\CommentTok{#use attr(object_name,"attribute_name") to refer to each attribute}
\KeywordTok{attr}\NormalTok{(hsls}\OperatorTok{$}\NormalTok{s3classes,}\StringTok{"label"}\NormalTok{)}
\KeywordTok{attr}\NormalTok{(hsls}\OperatorTok{$}\NormalTok{s3classes,}\StringTok{"labels"}\NormalTok{)}
\KeywordTok{attr}\NormalTok{(hsls}\OperatorTok{$}\NormalTok{s3classes,}\StringTok{"class"}\NormalTok{)}
\KeywordTok{attr}\NormalTok{(hsls}\OperatorTok{$}\NormalTok{s3classes,}\StringTok{"format.stata"}\NormalTok{)}
\end{Highlighting}
\end{Shaded}

\end{frame}

\begin{frame}[fragile]{Understanding \texttt{labelled} data}

What is \texttt{class==labelled}?

\begin{itemize}
\tightlist
\item
  An object class created by the \texttt{haven} package for importing
  variables from SAS/SPSS/Stata that have \textbf{value labels}
\item
  \textbf{value labels} {[}in Stata{]} are labels attached to specific
  values of a variable:

  \begin{itemize}
  \tightlist
  \item
    e.g., variable value \texttt{1} attached to value label ``married'',
    \texttt{2}=``single'', \texttt{3}=``divorced''
  \end{itemize}
\item
  Variables in an R data frame with \texttt{class==labelled}:

  \begin{itemize}
  \tightlist
  \item
    data \texttt{type} can be numeric(double) or character
  \item
    The \texttt{value\ labels} associated with each value:
    \texttt{attr(data\_frame\_name\$variable\_name,"labels")}
  \item
    In \texttt{filter()} refer to variable value, not the value label
  \end{itemize}
\end{itemize}

Working with \texttt{class==labelled} variables

\begin{Shaded}
\begin{Highlighting}[]
\CommentTok{#show variable label}
\NormalTok{hsls }\OperatorTok\StringTok{ }\KeywordTok{select}\NormalTok{(s3classes,s3clglvl) }\OperatorTok\StringTok{ }\NormalTok{var_label}
\CommentTok{#show value labels}
\NormalTok{hsls }\OperatorTok\StringTok{ }\KeywordTok{select}\NormalTok{(s3classes,s3clglvl) }\OperatorTok\StringTok{ }\NormalTok{val_labels}
\CommentTok{#Frequency table}
\NormalTok{hsls }\OperatorTok\StringTok{ }\KeywordTok{select}\NormalTok{(s3classes) }\OperatorTok\StringTok{ }\KeywordTok{count}\NormalTok{(s3classes)}
\CommentTok{#Frequency table, value labels instead of variable values}
\NormalTok{hsls }\OperatorTok\StringTok{ }\KeywordTok{select}\NormalTok{(s3classes) }\OperatorTok\StringTok{ }\KeywordTok{count}\NormalTok{(s3classes) }\OperatorTok\StringTok{ }\KeywordTok{as_factor}\NormalTok{()}
\CommentTok{#in filters, refer to variable value, not value label}
\NormalTok{hsls }\OperatorTok\StringTok{ }\KeywordTok{select}\NormalTok{(s3classes) }\OperatorTok\StringTok{ }\KeywordTok{filter}\NormalTok{(s3classes}\OperatorTok{==-}\DecValTok{8}\NormalTok{) }\OperatorTok\StringTok{ }\KeywordTok{count}\NormalTok{(s3classes)}
\end{Highlighting}
\end{Shaded}

\end{frame}

\begin{frame}[fragile]{Converting \texttt{class==labelled} to
\texttt{class==factor}}

The \texttt{as\_factor()} function from \texttt{haven} package converts
variables with \texttt{class==labelled} to \texttt{class==factor}

\begin{itemize}
\tightlist
\item
  Can be used for descriptive statistics
\end{itemize}

\begin{Shaded}
\begin{Highlighting}[]
\NormalTok{hsls }\OperatorTok\StringTok{ }\KeywordTok{select}\NormalTok{(s3classes) }\OperatorTok\StringTok{ }\KeywordTok{count}\NormalTok{(s3classes) }\OperatorTok\StringTok{ }\KeywordTok{as_factor}\NormalTok{()}
\end{Highlighting}
\end{Shaded}

\begin{itemize}
\tightlist
\item
  Can create object with some or all \texttt{labelled} vars converted to
  \texttt{factor}
\end{itemize}

\begin{Shaded}
\begin{Highlighting}[]
\NormalTok{hsls_f <-}\StringTok{ }\KeywordTok{as_factor}\NormalTok{(hsls,}\DataTypeTok{only_labelled =} \OtherTok{TRUE}\NormalTok{)}
\end{Highlighting}
\end{Shaded}

Let's examine this object

\begin{Shaded}
\begin{Highlighting}[]
\KeywordTok{glimpse}\NormalTok{(hsls_f)}
\NormalTok{hsls_f }\OperatorTok\StringTok{ }\KeywordTok{select}\NormalTok{(s3classes,s3clglvl) }\OperatorTok\StringTok{ }\KeywordTok{str}\NormalTok{()}
\KeywordTok{typeof}\NormalTok{(hsls_f}\OperatorTok{$}\NormalTok{s3classes)}
\KeywordTok{class}\NormalTok{(hsls_f}\OperatorTok{$}\NormalTok{s3classes)}
\KeywordTok{attributes}\NormalTok{(hsls_f}\OperatorTok{$}\NormalTok{s3classes)}

\NormalTok{hsls_f }\OperatorTok\StringTok{ }\KeywordTok{select}\NormalTok{(s3classes) }\OperatorTok\StringTok{ }\KeywordTok{var_label}\NormalTok{()}
\NormalTok{hsls_f }\OperatorTok\StringTok{ }\KeywordTok{select}\NormalTok{(s3classes) }\OperatorTok\StringTok{ }\KeywordTok{val_labels}\NormalTok{()}
\end{Highlighting}
\end{Shaded}

\end{frame}

\begin{frame}[fragile]{Working with \texttt{class==factor} data}

Showing values associated with factor levels

\begin{Shaded}
\begin{Highlighting}[]
\NormalTok{hsls_f }\OperatorTok\StringTok{ }\KeywordTok{select}\NormalTok{(s3classes) }\OperatorTok\StringTok{ }\KeywordTok{count}\NormalTok{(s3classes)}
\CommentTok{#> # A tibble: 5 x 2}
\CommentTok{#>   s3classes             n}
\CommentTok{#>   <fct>             <int>}
\CommentTok{#> 1 Missing              59}
\CommentTok{#> 2 Unit non-response  4945}
\CommentTok{#> 3 Yes               13477}
\CommentTok{#> 4 No                 3401}
\CommentTok{#> 5 Don't know         1621}
\end{Highlighting}
\end{Shaded}

In code, refer \texttt{level} attribute not variable value

\begin{Shaded}
\begin{Highlighting}[]
\NormalTok{hsls_f }\OperatorTok\StringTok{ }\KeywordTok{select}\NormalTok{(s3classes) }\OperatorTok\StringTok{ }\KeywordTok{filter}\NormalTok{(s3classes}\OperatorTok{==}\StringTok{"Yes"}\NormalTok{) }\OperatorTok\StringTok{ }\KeywordTok{count}\NormalTok{(s3classes)}
\CommentTok{#> # A tibble: 1 x 2}
\CommentTok{#>   s3classes     n}
\CommentTok{#>   <fct>     <int>}
\CommentTok{#> 1 Yes       13477}
\end{Highlighting}
\end{Shaded}

\end{frame}

\begin{frame}[fragile]{Comparing \texttt{class==labelled} to
\texttt{class==factor}}

\begin{longtable}[]{@{}lll@{}}
\toprule
& \texttt{class==labelled} & \texttt{class==factor}\tabularnewline
\midrule
\endhead
data type & numeric or character & integer\tabularnewline
name of value label attribute & labels & levels\tabularnewline
refer to data using & variable values & levels attribute\tabularnewline
\bottomrule
\end{longtable}

\end{frame}

\section{Exploratory data analysis
(EDA)}\label{exploratory-data-analysis-eda}

\subsection{Tools for EDA}\label{tools-for-eda}

\begin{frame}{What is exploratory data analysis (EDA)?}

The
\href{https://towardsdatascience.com/exploratory-data-analysis-8fc1cb20fd15}{Towards
Data Science} website has a nice definition of EDA:

\begin{quote}
``Exploratory Data Analysis refers to the critical process of performing
initial investigations on data so as to discover patterns,to spot
anomalies,to test hypothesis and to check assumptions with the help of
summary statistics and graphical representations.''
\end{quote}

This course focuses on ``data management'':

\begin{itemize}
\tightlist
\item
  investigating and cleaning data for the purpose of creating analysis
  variables
\item
  Basically, everything that happens before you conduct analyses
\end{itemize}

I think about ``Exploratory data analysis for data quality''

\begin{itemize}
\tightlist
\item
  Investigating values and patterns of variables from ``input data''
\item
  Identifying and cleaning errors or values that need to be changed
\item
  Creating analysis variables
\item
  Checking values of analysis variables agains values of input variables
\end{itemize}

\end{frame}

\begin{frame}{Tools of EDA}

To do EDA for data quality, must master the following tools:

\begin{itemize}
\tightlist
\item
  \medskip Select, sort, filter, and print

  \begin{itemize}
  \tightlist
  \item
    Select and sort particular values of particular variables
  \item
    Print particular values of particular variables
  \end{itemize}
\item
  One-way descriptive analyses (i.e,. focus on one variable)

  \begin{itemize}
  \tightlist
  \item
    Descriptive analyses for continuous variables
  \item
    Descriptive analyses for discreet/categorical variables
  \end{itemize}
\item
  Two-way descriptive analyses (relationship between two variables)

  \begin{itemize}
  \tightlist
  \item
    Categorical by categorical
  \item
    Categorical by continuous
  \item
    Continuous by continuous
  \end{itemize}
\end{itemize}

Whenever using any of these tools, \textbf{pay close attention to
missing values and how they are coded}

I'll focus on the \textbf{tidyverse} approach rather than \textbf{base
R}

\end{frame}

\begin{frame}[fragile]{Tools of EDA: select, sort, filter, and print}

Let's create a smaller version of the HSLS:09 dataset

\begin{Shaded}
\begin{Highlighting}[]
\NormalTok{hsls }\OperatorTok\StringTok{ }\KeywordTok{var_label}\NormalTok{()}
\CommentTok{#> $stu_id}
\CommentTok{#> [1] "Student ID"}
\CommentTok{#> }
\CommentTok{#> $sch_id}
\CommentTok{#> [1] "School ID"}
\CommentTok{#> }
\CommentTok{#> $x2univ1}
\CommentTok{#> [1] "X2 Sample member status in BY and F1 rounds"}
\CommentTok{#> }
\CommentTok{#> $x2univ2a}
\CommentTok{#> [1] "X2 Base year status and how sample member entered F1 sample"}
\CommentTok{#> }
\CommentTok{#> $x2univ2b}
\CommentTok{#> [1] "X2 Sample member F1 status"}
\CommentTok{#> }
\CommentTok{#> $x3univ1}
\CommentTok{#> [1] "X3 Sample member status in BY, F1, U13, and HS transcript rounds"}
\CommentTok{#> }
\CommentTok{#> $x4univ1}
\CommentTok{#> [1] "X4 Sample member status in BY, F1, U13,  HS transcript, and F2 rounds"}
\CommentTok{#> }
\CommentTok{#> $s3classes}
\CommentTok{#> [1] "S3 B01A Taking postsecondary classes as of Nov 1 2013"}
\CommentTok{#> }
\CommentTok{#> $s3apprentice}
\CommentTok{#> [1] "S3 B01B Apprenticing as of Nov 1 2013"}
\CommentTok{#> }
\CommentTok{#> $s3work}
\CommentTok{#> [1] "S3 B01C Working for pay as of Nov 1 2013"}
\CommentTok{#> }
\CommentTok{#> $s3military}
\CommentTok{#> [1] "S3 B01D Serving in the military as of Nov 1 2013"}
\CommentTok{#> }
\CommentTok{#> $s3family}
\CommentTok{#> [1] "S3 B01E Starting family/taking care of children as of Nov 1 2013"}
\CommentTok{#> }
\CommentTok{#> $s3hs}
\CommentTok{#> [1] "S3 B01F Attending high school or homeschool as of Nov 1 2013"}
\CommentTok{#> }
\CommentTok{#> $s3gedcourse}
\CommentTok{#> [1] "S3 B01G In a course to prepare for GED as of Nov 1 2013"}
\CommentTok{#> }
\CommentTok{#> $s3focus}
\CommentTok{#> [1] "S3 B02 Main focus as of Nov 1 2013"}
\CommentTok{#> }
\CommentTok{#> $s3clgft}
\CommentTok{#> [1] "S3 B03 Attending college full-time or part-time as of Nov 1 2013"}
\CommentTok{#> }
\CommentTok{#> $s3workft}
\CommentTok{#> [1] "S3 B04 Working full-time as of Nov 1 2013"}
\CommentTok{#> }
\CommentTok{#> $s3milbranch}
\CommentTok{#> [1] "S3 B05 Branch of the military will be serving in as of Nov 1 2013"}
\CommentTok{#> }
\CommentTok{#> $s3clgid}
\CommentTok{#> [1] "S3 C01 Postsecondary institution attending as of Nov 1 2013 - IPEDS ID"}
\CommentTok{#> }
\CommentTok{#> $s3clgcntrl}
\CommentTok{#> [1] "S3 Enrolled college IPEDS control"}
\CommentTok{#> }
\CommentTok{#> $s3clglvl}
\CommentTok{#> [1] "S3 Enrolled college IPEDS level"}
\CommentTok{#> }
\CommentTok{#> $s3clgsel}
\CommentTok{#> [1] "S3 Enrolled college IPEDS selectivity code"}
\CommentTok{#> }
\CommentTok{#> $s3clgstate}
\CommentTok{#> [1] "S3 Enrolled college IPEDS state code"}
\CommentTok{#> }
\CommentTok{#> $s3proglevel}
\CommentTok{#> [1] "S3 C02 Level of program enrolled in as of Nov 1 2013"}
\CommentTok{#> }
\CommentTok{#> $x3sqstat}
\CommentTok{#> [1] "X3 Student questionnaire status"}
\CommentTok{#> }
\CommentTok{#> $x2sex}
\CommentTok{#> [1] "X2 Student's sex"}
\CommentTok{#> }
\CommentTok{#> $x2race}
\CommentTok{#> [1] "X2 Student's race/ethnicity-composite"}
\CommentTok{#> }
\CommentTok{#> $x2paredu}
\CommentTok{#> [1] "X2 Parents'/guardians' highest level of education"}
\CommentTok{#> }
\CommentTok{#> $x2txmtscor}
\CommentTok{#> [1] "X2 Mathematics standardized theta score"}
\CommentTok{#> }
\CommentTok{#> $x4evrappclg}
\CommentTok{#> [1] "X4 Whether applied to or registered at a college"}
\CommentTok{#> }
\CommentTok{#> $x4clgappnum}
\CommentTok{#> [1] "X4 Number of colleges applied to when first applied"}
\CommentTok{#> }
\CommentTok{#> $x4evratndclg}
\CommentTok{#> [1] "X4 Imputed version of S4EVRATNDCLG"}
\CommentTok{#> }
\CommentTok{#> $x4evr2ypub}
\CommentTok{#> [1] "X4 Ever attended 2-year public institution after high school"}
\CommentTok{#> }
\CommentTok{#> $x4atndclg16fb}
\CommentTok{#> [1] "X4 Whether respondent was enrolled in postsecondary education in February 2016"}
\CommentTok{#> }
\CommentTok{#> $x4choiceappid}
\CommentTok{#> [1] "X4 First choice among colleges applied to"}
\CommentTok{#> }
\CommentTok{#> $x4choiceaccid}
\CommentTok{#> [1] "X4 First choice among colleges accepted to"}
\CommentTok{#> }
\CommentTok{#> $x4atndappinst}
\CommentTok{#> [1] "X4 Institution ended up attending as result of first applications"}
\CommentTok{#> }
\CommentTok{#> $x4hs2psmos}
\CommentTok{#> [1] "X4 Months between high school and postsecondary education"}
\CommentTok{#> }
\CommentTok{#> $x4psend}
\CommentTok{#> [1] "X4 Month and year of last postsecondary enrollment anywhere"}
\CommentTok{#> }
\CommentTok{#> $x4ps1}
\CommentTok{#> [1] "X4 First post-high school postsecondary institution"}
\CommentTok{#> }
\CommentTok{#> $x4ps1start}
\CommentTok{#> [1] "X4 Month and year of enrollment at first postsecondary institution"}
\CommentTok{#> }
\CommentTok{#> $x4ps1sector}
\CommentTok{#> [1] "X4 First postsecondary institution sector"}
\CommentTok{#> }
\CommentTok{#> $x4ps1level}
\CommentTok{#> [1] "X4 First postsecondary institution level"}
\CommentTok{#> }
\CommentTok{#> $x4ps1ctrl}
\CommentTok{#> [1] "X4 First postsecondary institution control"}
\CommentTok{#> }
\CommentTok{#> $x4ps1select}
\CommentTok{#> [1] "X4 First postsecondary institution selectivity"}
\CommentTok{#> }
\CommentTok{#> $x4refinst}
\CommentTok{#> [1] "X4 Reference institution ID"}
\CommentTok{#> }
\CommentTok{#> $x4refsector}
\CommentTok{#> [1] "X4 Sector of reference institution"}
\CommentTok{#> }
\CommentTok{#> $x4reflevel}
\CommentTok{#> [1] "X4 Level of reference institution"}
\CommentTok{#> }
\CommentTok{#> $x4refctrl}
\CommentTok{#> [1] "X4 Control of reference institution"}
\CommentTok{#> }
\CommentTok{#> $x4refselect}
\CommentTok{#> [1] "X4 Selectivity of reference institution"}
\CommentTok{#> }
\CommentTok{#> $x4x2ses}
\CommentTok{#> [1] "X4 Revised X2 Socio-economic status composite"}
\CommentTok{#> }
\CommentTok{#> $x4x2sesq5}
\CommentTok{#> [1] "X4 Revised X2 Quintile coding of X2SES composite"}
\CommentTok{#> }
\CommentTok{#> $x4sqstat}
\CommentTok{#> [1] "X4 Student questionnaire status"}
\NormalTok{hsls_small <-}\StringTok{ }\NormalTok{hsls }\OperatorTok\StringTok{ }\KeywordTok{select}\NormalTok{(stu_id,x3univ1,x3sqstat,x4univ1,x4sqstat,s3classes,s3work,s3focus,}
\NormalTok{  s3clgft,s3workft,s3clgid,s3clgcntrl,s3clglvl,s3clgsel,s3clgstate,s3proglevel,}
\NormalTok{  x4evrappclg,x4evratndclg,x4atndclg16fb,x4ps1sector,x4ps1level,x4ps1ctrl,x4ps1select,}
\NormalTok{  x4refsector,x4reflevel,x4refctrl,x4refselect,}
\NormalTok{  x2sex,x2race,x2paredu,x2txmtscor,x4x2ses,x4x2sesq5)}
\end{Highlighting}
\end{Shaded}

\end{frame}

\begin{frame}[fragile]{Tools of EDA: select, sort, filter, and print}

We've already know \texttt{select()}, \texttt{arrange()},
\texttt{filter()}

\medskip Select, sort, and print specific vars

\begin{Shaded}
\begin{Highlighting}[]
\NormalTok{hsls_small }\OperatorTok\StringTok{ }\KeywordTok{arrange}\NormalTok{(}\KeywordTok{desc}\NormalTok{(stu_id)) }\OperatorTok\StringTok{ }
\StringTok{  }\KeywordTok{select}\NormalTok{(stu_id,x3univ1,x3sqstat,s3classes,s3clglvl)}

\CommentTok{#print value labels}
\NormalTok{hsls_small }\OperatorTok\StringTok{ }\KeywordTok{arrange}\NormalTok{(}\KeywordTok{desc}\NormalTok{(stu_id)) }\OperatorTok\StringTok{ }
\StringTok{  }\KeywordTok{select}\NormalTok{(stu_id,x3univ1,x3sqstat,s3classes,s3clglvl) }\OperatorTok\StringTok{ }\KeywordTok{as_factor}\NormalTok{()}
\end{Highlighting}
\end{Shaded}

Sometimes helpful to increase the number of observations printed

\begin{Shaded}
\begin{Highlighting}[]
\KeywordTok{class}\NormalTok{(hsls_small) }\CommentTok{#it's a tibble}
\KeywordTok{options}\NormalTok{(}\DataTypeTok{tibble.print_min=}\DecValTok{50}\NormalTok{) }
\CommentTok{# execute this in console}
\NormalTok{hsls_small }\OperatorTok\StringTok{ }\KeywordTok{arrange}\NormalTok{(}\KeywordTok{desc}\NormalTok{(stu_id)) }\OperatorTok\StringTok{ }\KeywordTok{select}\NormalTok{(stu_id,x3univ1,x3sqstat,s3classes,s3clglvl)}
\KeywordTok{options}\NormalTok{(}\DataTypeTok{tibble.print_min=}\DecValTok{10}\NormalTok{) }\CommentTok{# set default printing back to 10 lines}
\end{Highlighting}
\end{Shaded}

\end{frame}

\begin{frame}[fragile]{One-way descriptive stats for continuous vars,
Base R approach {[}SKIP{]}}

\begin{Shaded}
\begin{Highlighting}[]
\KeywordTok{mean}\NormalTok{(hsls_small}\OperatorTok{$}\NormalTok{x2txmtscor)}
\KeywordTok{sd}\NormalTok{(hsls_small}\OperatorTok{$}\NormalTok{x2txmtscor)}

\CommentTok{#Careful: summary stats include value of -8!}
\KeywordTok{min}\NormalTok{(hsls_small}\OperatorTok{$}\NormalTok{x2txmtscor)}
\KeywordTok{max}\NormalTok{(hsls_small}\OperatorTok{$}\NormalTok{x2txmtscor)}
\end{Highlighting}
\end{Shaded}

Be careful with \texttt{NA} values

\begin{Shaded}
\begin{Highlighting}[]
\CommentTok{#Create variable replacing -8 with NA}
\NormalTok{hsls_small_temp <-}\StringTok{ }\NormalTok{hsls_small }\OperatorTok\StringTok{ }
\StringTok{  }\KeywordTok{mutate}\NormalTok{(}\DataTypeTok{x2txmtscorv2=}\KeywordTok{ifelse}\NormalTok{(x2txmtscor}\OperatorTok{==-}\DecValTok{8}\NormalTok{,}\OtherTok{NA}\NormalTok{,x2txmtscor))}
\NormalTok{hsls_small_temp }\OperatorTok\StringTok{ }\KeywordTok{filter}\NormalTok{(}\KeywordTok{is.na}\NormalTok{(x2txmtscorv2)) }\OperatorTok\StringTok{ }\KeywordTok{count}\NormalTok{(x2txmtscorv2)}

\KeywordTok{mean}\NormalTok{(hsls_small_temp}\OperatorTok{$}\NormalTok{x2txmtscorv2)}
\KeywordTok{mean}\NormalTok{(hsls_small_temp}\OperatorTok{$}\NormalTok{x2txmtscorv2, }\DataTypeTok{na.rm=}\OtherTok{TRUE}\NormalTok{)}
\KeywordTok{rm}\NormalTok{(hsls_small_temp)}
\end{Highlighting}
\end{Shaded}

\end{frame}

\begin{frame}[fragile]{One-way descriptive stats for continuous vars,
Tidyverse approach}

Use \texttt{summarise\_at()} which is a variation of
\texttt{summarise()} to calculate descriptive stats

\begin{itemize}
\tightlist
\item
  explain \texttt{.args=list(na.rm=TRUE)} on next slide
\end{itemize}

\begin{Shaded}
\begin{Highlighting}[]
\NormalTok{?summarise_at}
\CommentTok{#> starting httpd help server ... done}
\NormalTok{hsls_small }\OperatorTok\StringTok{ }
\StringTok{  }\KeywordTok{summarise_at}\NormalTok{(}
    \DataTypeTok{.vars =} \KeywordTok{vars}\NormalTok{(x2txmtscor),}
    \DataTypeTok{.funs =} \KeywordTok{funs}\NormalTok{(mean, sd, min, max, }\DataTypeTok{.args=}\KeywordTok{list}\NormalTok{(}\DataTypeTok{na.rm=}\OtherTok{TRUE}\NormalTok{))}
\NormalTok{  )}
\CommentTok{#> # A tibble: 1 x 4}
\CommentTok{#>    mean    sd   min   max}
\CommentTok{#>   <dbl> <dbl> <dbl> <dbl>}
\CommentTok{#> 1  44.1  21.8    -8  84.9}
\end{Highlighting}
\end{Shaded}

\end{frame}

\begin{frame}[fragile]{One-way descriptive stats for continuous vars,
Tidyverse approach}

``Input vars'' in survey data often have negative values for
missing/skips

\begin{itemize}
\tightlist
\item
  R includes those negative values when calculating statistics, which
  you probably don't want
\end{itemize}

Solution: create version of variable that replaces negative values with
NAs

\begin{Shaded}
\begin{Highlighting}[]
\NormalTok{hsls_small }\OperatorTok\StringTok{ }\KeywordTok{mutate}\NormalTok{(}\DataTypeTok{x2txmtscor_na=}\KeywordTok{ifelse}\NormalTok{(x2txmtscor}\OperatorTok{<}\DecValTok{0}\NormalTok{,}\OtherTok{NA}\NormalTok{,x2txmtscor)) }\OperatorTok
\StringTok{  }\KeywordTok{summarise_at}\NormalTok{(}
    \DataTypeTok{.vars =} \KeywordTok{vars}\NormalTok{(x2txmtscor_na),}
    \DataTypeTok{.funs =} \KeywordTok{funs}\NormalTok{(mean, sd, min, max, }\DataTypeTok{.args=}\KeywordTok{list}\NormalTok{(}\DataTypeTok{na.rm=}\OtherTok{TRUE}\NormalTok{))}
\NormalTok{  )}
\CommentTok{#> # A tibble: 1 x 4}
\CommentTok{#>    mean    sd   min   max}
\CommentTok{#>   <dbl> <dbl> <dbl> <dbl>}
\CommentTok{#> 1  51.5  10.2  22.2  84.9}
\end{Highlighting}
\end{Shaded}

What if you didn't include \texttt{.args=list(na.rm=TRUE)}?

\begin{Shaded}
\begin{Highlighting}[]
\NormalTok{hsls_small }\OperatorTok\StringTok{ }\KeywordTok{mutate}\NormalTok{(}\DataTypeTok{x2txmtscor_na=}\KeywordTok{ifelse}\NormalTok{(x2txmtscor}\OperatorTok{<}\DecValTok{0}\NormalTok{,}\OtherTok{NA}\NormalTok{,x2txmtscor)) }\OperatorTok
\StringTok{  }\KeywordTok{summarise_at}\NormalTok{(}
    \DataTypeTok{.vars =} \KeywordTok{vars}\NormalTok{(x2txmtscor_na),}
    \DataTypeTok{.funs =} \KeywordTok{funs}\NormalTok{(mean, sd, min, max)}
\NormalTok{  )}
\CommentTok{#> # A tibble: 1 x 4}
\CommentTok{#>    mean    sd   min   max}
\CommentTok{#>   <dbl> <dbl> <dbl> <dbl>}
\CommentTok{#> 1    NA   NaN    NA    NA}
\end{Highlighting}
\end{Shaded}

\end{frame}

\begin{frame}[fragile]{One-way descriptive stats for continuous vars,
Tidyverse approach}

How to identify these missing/skip values if you don't have a codebook?

\texttt{count()} combined with \texttt{filter()} helpful for finding
extreme values of continuous vars, which are often associated with
missing or skip

\begin{Shaded}
\begin{Highlighting}[]
\NormalTok{hsls_small }\OperatorTok\StringTok{ }\KeywordTok{filter}\NormalTok{(x2txmtscor}\OperatorTok{<}\DecValTok{0}\NormalTok{) }\OperatorTok\StringTok{ }\KeywordTok{count}\NormalTok{(x2txmtscor)}
\CommentTok{#> # A tibble: 1 x 2}
\CommentTok{#>   x2txmtscor     n}
\CommentTok{#>        <dbl> <int>}
\CommentTok{#> 1         -8  2909}

\NormalTok{hsls_small }\OperatorTok\StringTok{ }\KeywordTok{filter}\NormalTok{(s3clglvl}\OperatorTok{<}\DecValTok{0}\NormalTok{) }\OperatorTok\StringTok{ }\KeywordTok{count}\NormalTok{(s3clglvl)}
\CommentTok{#> # A tibble: 3 x 2}
\CommentTok{#>   s3clglvl      n}
\CommentTok{#>   <dbl+lbl> <int>}
\CommentTok{#> 1 -9          487}
\CommentTok{#> 2 -8         4945}
\CommentTok{#> 3 -7         5022}
\end{Highlighting}
\end{Shaded}

\end{frame}

\begin{frame}{Student exercise}

\end{frame}

\begin{frame}[fragile]{One-way descriptive stats for
discrete/categorical vars, Tidyverse approach}

Use \texttt{count()} to investigate values of discreet or categorical
variables

For variables where \texttt{class==labelled}

\begin{Shaded}
\begin{Highlighting}[]
\KeywordTok{class}\NormalTok{(hsls_small}\OperatorTok{$}\NormalTok{s3classes)}
\CommentTok{#show counts of variable values}
\NormalTok{hsls_small }\OperatorTok\StringTok{ }\KeywordTok{count}\NormalTok{(s3classes)}
\CommentTok{#show counts of value labels}
\NormalTok{hsls_small }\OperatorTok\StringTok{ }\KeywordTok{count}\NormalTok{(s3classes) }\OperatorTok\StringTok{ }\KeywordTok{as_factor}\NormalTok{()}
\end{Highlighting}
\end{Shaded}

\begin{itemize}
\tightlist
\item
  I like \texttt{count()} because the default setting is to show
  \texttt{NA} values too!
\end{itemize}

\begin{Shaded}
\begin{Highlighting}[]
\NormalTok{hsls_small }\OperatorTok\StringTok{ }\KeywordTok{mutate}\NormalTok{(}\DataTypeTok{s3classes_na=}\KeywordTok{ifelse}\NormalTok{(s3classes}\OperatorTok{<}\DecValTok{0}\NormalTok{,}\OtherTok{NA}\NormalTok{,s3classes)) }\OperatorTok\StringTok{ }
\StringTok{  }\KeywordTok{count}\NormalTok{(s3classes_na)}
\end{Highlighting}
\end{Shaded}

Show both values and value labels on count tables for
\texttt{class==labelled}

\begin{Shaded}
\begin{Highlighting}[]
\CommentTok{#CRYSTAL'S SOLUTION}
\NormalTok{x <-}\StringTok{ }\NormalTok{hsls_small }\OperatorTok\StringTok{ }\KeywordTok{count}\NormalTok{(s3classes)}
\NormalTok{y <-}\StringTok{ }\NormalTok{hsls_small }\OperatorTok\StringTok{ }\KeywordTok{count}\NormalTok{(s3classes) }\OperatorTok\StringTok{ }\KeywordTok{as_factor}\NormalTok{()}
\KeywordTok{bind_cols}\NormalTok{(x[,}\DecValTok{1}\NormalTok{], y)}
\end{Highlighting}
\end{Shaded}

\end{frame}

\begin{frame}[fragile]{One-way descriptive stats for
discrete/categorical vars, Tidyverse approach}

For variables where \texttt{class==factor} {[}PROBLEM: HOW TO RETAIN
FACTOR LEVELS AFTER MUTATE{]}

\begin{Shaded}
\begin{Highlighting}[]
\CommentTok{#use variable from the hsls data frame where vars are factors}
\KeywordTok{class}\NormalTok{(hsls_f}\OperatorTok{$}\NormalTok{s3classes)}
\KeywordTok{attributes}\NormalTok{(hsls_f}\OperatorTok{$}\NormalTok{s3classes)}

\CommentTok{#show frequency table}
\NormalTok{hsls_f }\OperatorTok\StringTok{ }\KeywordTok{count}\NormalTok{(s3classes)}
\CommentTok{#frequency table with NAs}
  \CommentTok{#note: within ifelse() used levels(s3classes)[s3classes]) rather than s3classes  to show attribute levels not values}
\NormalTok{hsls_f }\OperatorTok\StringTok{ }\KeywordTok{mutate}\NormalTok{(}\DataTypeTok{s3classes_f=}\KeywordTok{ifelse}\NormalTok{(s3classes }\OperatorTok\StringTok{ }\KeywordTok{c}\NormalTok{(}\StringTok{"Missing"}\NormalTok{,}\StringTok{"Unit non-response"}\NormalTok{),}\OtherTok{NA}\NormalTok{,}\KeywordTok{levels}\NormalTok{(s3classes)[s3classes])) }\OperatorTok\StringTok{ }
\StringTok{  }\KeywordTok{count}\NormalTok{(s3classes_f)}
\end{Highlighting}
\end{Shaded}

\end{frame}

\begin{frame}[fragile]{Relationship between variables, categorical by
categorical}

Two-way frequency table, sometimes called ``Cross tabulation'', is
important for checking data quality - When you create categorical
analysis var from single categorical ``input'' var - Two-way tables show
us whether we did this correctly - Two-way tables helpful for
understanding skip patterns in surveys

Task: Create a two-way table between \texttt{s3classes} and
\texttt{s3clglvl}

\begin{Shaded}
\begin{Highlighting}[]
\NormalTok{hsls_small }\OperatorTok\StringTok{ }\KeywordTok{select}\NormalTok{(s3classes,s3clglvl) }\OperatorTok\StringTok{ }\KeywordTok{var_label}\NormalTok{()}

\NormalTok{hsls_small }\OperatorTok\StringTok{ }\KeywordTok{group_by}\NormalTok{(s3classes) }\OperatorTok\StringTok{ }\KeywordTok{count}\NormalTok{(s3clglvl)}
\NormalTok{hsls_small }\OperatorTok\StringTok{ }\KeywordTok{group_by}\NormalTok{(s3classes) }\OperatorTok\StringTok{ }\KeywordTok{count}\NormalTok{(s3clglvl) }\OperatorTok\StringTok{ }\KeywordTok{as_factor}\NormalTok{()}
\end{Highlighting}
\end{Shaded}

What if one of the variables has \texttt{NAs}?

\begin{itemize}
\tightlist
\item
  Table created by \texttt{group\_by()} and \texttt{count()} shows
  \texttt{NAs}!
\end{itemize}

\begin{Shaded}
\begin{Highlighting}[]
\NormalTok{hsls_small }\OperatorTok\StringTok{ }\KeywordTok{select}\NormalTok{(s3classes,s3clglvl) }\OperatorTok
\StringTok{  }\KeywordTok{mutate}\NormalTok{(}\DataTypeTok{s3classes_na=}\KeywordTok{ifelse}\NormalTok{(s3classes}\OperatorTok{<}\DecValTok{0}\NormalTok{,}\OtherTok{NA}\NormalTok{,s3classes)) }\OperatorTok
\StringTok{  }\KeywordTok{group_by}\NormalTok{(s3classes_na) }\OperatorTok\StringTok{ }\KeywordTok{count}\NormalTok{(s3clglvl)}

\NormalTok{hsls_small }\OperatorTok\StringTok{ }\KeywordTok{select}\NormalTok{(s3classes,s3clglvl) }\OperatorTok
\StringTok{  }\KeywordTok{mutate}\NormalTok{(}\DataTypeTok{s3classes_na=}\KeywordTok{ifelse}\NormalTok{(s3classes}\OperatorTok{<}\DecValTok{0}\NormalTok{,}\OtherTok{NA}\NormalTok{,s3classes),}
         \DataTypeTok{s3clglvl_na=}\KeywordTok{ifelse}\NormalTok{(s3clglvl}\OperatorTok{==-}\DecValTok{7}\NormalTok{,}\OtherTok{NA}\NormalTok{,s3clglvl)) }\OperatorTok
\StringTok{  }\KeywordTok{group_by}\NormalTok{(s3classes_na) }\OperatorTok\StringTok{ }\KeywordTok{count}\NormalTok{(s3clglvl_na)}
\end{Highlighting}
\end{Shaded}

\end{frame}

\begin{frame}[fragile]{Relationship between variables, categorical by
categorical}

Tables above are pretty ugly

Use the \texttt{spread()} function from \texttt{tidyr} package to create
table with one variable as columns and the other variable as rows

\begin{itemize}
\tightlist
\item
  The variable you place in \texttt{spread()} will be columns
\end{itemize}

\begin{Shaded}
\begin{Highlighting}[]
\NormalTok{hsls_small }\OperatorTok\StringTok{ }\KeywordTok{group_by}\NormalTok{(s3classes) }\OperatorTok\StringTok{ }\KeywordTok{count}\NormalTok{(s3clglvl) }\OperatorTok\StringTok{ }
\StringTok{  }\KeywordTok{spread}\NormalTok{(s3classes, n)}

\NormalTok{hsls_small }\OperatorTok\StringTok{ }\KeywordTok{group_by}\NormalTok{(s3classes) }\OperatorTok\StringTok{ }\KeywordTok{count}\NormalTok{(s3clglvl) }\OperatorTok\StringTok{ }
\StringTok{  }\KeywordTok{as_factor}\NormalTok{() }\OperatorTok\StringTok{  }\KeywordTok{spread}\NormalTok{(s3classes, n)}
\NormalTok{hsls_small }\OperatorTok\StringTok{ }\KeywordTok{group_by}\NormalTok{(s3classes) }\OperatorTok\StringTok{ }\KeywordTok{count}\NormalTok{(s3clglvl) }\OperatorTok\StringTok{ }
\StringTok{  }\KeywordTok{as_factor}\NormalTok{() }\OperatorTok\StringTok{  }\KeywordTok{spread}\NormalTok{(s3clglvl, n)}
\end{Highlighting}
\end{Shaded}

\end{frame}

\begin{frame}[fragile]{Relationship between variables, categorical by
continuous}

Investigating relationship between multiple variables is a little
tougher when at least one of the variables is continuous

One approach is the \textbf{conditional mean}:

\begin{itemize}
\tightlist
\item
  Shows average values of continous variables within groups
\item
  Groups are defined by your categorical variable(s)
\end{itemize}

Relationship to regression

\begin{itemize}
\tightlist
\item
  Conditional mean is similar to regression with a continuous dependent
  variable and a categorical X variable
\end{itemize}

Task:

\begin{itemize}
\tightlist
\item
  Investigate the relationship between math test score,
  \texttt{x2txmtscor}, and parental education, \texttt{x2paredu}
\end{itemize}

\begin{Shaded}
\begin{Highlighting}[]
\CommentTok{#first, investigate parental education}
\NormalTok{hsls_small }\OperatorTok\StringTok{ }\KeywordTok{count}\NormalTok{(x2paredu)}
\NormalTok{hsls_small }\OperatorTok\StringTok{ }\KeywordTok{count}\NormalTok{(x2paredu) }\OperatorTok\StringTok{ }\NormalTok{as_factor}

\NormalTok{## using dplyr to get average math score by parental education level}
\NormalTok{hsls_small }\OperatorTok
\StringTok{    }\KeywordTok{group_by}\NormalTok{(x2paredu) }\OperatorTok
\StringTok{    }\KeywordTok{summarise_at}\NormalTok{(}\DataTypeTok{.vars =} \KeywordTok{vars}\NormalTok{(x2txmtscor),}
                 \DataTypeTok{.funs =} \KeywordTok{funs}\NormalTok{(mean, }\DataTypeTok{.args =} \KeywordTok{list}\NormalTok{(}\DataTypeTok{na.rm =} \OtherTok{TRUE}\NormalTok{))) }\OperatorTok
\StringTok{    }\KeywordTok{as_factor}\NormalTok{()}
\end{Highlighting}
\end{Shaded}

\end{frame}

\begin{frame}[fragile]{Relationship between variables, categorical by
continuous}

Task: Investigate the relationship between math test score,
\texttt{x2txmtscor}, and parental education, \texttt{x2paredu}

For checking data quality, helpful to calculate other stats besides mean

\begin{Shaded}
\begin{Highlighting}[]
\NormalTok{hsls_small }\OperatorTok\StringTok{ }\KeywordTok{group_by}\NormalTok{(x2paredu) }\OperatorTok
\StringTok{    }\KeywordTok{summarise_at}\NormalTok{(}\DataTypeTok{.vars =} \KeywordTok{vars}\NormalTok{(x2txmtscor),}
                 \DataTypeTok{.funs =} \KeywordTok{funs}\NormalTok{(mean, min, max, }\DataTypeTok{.args =} \KeywordTok{list}\NormalTok{(}\DataTypeTok{na.rm =} \OtherTok{TRUE}\NormalTok{))) }\OperatorTok
\StringTok{    }\KeywordTok{as_factor}\NormalTok{()}
\end{Highlighting}
\end{Shaded}

Always Investigate presence of missing/skip values

\begin{Shaded}
\begin{Highlighting}[]
\NormalTok{hsls_small }\OperatorTok\StringTok{ }\KeywordTok{filter}\NormalTok{(x2paredu}\OperatorTok{<}\DecValTok{0}\NormalTok{) }\OperatorTok\StringTok{ }\KeywordTok{count}\NormalTok{(x2paredu)}
\NormalTok{hsls_small }\OperatorTok\StringTok{ }\KeywordTok{filter}\NormalTok{(x2txmtscor}\OperatorTok{<}\DecValTok{0}\NormalTok{) }\OperatorTok\StringTok{ }\KeywordTok{count}\NormalTok{(x2txmtscor)}
\end{Highlighting}
\end{Shaded}

Replace \texttt{-8} with \texttt{NA} and re-calculate conditional stats

\begin{Shaded}
\begin{Highlighting}[]
\NormalTok{hsls_small }\OperatorTok\StringTok{ }\KeywordTok{select}\NormalTok{(x2paredu,x2txmtscor) }\OperatorTok
\StringTok{  }\KeywordTok{mutate}\NormalTok{(}\DataTypeTok{x2paredu_na=}\KeywordTok{ifelse}\NormalTok{(x2paredu}\OperatorTok{<}\DecValTok{0}\NormalTok{,}\OtherTok{NA}\NormalTok{,x2paredu),}
         \DataTypeTok{x2txmtscor_na=}\KeywordTok{ifelse}\NormalTok{(x2txmtscor}\OperatorTok{<}\DecValTok{0}\NormalTok{,}\OtherTok{NA}\NormalTok{,x2txmtscor)) }\OperatorTok\StringTok{ }
\StringTok{  }\KeywordTok{group_by}\NormalTok{(x2paredu_na) }\OperatorTok
\StringTok{  }\KeywordTok{summarise_at}\NormalTok{(}\DataTypeTok{.vars =} \KeywordTok{vars}\NormalTok{(x2txmtscor_na),}
               \DataTypeTok{.funs =} \KeywordTok{funs}\NormalTok{(mean, min, max, }\DataTypeTok{.args =} \KeywordTok{list}\NormalTok{(}\DataTypeTok{na.rm =} \OtherTok{TRUE}\NormalTok{))) }\OperatorTok
\StringTok{  }\KeywordTok{as_factor}\NormalTok{()}
\end{Highlighting}
\end{Shaded}

\end{frame}

\begin{frame}[fragile]{Relationship between variables, categorical by
continuous}

{[}MAKE THIS A STUDENT EXERCISE?{]}

Can use same approach to calculate conditional mean by multiple
\texttt{group\_by()} variables

\begin{itemize}
\tightlist
\item
  Just add additional variables within \texttt{group\_by()}
\end{itemize}

Task: Calculate mean math test score (\texttt{x2txmtscor}), for each
combination of parental education (\texttt{x2paredu}) and sex
(\texttt{x2sex})

\begin{Shaded}
\begin{Highlighting}[]
\NormalTok{hsls_small }\OperatorTok
\StringTok{    }\KeywordTok{group_by}\NormalTok{(x2paredu,x2sex) }\OperatorTok
\StringTok{    }\KeywordTok{summarise_at}\NormalTok{(}\DataTypeTok{.vars =} \KeywordTok{vars}\NormalTok{(x2txmtscor),}
                 \DataTypeTok{.funs =} \KeywordTok{funs}\NormalTok{(mean, }\DataTypeTok{.args =} \KeywordTok{list}\NormalTok{(}\DataTypeTok{na.rm =} \OtherTok{TRUE}\NormalTok{))) }\OperatorTok
\StringTok{    }\KeywordTok{as_factor}\NormalTok{()}
\end{Highlighting}
\end{Shaded}

\end{frame}

\subsection{Guidelines for EDA}\label{guidelines-for-eda}

\begin{frame}{Guidelines for ``EDA for data quality''}

Assme that your goal in ``EDA for data quality'' is to investigate
``input'' data sources and create ``analysis variables''

\begin{itemize}
\tightlist
\item
  Usually, your analysis dataset will incorporate multiple sources of
  input data, including data you collect (primary data) and/or data
  collected by others (secondary data)
\end{itemize}

While this is not a linear process, these are the broad steps I follow

\begin{enumerate}
\def\labelenumi{\arabic{enumi}.}
\tightlist
\item
  Understand how inout data sources were created

  \begin{itemize}
  \tightlist
  \item
    e.g., when working with survey data, have survey questionnaire and
    codebooks on hand
  \end{itemize}
\item
  For each input data source, identify the ``unit of analysis'' and
  which combination of variables uniquely identify observations
\item
  Investigate patterns in input variables
\item
  Create analysis variable from input variable(s)
\item
  Verify that analysis variable is created correctly through descriptive
  statistics that compare values of input variable(s) against values of
  the analysis variable
\end{enumerate}

\textbf{Always be aware of missing values}

\end{frame}

\begin{frame}{``Unit of analysis'' and which variables uniquely identify
observations}

``Unit of analysis'' refers to ``what does each observation represent''
in an input data source

\begin{itemize}
\tightlist
\item
  If each obs represents a student, you have ``student level data''
\item
  If each obs represents a student-course, you have ``student-course
  level data''
\item
  If each obs represents a school, you have ``school-level data''
\item
  If each obs represents a school-year, you have ``school-year level
  data''
\end{itemize}

How to identify unit of analysis

\begin{itemize}
\tightlist
\item
  data documentation
\item
  investigating the data set
\end{itemize}

\end{frame}

\begin{frame}[fragile]{``Unit of analysis'' and which variables uniquely
identify observations}

Identify the variable -- or group of vars -- that ``uniquely
identifies'' observations

\begin{itemize}
\tightlist
\item
  ``uniquely identifies observations'': each value of the var has a
  frequency count of \texttt{1}
\end{itemize}

This is important for many data management tasks

\begin{itemize}
\tightlist
\item
  e.g., merging data sources, ``tidying'' data
\end{itemize}

How to identify which variable(s) uniquely identify observations

\begin{itemize}
\tightlist
\item
  data documentation
\item
  investigating the data set
\end{itemize}

\begin{Shaded}
\begin{Highlighting}[]
\NormalTok{hsls_small }\OperatorTok\StringTok{ }\KeywordTok{group_by}\NormalTok{(stu_id) }\OperatorTok\StringTok{ }\CommentTok{# group_by our candidate}
\StringTok{  }\KeywordTok{mutate}\NormalTok{(}\DataTypeTok{n_per_id=}\KeywordTok{n}\NormalTok{()) }\OperatorTok\StringTok{ }\CommentTok{# calculate number of obs per group}
\StringTok{  }\KeywordTok{ungroup}\NormalTok{() }\OperatorTok\StringTok{ }\CommentTok{# ungroup the data}
\StringTok{  }\KeywordTok{count}\NormalTok{(n_per_id}\OperatorTok{==}\DecValTok{1}\NormalTok{) }\CommentTok{# count "true that only one obs per group"}
\CommentTok{#> # A tibble: 1 x 2}
\CommentTok{#>   `n_per_id == 1`     n}
\CommentTok{#>   <lgl>           <int>}
\CommentTok{#> 1 TRUE            23503}

\CommentTok{#hsls_small %>% count(stu_id) %>% filter(n> 1) ## Patricia's approach}
\CommentTok{#Karina approaches using asserthat package}
\CommentTok{#library(assertthat)}
\CommentTok{#stopifnot(length(unique(uga_pub$ncessch))==nrow(uga_pub))}
\CommentTok{#assert_that(any(duplicated(uga_pub, by=c("ncessch", "var2")))==FALSE)}
\end{Highlighting}
\end{Shaded}

\end{frame}

\begin{frame}{Rules for variable creation}

Rules I follow for variable creation

\begin{enumerate}
\def\labelenumi{\arabic{enumi}.}
\tightlist
\item
  \medskip Never modify ``input variable''; instead create new variable
  based on input variable(s)

  \begin{itemize}
  \tightlist
  \item
    Always keep input variables used to create new variables
  \end{itemize}
\item
  Investigate input variable(s) and relationship between input variables
\item
  Developing a plan for creation of analysis variable

  \begin{itemize}
  \tightlist
  \item
    e.g., for each possible value of input variables, what should value
    of analysis variable be?
  \end{itemize}
\item
  Write code to create analysis variable
\item
  Run descriptive checks to verify new variables are constructed
  correctly

  \begin{itemize}
  \tightlist
  \item
    Can ``comment out'' these checks, but don't delete them
  \end{itemize}
\item
  Document new variables with notes and labels
\end{enumerate}

\end{frame}

\begin{frame}[fragile]{Rules for variable creation}

Task: Create analysis fir ses qunitile called \texttt{sesq5} based on
\texttt{x4x2sesq5}

\begin{Shaded}
\begin{Highlighting}[]
\CommentTok{#investigate input variable}
\NormalTok{hsls_small }\OperatorTok\StringTok{ }\KeywordTok{select}\NormalTok{(x4x2sesq5) }\OperatorTok\StringTok{ }\KeywordTok{var_label}\NormalTok{()}
\NormalTok{hsls_small }\OperatorTok\StringTok{ }\KeywordTok{select}\NormalTok{(x4x2sesq5) }\OperatorTok\StringTok{ }\KeywordTok{val_labels}\NormalTok{()}
\NormalTok{hsls_small }\OperatorTok\StringTok{ }\KeywordTok{select}\NormalTok{(x4x2sesq5) }\OperatorTok\StringTok{ }\KeywordTok{count}\NormalTok{(x4x2sesq5)}
\NormalTok{hsls_small }\OperatorTok\StringTok{ }\KeywordTok{select}\NormalTok{(x4x2sesq5) }\OperatorTok\StringTok{ }\KeywordTok{count}\NormalTok{(x4x2sesq5) }\OperatorTok\StringTok{ }\KeywordTok{as_factor}\NormalTok{()}

\CommentTok{#create analysis variable}
\NormalTok{hsls_small_temp <-}\StringTok{ }\NormalTok{hsls_small }\OperatorTok\StringTok{ }
\StringTok{  }\KeywordTok{mutate}\NormalTok{(}\DataTypeTok{sesq5=}\KeywordTok{ifelse}\NormalTok{(x4x2sesq5}\OperatorTok{==-}\DecValTok{8}\NormalTok{,}\OtherTok{NA}\NormalTok{,x4x2sesq5)) }\CommentTok{# approach 1}
\NormalTok{hsls_small_temp <-}\StringTok{ }\NormalTok{hsls_small }\OperatorTok\StringTok{ }
\StringTok{  }\KeywordTok{mutate}\NormalTok{(}\DataTypeTok{sesq5=}\KeywordTok{ifelse}\NormalTok{(x4x2sesq5}\OperatorTok{<}\DecValTok{0}\NormalTok{,}\OtherTok{NA}\NormalTok{,x4x2sesq5)) }\CommentTok{# approach 1}

\CommentTok{#verify}
\NormalTok{hsls_small_temp }\OperatorTok\StringTok{ }\KeywordTok{group_by}\NormalTok{(x4x2sesq5) }\OperatorTok\StringTok{ }\KeywordTok{count}\NormalTok{(sesq5)}
\end{Highlighting}
\end{Shaded}

\end{frame}

\begin{frame}{How to be a good researcher, good research assistant (RA)
{[}CUT?{]}}

{[}keep or cut this section?{]}{[}hidden curriculum stuff; look at notes
from last time you taught Stata data mgt{]}

\begin{itemize}
\tightlist
\item
  Your advisor/PI getting \$ to support students depends on previous RAs
  doing a good job
\item
  Must master certain skills, but beyond that it is mindset and approach
  that matters
\end{itemize}

\end{frame}

\subsection{Skip patterns}\label{skip-patterns}

\begin{frame}[fragile]{What are skip patterns}

Pretty easy to create an analysis variable based on a single input
variable

Harder to create analysis variables based on multiple input variables

\begin{itemize}
\tightlist
\item
  When working with survey data, even seemingly simple analysis
  variables require multiple input variables due to ``skip patterns''
\end{itemize}

What are ``skip patterns''? {[}students answer{]}

\begin{itemize}
\tightlist
\item
  {[}?DELETE OR DELAY?{]} Response on a particular survey item
  determines whether respondent answers some set of subsequent questions
\item
  What are some examples of this?
\end{itemize}

Key to working with skip patterns

\begin{itemize}
\tightlist
\item
  Have the survey questionnaire on hand
\item
  Sometimes it appears that analysis variable requires only one input
  variable, but really depends on several input variables because of
  skip patterns

  \begin{itemize}
  \tightlist
  \item
    Don't just blindly turn ``missing'' and ``skips'' from survey data
    to \texttt{NAs} in your analysis variable
  \item
    Rahter, trace why these ``missing'' and ``skips'' appear and decide
    how they should be coded in your analysis variable
  \end{itemize}
\end{itemize}

\end{frame}

\begin{frame}[fragile]{Creating analysis variables in the presence of
skip patterns}

Task: Create a measure of ``level'' of postsecondary institution
attended in 2013 from HSLS:09 survey data

\begin{itemize}
\tightlist
\item
  ``level'' is highest award-level of the postsecondary institution

  \begin{itemize}
  \tightlist
  \item
    e.g., if highest award is associate's degree (a two-year degree),
    then `level==2'
  \end{itemize}
\item
  The measure, \texttt{pselev2013}, should have following
  {[}non-missing{]} values:

  \begin{enumerate}
  \def\labelenumi{\arabic{enumi}.}
  \tightlist
  \item
    Not attending postsecondary education institution
  \item
    Attending a 2-year or less-than-2-year institution
  \item
    Attending 4-year or greater-than-4year institution
  \end{enumerate}
\end{itemize}

Background info:

\begin{itemize}
\tightlist
\item
  In ``2013 Update'' of HSLS:09, students asked about college attendance

  \begin{itemize}
  \tightlist
  \item
    Variables from student responses to ``2013 Update'' have prefix
    \texttt{s3}
  \end{itemize}
\item
  Survey questionnaire for 2013 update can be found
  \href{https://nces.ed.gov/surveys/hsls09/pdf/2013_Student_Parent_Questionnaire.pdf}{HERE}
\item
  The ``online codebook'' website
  \href{https://nces.ed.gov/onlinecodebook}{HERE} has info about
  specific variables
\item
  Measure has 3 input variables {[}usually must figure this out
  yourself{]}:

  \begin{enumerate}
  \def\labelenumi{\arabic{enumi}.}
  \tightlist
  \item
    \texttt{x3sqstat}: ``X3 Student questionnaire status''
  \item
    \texttt{s3classes}: ``S3 B01A Taking postsecondary classes as of Nov
    1 2013''
  \item
    \texttt{s3clglvl}: ``S3 Enrolled college IPEDS level''
  \end{enumerate}
\end{itemize}

\begin{Shaded}
\begin{Highlighting}[]
\NormalTok{hsls_small }\OperatorTok\StringTok{ }\KeywordTok{select}\NormalTok{(x3sqstat,s3classes,s3clglvl) }\OperatorTok\StringTok{ }\KeywordTok{var_label}\NormalTok{()}
\end{Highlighting}
\end{Shaded}

You won't have time to complete this task, but develop a plan for the
task and get as far as you can

\end{frame}

\begin{frame}[fragile]{Creating analysis variables in the presence of
skip patterns}

Step 1a: Investigate each input variable separately

\begin{Shaded}
\begin{Highlighting}[]
\CommentTok{#variable labels}
\NormalTok{hsls_small }\OperatorTok\StringTok{ }\KeywordTok{select}\NormalTok{(x3sqstat,s3classes,s3clglvl) }\OperatorTok\StringTok{ }\KeywordTok{var_label}\NormalTok{()}

\NormalTok{hsls_small }\OperatorTok\StringTok{ }\KeywordTok{count}\NormalTok{(x3sqstat)}
\NormalTok{hsls_small }\OperatorTok\StringTok{ }\KeywordTok{count}\NormalTok{(x3sqstat) }\OperatorTok\StringTok{ }\KeywordTok{as_factor}\NormalTok{()}

\NormalTok{hsls_small }\OperatorTok\StringTok{ }\KeywordTok{count}\NormalTok{(s3classes)}
\NormalTok{hsls_small }\OperatorTok\StringTok{ }\KeywordTok{count}\NormalTok{(s3classes) }\OperatorTok\StringTok{ }\KeywordTok{as_factor}\NormalTok{()}

\NormalTok{hsls_small }\OperatorTok\StringTok{ }\KeywordTok{count}\NormalTok{(s3clglvl)}
\NormalTok{hsls_small }\OperatorTok\StringTok{ }\KeywordTok{count}\NormalTok{(s3clglvl) }\OperatorTok\StringTok{ }\KeywordTok{as_factor}\NormalTok{()}
\end{Highlighting}
\end{Shaded}

\end{frame}

\begin{frame}[fragile]{Creating analysis variables in the presence of
skip patterns}

Step 1b: Investigate relationship between input variables

\begin{Shaded}
\begin{Highlighting}[]
\CommentTok{#x3sqstate and s3classes}
\NormalTok{hsls_small }\OperatorTok\StringTok{ }\KeywordTok{group_by}\NormalTok{(x3sqstat) }\OperatorTok\StringTok{ }\KeywordTok{count}\NormalTok{(s3classes) }
\NormalTok{hsls_small }\OperatorTok\StringTok{ }\KeywordTok{group_by}\NormalTok{(x3sqstat) }\OperatorTok\StringTok{ }\KeywordTok{count}\NormalTok{(s3classes) }\OperatorTok\StringTok{ }\KeywordTok{as_factor}\NormalTok{()}

\NormalTok{hsls_small }\OperatorTok\StringTok{ }\KeywordTok{filter}\NormalTok{(x3sqstat}\OperatorTok{==}\DecValTok{8}\NormalTok{) }\OperatorTok\StringTok{ }\KeywordTok{count}\NormalTok{(s3classes)}
\NormalTok{hsls_small }\OperatorTok\StringTok{ }\KeywordTok{filter}\NormalTok{(x3sqstat}\OperatorTok{==}\DecValTok{8}\NormalTok{) }\OperatorTok\StringTok{ }\KeywordTok{count}\NormalTok{(s3classes}\OperatorTok{==-}\DecValTok{8}\NormalTok{)}
\NormalTok{hsls_small }\OperatorTok\StringTok{ }\KeywordTok{filter}\NormalTok{(x3sqstat }\OperatorTok{!=}\DecValTok{8}\NormalTok{) }\OperatorTok\StringTok{ }\KeywordTok{count}\NormalTok{(s3classes)}

\CommentTok{#x3sqstate, s3classes and s3clglvl}
\NormalTok{hsls_small }\OperatorTok\StringTok{ }\KeywordTok{group_by}\NormalTok{(s3classes) }\OperatorTok\StringTok{ }\KeywordTok{count}\NormalTok{(s3clglvl) }
\NormalTok{hsls_small }\OperatorTok\StringTok{ }\KeywordTok{group_by}\NormalTok{(s3classes) }\OperatorTok\StringTok{ }\KeywordTok{count}\NormalTok{(s3clglvl) }\OperatorTok\StringTok{ }\KeywordTok{as_factor}\NormalTok{()}

\CommentTok{#add filter for whether student did not respond to X3 questionnaire}
\NormalTok{hsls_small }\OperatorTok\StringTok{ }\KeywordTok{filter}\NormalTok{(x3sqstat}\OperatorTok{==}\DecValTok{8}\NormalTok{) }\OperatorTok\StringTok{ }\KeywordTok{group_by}\NormalTok{(s3classes) }\OperatorTok\StringTok{ }\KeywordTok{count}\NormalTok{(s3clglvl) }
\NormalTok{hsls_small }\OperatorTok\StringTok{ }\KeywordTok{filter}\NormalTok{(x3sqstat }\OperatorTok{!=}\DecValTok{8}\NormalTok{) }\OperatorTok\StringTok{ }\KeywordTok{group_by}\NormalTok{(s3classes) }\OperatorTok\StringTok{ }\KeywordTok{count}\NormalTok{(s3clglvl)}

\CommentTok{#add filter for s3classes is "missing" [-9]}
\NormalTok{hsls_small }\OperatorTok\StringTok{ }\KeywordTok{filter}\NormalTok{(x3sqstat }\OperatorTok{!=}\DecValTok{8}\NormalTok{,s3classes}\OperatorTok{==-}\DecValTok{9}\NormalTok{) }\OperatorTok\StringTok{ }\KeywordTok{group_by}\NormalTok{(s3classes) }\OperatorTok\StringTok{ }\KeywordTok{count}\NormalTok{(s3clglvl)}
\NormalTok{hsls_small }\OperatorTok\StringTok{ }\KeywordTok{filter}\NormalTok{(x3sqstat }\OperatorTok{!=}\DecValTok{8}\NormalTok{,s3classes}\OperatorTok{!=-}\DecValTok{9}\NormalTok{) }\OperatorTok\StringTok{ }\KeywordTok{group_by}\NormalTok{(s3classes) }\OperatorTok\StringTok{ }\KeywordTok{count}\NormalTok{(s3clglvl)}

\CommentTok{#add filter for s3classes equal to "no" or "don't know"}
\NormalTok{hsls_small }\OperatorTok\StringTok{ }\KeywordTok{filter}\NormalTok{(x3sqstat }\OperatorTok{!=}\DecValTok{8}\NormalTok{,s3classes}\OperatorTok{!=-}\DecValTok{9}\NormalTok{, s3classes }\OperatorTok\StringTok{ }\KeywordTok{c}\NormalTok{(}\DecValTok{2}\NormalTok{,}\DecValTok{3}\NormalTok{)) }\OperatorTok\StringTok{ }\KeywordTok{group_by}\NormalTok{(s3classes) }\OperatorTok\StringTok{ }\KeywordTok{count}\NormalTok{(s3clglvl)}
\NormalTok{hsls_small }\OperatorTok\StringTok{ }\KeywordTok{filter}\NormalTok{(x3sqstat }\OperatorTok{!=}\DecValTok{8}\NormalTok{,s3classes}\OperatorTok{!=-}\DecValTok{9}\NormalTok{, s3classes }\OperatorTok\StringTok{ }\KeywordTok{c}\NormalTok{(}\DecValTok{2}\NormalTok{,}\DecValTok{3}\NormalTok{)) }\OperatorTok\StringTok{ }\KeywordTok{group_by}\NormalTok{(s3classes) }\OperatorTok\StringTok{ }\KeywordTok{count}\NormalTok{(s3clglvl) }\OperatorTok\StringTok{ }\KeywordTok{as_factor}\NormalTok{()}

\NormalTok{hsls_small }\OperatorTok\StringTok{ }\KeywordTok{filter}\NormalTok{(x3sqstat }\OperatorTok{!=}\DecValTok{8}\NormalTok{,s3classes}\OperatorTok{!=-}\DecValTok{9}\NormalTok{, s3classes}\OperatorTok{==}\DecValTok{1}\NormalTok{) }\OperatorTok\StringTok{ }\KeywordTok{group_by}\NormalTok{(s3classes) }\OperatorTok\StringTok{ }\KeywordTok{count}\NormalTok{(s3clglvl)}
\NormalTok{hsls_small }\OperatorTok\StringTok{ }\KeywordTok{filter}\NormalTok{(x3sqstat }\OperatorTok{!=}\DecValTok{8}\NormalTok{,s3classes}\OperatorTok{!=-}\DecValTok{9}\NormalTok{, s3classes}\OperatorTok{==}\DecValTok{1}\NormalTok{) }\OperatorTok\StringTok{ }\KeywordTok{group_by}\NormalTok{(s3classes) }\OperatorTok\StringTok{ }\KeywordTok{count}\NormalTok{(s3clglvl) }\OperatorTok\StringTok{ }\KeywordTok{as_factor}\NormalTok{()}
\end{Highlighting}
\end{Shaded}

\end{frame}

\section{Brainstorm next assignment: create GPA from course-level
data}\label{brainstorm-next-assignment-create-gpa-from-course-level-data}

\begin{frame}[fragile]{Brainstorm in your homework groups for next
assignment}

Assignment: create GPA variables from student-course level data

\begin{itemize}
\tightlist
\item
  Link to assignment HERE
\end{itemize}

Data source: \href{https://nces.ed.gov/surveys/nls72/}{National
Longitudinal Study of 1972 (NLS72)}

\begin{itemize}
\tightlist
\item
  Follows 12th graders from 1972
\item
  Data collection waves

  \begin{itemize}
  \tightlist
  \item
    Base year: 1972
  \item
    Follow-up surveys in: 1973, 1974, 1976, 1979, 1986
  \item
    Postsecondary transcripts collected in 1984
  \end{itemize}
\item
  Why use such an old survey for this assignment?

  \begin{itemize}
  \tightlist
  \item
    NLS72 predates data privacy agreements so postsecondary transcript
    data are publicly available
  \end{itemize}
\end{itemize}

\begin{Shaded}
\begin{Highlighting}[]
\CommentTok{#LOAD DATA}
\end{Highlighting}
\end{Shaded}

Work on the following in your homework groups:

\begin{itemize}
\tightlist
\item
  Read assignment
\item
  Conduct EDA investigations of input data

  \begin{itemize}
  \tightlist
  \item
    For homework assignment, you only need to focus on these variables
    {[}and can ignore the others{]}
  \end{itemize}
\item
  Develop a plan for how you will go about creating GPA variables
\end{itemize}

\end{frame}

\begin{frame}{Brainstorm in your homework groups for next assignment}

PUT CODE FOR EDA INVESTIGATIONS OF NLS72 HERE? OR PUT LINK TO .R SCRIPT
W/ NLS EDA INVESTIGATIONS HERE?

\end{frame}

\end{document}
