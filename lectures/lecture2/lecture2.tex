\documentclass[8pt,ignorenonframetext,]{beamer}
\setbeamertemplate{caption}[numbered]
\setbeamertemplate{caption label separator}{: }
\setbeamercolor{caption name}{fg=normal text.fg}
\beamertemplatenavigationsymbolsempty
\usepackage{lmodern}
\usepackage{amssymb,amsmath}
\usepackage{ifxetex,ifluatex}
\usepackage{fixltx2e} % provides \textsubscript
\ifnum 0\ifxetex 1\fi\ifluatex 1\fi=0 % if pdftex
  \usepackage[T1]{fontenc}
  \usepackage[utf8]{inputenc}
\else % if luatex or xelatex
  \ifxetex
    \usepackage{mathspec}
  \else
    \usepackage{fontspec}
  \fi
  \defaultfontfeatures{Ligatures=TeX,Scale=MatchLowercase}
\fi
% use upquote if available, for straight quotes in verbatim environments
\IfFileExists{upquote.sty}{\usepackage{upquote}}{}
% use microtype if available
\IfFileExists{microtype.sty}{%
\usepackage{microtype}
\UseMicrotypeSet[protrusion]{basicmath} % disable protrusion for tt fonts
}{}
\newif\ifbibliography
\hypersetup{
            pdftitle={Managing and Manipulating Data Using R},
            pdfauthor={Ozan Jaquette},
            pdfborder={0 0 0},
            breaklinks=true}
\urlstyle{same}  % don't use monospace font for urls
\usepackage{color}
\usepackage{fancyvrb}
\newcommand{\VerbBar}{|}
\newcommand{\VERB}{\Verb[commandchars=\\\{\}]}
\DefineVerbatimEnvironment{Highlighting}{Verbatim}{commandchars=\\\{\}}
% Add ',fontsize=\small' for more characters per line
\usepackage{framed}
\definecolor{shadecolor}{RGB}{248,248,248}
\newenvironment{Shaded}{\begin{snugshade}}{\end{snugshade}}
\newcommand{\KeywordTok}[1]{\textcolor[rgb]{0.13,0.29,0.53}{\textbf{#1}}}
\newcommand{\DataTypeTok}[1]{\textcolor[rgb]{0.13,0.29,0.53}{#1}}
\newcommand{\DecValTok}[1]{\textcolor[rgb]{0.00,0.00,0.81}{#1}}
\newcommand{\BaseNTok}[1]{\textcolor[rgb]{0.00,0.00,0.81}{#1}}
\newcommand{\FloatTok}[1]{\textcolor[rgb]{0.00,0.00,0.81}{#1}}
\newcommand{\ConstantTok}[1]{\textcolor[rgb]{0.00,0.00,0.00}{#1}}
\newcommand{\CharTok}[1]{\textcolor[rgb]{0.31,0.60,0.02}{#1}}
\newcommand{\SpecialCharTok}[1]{\textcolor[rgb]{0.00,0.00,0.00}{#1}}
\newcommand{\StringTok}[1]{\textcolor[rgb]{0.31,0.60,0.02}{#1}}
\newcommand{\VerbatimStringTok}[1]{\textcolor[rgb]{0.31,0.60,0.02}{#1}}
\newcommand{\SpecialStringTok}[1]{\textcolor[rgb]{0.31,0.60,0.02}{#1}}
\newcommand{\ImportTok}[1]{#1}
\newcommand{\CommentTok}[1]{\textcolor[rgb]{0.56,0.35,0.01}{\textit{#1}}}
\newcommand{\DocumentationTok}[1]{\textcolor[rgb]{0.56,0.35,0.01}{\textbf{\textit{#1}}}}
\newcommand{\AnnotationTok}[1]{\textcolor[rgb]{0.56,0.35,0.01}{\textbf{\textit{#1}}}}
\newcommand{\CommentVarTok}[1]{\textcolor[rgb]{0.56,0.35,0.01}{\textbf{\textit{#1}}}}
\newcommand{\OtherTok}[1]{\textcolor[rgb]{0.56,0.35,0.01}{#1}}
\newcommand{\FunctionTok}[1]{\textcolor[rgb]{0.00,0.00,0.00}{#1}}
\newcommand{\VariableTok}[1]{\textcolor[rgb]{0.00,0.00,0.00}{#1}}
\newcommand{\ControlFlowTok}[1]{\textcolor[rgb]{0.13,0.29,0.53}{\textbf{#1}}}
\newcommand{\OperatorTok}[1]{\textcolor[rgb]{0.81,0.36,0.00}{\textbf{#1}}}
\newcommand{\BuiltInTok}[1]{#1}
\newcommand{\ExtensionTok}[1]{#1}
\newcommand{\PreprocessorTok}[1]{\textcolor[rgb]{0.56,0.35,0.01}{\textit{#1}}}
\newcommand{\AttributeTok}[1]{\textcolor[rgb]{0.77,0.63,0.00}{#1}}
\newcommand{\RegionMarkerTok}[1]{#1}
\newcommand{\InformationTok}[1]{\textcolor[rgb]{0.56,0.35,0.01}{\textbf{\textit{#1}}}}
\newcommand{\WarningTok}[1]{\textcolor[rgb]{0.56,0.35,0.01}{\textbf{\textit{#1}}}}
\newcommand{\AlertTok}[1]{\textcolor[rgb]{0.94,0.16,0.16}{#1}}
\newcommand{\ErrorTok}[1]{\textcolor[rgb]{0.64,0.00,0.00}{\textbf{#1}}}
\newcommand{\NormalTok}[1]{#1}
\usepackage{longtable,booktabs}
\usepackage{caption}
% These lines are needed to make table captions work with longtable:
\makeatletter
\def\fnum@table{\tablename~\thetable}
\makeatother
\usepackage{graphicx,grffile}
\makeatletter
\def\maxwidth{\ifdim\Gin@nat@width>\linewidth\linewidth\else\Gin@nat@width\fi}
\def\maxheight{\ifdim\Gin@nat@height>\textheight0.8\textheight\else\Gin@nat@height\fi}
\makeatother
% Scale images if necessary, so that they will not overflow the page
% margins by default, and it is still possible to overwrite the defaults
% using explicit options in \includegraphics[width, height, ...]{}
\setkeys{Gin}{width=\maxwidth,height=\maxheight,keepaspectratio}

% Prevent slide breaks in the middle of a paragraph:
\widowpenalties 1 10000
\raggedbottom

\AtBeginPart{
  \let\insertpartnumber\relax
  \let\partname\relax
  \frame{\partpage}
}
\AtBeginSection{
  \ifbibliography
  \else
    \let\insertsectionnumber\relax
    \let\sectionname\relax
    \frame{\sectionpage}
  \fi
}
\AtBeginSubsection{
  \let\insertsubsectionnumber\relax
  \let\subsectionname\relax
  \frame{\subsectionpage}
}

\setlength{\parindent}{0pt}
\setlength{\parskip}{6pt plus 2pt minus 1pt}
\setlength{\emergencystretch}{3em}  % prevent overfull lines
\providecommand{\tightlist}{%
  \setlength{\itemsep}{0pt}\setlength{\parskip}{0pt}}
\setcounter{secnumdepth}{0}

%packages
\usepackage{graphicx}
\usepackage{rotating}
\usepackage{hyperref}

\usepackage{tikz} % used for text highlighting, amongst others
%title slide stuff
%\institute{Department of Education}
%\title{Managing and Manipulating Data Using R}

%
\setbeamertemplate{navigation symbols}{} % get rid of navigation icons:

%\setbeamertemplate{frametitle}{\thesection \hspace{0.2cm} \insertframetitle}
\setbeamertemplate{section in toc}[sections numbered]
\setbeamertemplate{subsection in toc}[subsections numbered]

%define colors
%\definecolor{uva_orange}{RGB}{216,141,42} % UVa orange (Rotunda orange)
\definecolor{mygray}{rgb}{0.95, 0.95, 0.95} % for highlighted text
	% grey is equal parts red, green, blue. higher values >> lighter grey
	%\definecolor{lightgraybo}{rgb}{0.83, 0.83, 0.83}

% new commands

%highlight text with very light grey
\newcommand*{\hlg}[1]{%
	\tikz[baseline=(X.base)] \node[rectangle, fill=mygray] (X) {#1};%
}
%, inner sep=0.3mm
%highlight text with very light grey and use font associated with code
\newcommand*{\hlgc}[1]{\texttt{\hlg{#1}}}

\title{Managing and Manipulating Data Using R}
\subtitle{Lecture 2}
\author{Ozan Jaquette}
\date{}

\begin{document}
\frame{\titlepage}

\begin{frame}
\tableofcontents[hideallsubsections]
\end{frame}

\begin{frame}

\end{frame}

\begin{frame}[fragile]{Libraries we will use today}

\begin{Shaded}
\begin{Highlighting}[]
\KeywordTok{library}\NormalTok{(tidyverse)}
\CommentTok{#> -- Attaching packages --------------------------------------------------------------------------------------------- tidyverse 1.2.1 --}
\CommentTok{#> v ggplot2 3.0.0     v purrr   0.2.5}
\CommentTok{#> v tibble  1.4.2     v dplyr   0.7.6}
\CommentTok{#> v tidyr   0.8.1     v stringr 1.3.1}
\CommentTok{#> v readr   1.1.1     v forcats 0.3.0}
\CommentTok{#> -- Conflicts ------------------------------------------------------------------------------------------------ tidyverse_conflicts() --}
\CommentTok{#> x dplyr::filter() masks stats::filter()}
\CommentTok{#> x dplyr::lag()    masks stats::lag()}
\end{Highlighting}
\end{Shaded}

\end{frame}

\section{Directories and filepaths {[}?lec1 or
lec2?{]}}\label{directories-and-filepaths-lec1-or-lec2}

\begin{frame}[fragile]{Working directory}

\textbf{(Current) Working directory}

\begin{itemize}
\tightlist
\item
  the folder/directory in which you are currently working
\item
  this is where R looks for files
\item
  Files located in your current working directory can be accessed
  without specifying a filepath because R automatically looks in this
  folder
\end{itemize}

Function \hlgc{getwd()} shows current working directory

\begin{Shaded}
\begin{Highlighting}[]
\KeywordTok{getwd}\NormalTok{()}
\CommentTok{#> [1] "C:/Users/ozanj/Documents/rclass/lectures/lecture2"}
\end{Highlighting}
\end{Shaded}

Command \hlgc{list.files()} lists all files located in working directory

\begin{Shaded}
\begin{Highlighting}[]
\KeywordTok{getwd}\NormalTok{()}
\CommentTok{#> [1] "C:/Users/ozanj/Documents/rclass/lectures/lecture2"}
\KeywordTok{list.files}\NormalTok{()}
\CommentTok{#> [1] "lecture2.pdf"          "lecture2.Rmd"          "lecture2.tex"         }
\CommentTok{#> [4] "text"                  "transform-logical.png"}
\end{Highlighting}
\end{Shaded}

\end{frame}

\begin{frame}[fragile]{Working directory, ``Code chunks'' vs.
``console'' and ``R scripts''}

When you run \textbf{code chunks} in RMarkdown files (.Rmd), the working
directory is set to the filepath where the .Rmd file is stored

\begin{Shaded}
\begin{Highlighting}[]
\KeywordTok{getwd}\NormalTok{()}
\CommentTok{#> [1] "C:/Users/ozanj/Documents/rclass/lectures/lecture2"}
\KeywordTok{list.files}\NormalTok{()}
\CommentTok{#> [1] "lecture2.pdf"          "lecture2.Rmd"          "lecture2.tex"         }
\CommentTok{#> [4] "text"                  "transform-logical.png"}
\end{Highlighting}
\end{Shaded}

When you run code from the \textbf{R Console} or an \textbf{R Script},
the working directory is\ldots{}.

\medskip I find this very annoying

Command \hlgc{getwd()} shows current working directory

\begin{Shaded}
\begin{Highlighting}[]
\KeywordTok{getwd}\NormalTok{()}
\CommentTok{#> [1] "C:/Users/ozanj/Documents/rclass/lectures/lecture2"}
\end{Highlighting}
\end{Shaded}

\end{frame}

\begin{frame}{Absolute vs.~relative filepath}

\end{frame}

\section{Investigating objects}\label{investigating-objects}

\begin{frame}[fragile]{Data on off-campus recruiting events by public
universities}

\begin{Shaded}
\begin{Highlighting}[]
\KeywordTok{rm}\NormalTok{(}\DataTypeTok{list =} \KeywordTok{ls}\NormalTok{()) }\CommentTok{# remove all objects}
\KeywordTok{getwd}\NormalTok{()}
\CommentTok{#> [1] "C:/Users/ozanj/Documents/rclass/lectures/lecture2"}

\CommentTok{#load dataset with one obs per recruiting event}
\KeywordTok{load}\NormalTok{(}\StringTok{"../../data/recruiting/recruit_event_somevars.Rdata"}\NormalTok{)}

\CommentTok{#load dataset with one obs per high school}
\KeywordTok{load}\NormalTok{(}\StringTok{"../../data/recruiting/recruit_school_somevars.Rdata"}\NormalTok{)}
\end{Highlighting}
\end{Shaded}

Object \textbackslash{}hlgc\{df\_event\}

\begin{itemize}
\tightlist
\item
  One observation per university, recruiting event
\end{itemize}

Object \textbackslash{}hlgc\{df\_event\}

\begin{itemize}
\tightlist
\item
  One observation per high school (visited and non-visited)
\end{itemize}

\end{frame}

\begin{frame}[fragile]{Listing objects}

\hlgc{ls()} function lists objects currently open in R

\begin{itemize}
\tightlist
\item
  This is different from files in current working directory
\end{itemize}

\begin{Shaded}
\begin{Highlighting}[]
\NormalTok{x <-}\StringTok{ "hello!"}
\KeywordTok{ls}\NormalTok{() }\CommentTok{# Objects open in R}
\CommentTok{#> [1] "df_event"  "df_school" "x"}
\KeywordTok{list.files}\NormalTok{() }\CommentTok{# files in working directory}
\CommentTok{#> [1] "lecture2.pdf"          "lecture2.Rmd"          "lecture2.tex"         }
\CommentTok{#> [4] "text"                  "transform-logical.png"}
\end{Highlighting}
\end{Shaded}

\hlgc{rm()} function removes specified objects open in R

\begin{Shaded}
\begin{Highlighting}[]
\KeywordTok{rm}\NormalTok{(x)}
\KeywordTok{ls}\NormalTok{()}
\CommentTok{#> [1] "df_event"  "df_school"}
\end{Highlighting}
\end{Shaded}

Command to remove all objects open in R (I don't run it)

\begin{Shaded}
\begin{Highlighting}[]
\KeywordTok{rm}\NormalTok{(}\DataTypeTok{list =} \KeywordTok{ls}\NormalTok{())}
\end{Highlighting}
\end{Shaded}

\end{frame}

\begin{frame}[fragile]{Describing objects, focus on \textbf{data
frames}}

Object \textbf{type} and \textbf{length}

\begin{Shaded}
\begin{Highlighting}[]
\KeywordTok{typeof}\NormalTok{(df_event)}
\CommentTok{#> [1] "list"}
\KeywordTok{length}\NormalTok{(df_event) }\CommentTok{# = num elements = num columns}
\CommentTok{#> [1] 28}
\end{Highlighting}
\end{Shaded}

Number of \textbf{rows} and \textbf{columns}

\begin{Shaded}
\begin{Highlighting}[]
\KeywordTok{nrow}\NormalTok{(df_event) }\CommentTok{# num rows = num observations}
\CommentTok{#> [1] 17976}
\KeywordTok{ncol}\NormalTok{(df_event) }\CommentTok{# num columns = num variables}
\CommentTok{#> [1] 28}
\KeywordTok{dim}\NormalTok{(df_event) }\CommentTok{# shows number rows by columns}
\CommentTok{#> [1] 17976    28}
\end{Highlighting}
\end{Shaded}

\end{frame}

\section{Variables names}\label{variables-names}

\begin{frame}{Introduction to the \hlgc{dplyr} library}

\hlgc{dplyr}, a package within the \hlgc{tidyverse} suite of packages,
provide tools for manipulating data frames

\begin{itemize}
\tightlist
\item
  Wickham describes functions within \hlgc{dplyr} as a set of ``verbs''
  that fall in the broader categories of \textbf{subsetting},
  \textbf{sorting}, and \textbf{transforming}
\end{itemize}

\begin{longtable}[]{@{}ll@{}}
\toprule
Today & Next two weeks\tabularnewline
\midrule
\endhead
\textbf{Subsetting data} & \textbf{Transforming data}\tabularnewline
- \hlgc{select()} variables & - \hlgc{mutate()} creates new
variables\tabularnewline
- \hlgc{filter()} observations & - \hlgc{summarize()} calculates across
rows\tabularnewline
\textbf{Sorting data} & - \hlgc{group\_by()} to calculate across rows
within groups\tabularnewline
- \hlgc{arrange()} &\tabularnewline
\bottomrule
\end{longtable}

All \hlgc{dplyr} verbs (i.e., functions) work as follows

\begin{enumerate}
\def\labelenumi{\arabic{enumi}.}
\tightlist
\item
  first argument is a data frame"
\item
  subsequent arguments describe what to do with variables and
  observations in data frame

  \begin{itemize}
  \tightlist
  \item
    refer to variable names without quotes
  \end{itemize}
\item
  result of the function is a new data frame
\end{enumerate}

\end{frame}

\begin{frame}[fragile]{Variable names}

\hlgc{names()} function lists names of elements in an object

\begin{itemize}
\tightlist
\item
  when object is a data frame, each element name is a variable name
\end{itemize}

\begin{Shaded}
\begin{Highlighting}[]
\CommentTok{#Output omitted}
\KeywordTok{names}\NormalTok{(df_event)}
\end{Highlighting}
\end{Shaded}

Refer to named elements in object using
\texttt{obj\_name\$element\_name}

\begin{itemize}
\tightlist
\item
  when object is a dataframe: \texttt{obj\_name\$varname}
\end{itemize}

\begin{Shaded}
\begin{Highlighting}[]
\KeywordTok{typeof}\NormalTok{(df_event}\OperatorTok{$}\NormalTok{instnm)}
\CommentTok{#> [1] "character"}
\KeywordTok{typeof}\NormalTok{(df_event}\OperatorTok{$}\NormalTok{avgmedian_inc_}\DecValTok{2564}\NormalTok{)}
\CommentTok{#> Warning: Unknown or uninitialised column: 'avgmedian_inc_2564'.}
\CommentTok{#> [1] "NULL"}
\end{Highlighting}
\end{Shaded}

This approach to isolating variables is very useful for investigating
and manipulating data

\end{frame}

\begin{frame}[fragile]{Rename variables}

\hlgc{rename()} function renames variables within a data frame object

\medskip Syntax:

\begin{itemize}
\tightlist
\item
  \hlgc{rename(obj\_name, new\_name = old\_name,\ldots)} \medskip
\end{itemize}

\begin{Shaded}
\begin{Highlighting}[]
\KeywordTok{rename}\NormalTok{(df_event, }\DataTypeTok{zip =}\NormalTok{ determined_zip, }\DataTypeTok{titlei =}\NormalTok{ titlei_status_pub)}
\KeywordTok{names}\NormalTok{(df_event)}
\end{Highlighting}
\end{Shaded}

Variable names do not change permanently unless we combine rename with
assignment \medskip

\begin{Shaded}
\begin{Highlighting}[]
\NormalTok{rename_event <-}\StringTok{ }\KeywordTok{rename}\NormalTok{(df_event, }\DataTypeTok{zip =}\NormalTok{ determined_zip, }\DataTypeTok{titlei =}\NormalTok{ titlei_status_pub)}
\KeywordTok{names}\NormalTok{(rename_event)}
\KeywordTok{rm}\NormalTok{(rename_event)}
\end{Highlighting}
\end{Shaded}

\end{frame}

\section{Selecting variables and printing
data}\label{selecting-variables-and-printing-data}

\begin{frame}[fragile]{Select variables using \hlgc{select()} function}

Printing observations is key to investigating data, but datasets often
have hundreds, thousands of variables

\hlgc{select()} function selects \textbf{columns} of data (i.e.,
variables) you specify

\begin{itemize}
\tightlist
\item
  See syntax in help file
\end{itemize}

Select \textbf{without assignment} simply prints data for selected
variables

\begin{Shaded}
\begin{Highlighting}[]
\KeywordTok{select}\NormalTok{(df_event,instnm,event_date,eventtype,event_state,med_inc)}
\CommentTok{#> # A tibble: 17,976 x 5}
\CommentTok{#>    instnm      event_date eventtype  event_state med_inc}
\CommentTok{#>  * <chr>       <fct>      <chr>      <fct>         <dbl>}
\CommentTok{#>  1 Stony Brook 2017-05-08 other      MA          149193 }
\CommentTok{#>  2 Stony Brook 2017-05-08 other      RI           33412 }
\CommentTok{#>  3 Stony Brook 2017-03-01 public hs  NJ          126918.}
\CommentTok{#>  4 Stony Brook 2017-03-15 public hs  NY          140832.}
\CommentTok{#>  5 Stony Brook 2017-01-26 public hs  NY          138843 }
\CommentTok{#>  6 Stony Brook 2017-04-06 private hs NJ          139421 }
\CommentTok{#>  7 Stony Brook 2017-03-08 private hs CT          135033 }
\CommentTok{#>  8 Stony Brook 2017-03-08 public hs  NY          117548 }
\CommentTok{#>  9 Stony Brook 2017-03-01 public hs  NY          122307 }
\CommentTok{#> 10 Stony Brook 2017-01-10 public hs  NY          159504.}
\CommentTok{#> # ... with 17,966 more rows}
\end{Highlighting}
\end{Shaded}

\end{frame}

\begin{frame}[fragile]{Select variables using \hlgc{select()} function}

Select \textbf{with assignment} creates a new object containing only the
variables you specify

\begin{Shaded}
\begin{Highlighting}[]
\NormalTok{event_small <-}\StringTok{ }\KeywordTok{select}\NormalTok{(df_event,instnm,event_date,eventtype,event_state,med_inc)}
\NormalTok{event_small}
\CommentTok{#> # A tibble: 17,976 x 5}
\CommentTok{#>    instnm      event_date eventtype  event_state med_inc}
\CommentTok{#>  * <chr>       <fct>      <chr>      <fct>         <dbl>}
\CommentTok{#>  1 Stony Brook 2017-05-08 other      MA          149193 }
\CommentTok{#>  2 Stony Brook 2017-05-08 other      RI           33412 }
\CommentTok{#>  3 Stony Brook 2017-03-01 public hs  NJ          126918.}
\CommentTok{#>  4 Stony Brook 2017-03-15 public hs  NY          140832.}
\CommentTok{#>  5 Stony Brook 2017-01-26 public hs  NY          138843 }
\CommentTok{#>  6 Stony Brook 2017-04-06 private hs NJ          139421 }
\CommentTok{#>  7 Stony Brook 2017-03-08 private hs CT          135033 }
\CommentTok{#>  8 Stony Brook 2017-03-08 public hs  NY          117548 }
\CommentTok{#>  9 Stony Brook 2017-03-01 public hs  NY          122307 }
\CommentTok{#> 10 Stony Brook 2017-01-10 public hs  NY          159504.}
\CommentTok{#> # ... with 17,966 more rows}
\end{Highlighting}
\end{Shaded}

\end{frame}

\begin{frame}[fragile]{Select}

We can also use ``helper functions'' \hlgc{starts\_with()},
\hlgc{contains()}, and \hlgc{ends\_with()} to choose columns

\begin{Shaded}
\begin{Highlighting}[]
\KeywordTok{names}\NormalTok{(df_event)}
\CommentTok{#>  [1] "instnm"               "univ_id"              "instst"              }
\CommentTok{#>  [4] "pid"                  "event_date"           "eventtype"           }
\CommentTok{#>  [7] "determined_zip"       "school_id"            "ipeds_id"            }
\CommentTok{#> [10] "event_state"          "event_inst"           "med_inc"             }
\CommentTok{#> [13] "pop_total"            "pct_white_zip"        "pct_black_zip"       }
\CommentTok{#> [16] "pct_asian_zip"        "pct_hispanic_zip"     "pct_amerindian_zip"  }
\CommentTok{#> [19] "pct_nativehawaii_zip" "pct_tworaces_zip"     "pct_otherrace_zip"   }
\CommentTok{#> [22] "fr_lunch"             "titlei_status_pub"    "total_12"            }
\CommentTok{#> [25] "school_type_pri"      "school_type_pub"      "g12offered"          }
\CommentTok{#> [28] "g12"}
\KeywordTok{select}\NormalTok{(df_event,instnm,}\KeywordTok{starts_with}\NormalTok{(}\StringTok{"event"}\NormalTok{))}
\CommentTok{#> # A tibble: 17,976 x 5}
\CommentTok{#>    instnm      event_date eventtype  event_state event_inst}
\CommentTok{#>  * <chr>       <fct>      <chr>      <fct>       <chr>     }
\CommentTok{#>  1 Stony Brook 2017-05-08 other      MA          Out-State }
\CommentTok{#>  2 Stony Brook 2017-05-08 other      RI          Out-State }
\CommentTok{#>  3 Stony Brook 2017-03-01 public hs  NJ          Out-State }
\CommentTok{#>  4 Stony Brook 2017-03-15 public hs  NY          In-State  }
\CommentTok{#>  5 Stony Brook 2017-01-26 public hs  NY          In-State  }
\CommentTok{#>  6 Stony Brook 2017-04-06 private hs NJ          Out-State }
\CommentTok{#>  7 Stony Brook 2017-03-08 private hs CT          Out-State }
\CommentTok{#>  8 Stony Brook 2017-03-08 public hs  NY          In-State  }
\CommentTok{#>  9 Stony Brook 2017-03-01 public hs  NY          In-State  }
\CommentTok{#> 10 Stony Brook 2017-01-10 public hs  NY          In-State  }
\CommentTok{#> # ... with 17,966 more rows}
\end{Highlighting}
\end{Shaded}

\end{frame}

\begin{frame}[fragile]{Exercise}

The data frame ``df\_school'' has one observation for each high school
and indicators for whether the high school received a recruiting visit.

\begin{Shaded}
\begin{Highlighting}[]
\KeywordTok{names}\NormalTok{(df_school)}
\end{Highlighting}
\end{Shaded}

\begin{enumerate}
\def\labelenumi{\arabic{enumi}.}
\tightlist
\item
  Use \hlgc{select()} to familiarize yourself with the data frame
\item
  Practice using the \hlgc{contains()} and \hlgc{ends\_with()} helper
  functions to to choose variables
\end{enumerate}

\end{frame}

\begin{frame}[fragile]{Viewing and printing data}

\begin{itemize}
\tightlist
\item
  Use the \hlgc{View()} function to view data in a browser
\end{itemize}

\begin{Shaded}
\begin{Highlighting}[]
\KeywordTok{View}\NormalTok{(df_event)}
\end{Highlighting}
\end{Shaded}

\begin{itemize}
\tightlist
\item
  \hlgc{head()} to show the first \emph{n} rows
\end{itemize}

\begin{Shaded}
\begin{Highlighting}[]
\KeywordTok{head}\NormalTok{(df_event, }\DataTypeTok{n=}\DecValTok{5}\NormalTok{)}
\end{Highlighting}
\end{Shaded}

\end{frame}

\begin{frame}[fragile]{Viewing and printing data}

Use
\texttt{\{obj\_name{[}\textless{}rows\textgreater{},\textless{}cols\textgreater{}{]}}
to print specific rows and columns of a data frame

\begin{itemize}
\tightlist
\item
  particularly powerful when combined with sequences (e.g., \hlgc{1:10})
\end{itemize}

\medskip Examples:

\begin{itemize}
\tightlist
\item
  Print first five rows
\end{itemize}

\begin{Shaded}
\begin{Highlighting}[]
\NormalTok{df_event[}\DecValTok{1}\OperatorTok{:}\DecValTok{5}\NormalTok{, ]}
\end{Highlighting}
\end{Shaded}

\begin{itemize}
\tightlist
\item
  Print first five rows and first three columns
\end{itemize}

\begin{Shaded}
\begin{Highlighting}[]
\NormalTok{df_event[}\DecValTok{1}\OperatorTok{:}\DecValTok{5}\NormalTok{, }\DecValTok{1}\OperatorTok{:}\DecValTok{3}\NormalTok{]}
\end{Highlighting}
\end{Shaded}

\begin{itemize}
\tightlist
\item
  Print first three columns of the 100th observation
\end{itemize}

\begin{Shaded}
\begin{Highlighting}[]
\NormalTok{df_event[}\DecValTok{100}\NormalTok{, }\DecValTok{1}\OperatorTok{:}\DecValTok{3}\NormalTok{]}
\end{Highlighting}
\end{Shaded}

Print the 50th observation, all variables

\begin{Shaded}
\begin{Highlighting}[]
\NormalTok{df_event[}\DecValTok{50}\NormalTok{,]}
\end{Highlighting}
\end{Shaded}

\end{frame}

\begin{frame}[fragile]{Viewing and printing data}

\begin{itemize}
\tightlist
\item
  type \hlgc{obj\_name\textdollar var\_name} to print obs for a variable
\end{itemize}

\begin{Shaded}
\begin{Highlighting}[]
\NormalTok{df_event}\OperatorTok{$}\NormalTok{event_state}
\end{Highlighting}
\end{Shaded}

\begin{itemize}
\tightlist
\item
  can be combined with sequences
\end{itemize}

\begin{Shaded}
\begin{Highlighting}[]
\NormalTok{df_event}\OperatorTok{$}\NormalTok{event_state[}\DecValTok{1}\OperatorTok{:}\DecValTok{10}\NormalTok{]}
\CommentTok{#>  [1] MA RI NJ NY NY NJ CT NY NY NY}
\CommentTok{#> 48 Levels: AL AR AZ CA CO CT DC DE FL GA HI IA ID IL IN KS KY LA MA ... WV}
\NormalTok{df_event}\OperatorTok{$}\NormalTok{eventtype[}\DecValTok{6}\OperatorTok{:}\DecValTok{10}\NormalTok{]}
\CommentTok{#> [1] "private hs" "private hs" "public hs"  "public hs"  "public hs"}
\end{Highlighting}
\end{Shaded}

\begin{itemize}
\tightlist
\item
  can also print multiple variables using \hlgc{combine()} function
\end{itemize}

\begin{Shaded}
\begin{Highlighting}[]
\KeywordTok{c}\NormalTok{(df_event}\OperatorTok{$}\NormalTok{event_state[}\DecValTok{1}\OperatorTok{:}\DecValTok{5}\NormalTok{],df_event}\OperatorTok{$}\NormalTok{eventtype[}\DecValTok{1}\OperatorTok{:}\DecValTok{5}\NormalTok{])}
\CommentTok{#>  [1] "19"        "38"        "30"        "33"        "33"       }
\CommentTok{#>  [6] "other"     "other"     "public hs" "public hs" "public hs"}
\end{Highlighting}
\end{Shaded}

\end{frame}

\begin{frame}{Exercise}

Create a printing exercise using the df\_school

\begin{enumerate}
\def\labelenumi{\arabic{enumi}.}
\tightlist
\item
  Use head() to print first 5 observations
\item
  Use select to print observations of variables of your choosing
\item
  Use \hlgc{obj\_name\textdollar var\_name[1:10]} to print the first 10
  observations of a variable
\item
  use combine()
\end{enumerate}

\end{frame}

\section{Filtering data}\label{filtering-data}

\begin{frame}[fragile]{Filter}

filter() allows you to select observations based on values of variables

\begin{itemize}
\tightlist
\item
  Arguments

  \begin{itemize}
  \tightlist
  \item
    first argument is name of data frame
  \item
    subsequent arguments are \emph{expressions} to filter the data frame
  \end{itemize}
\item
  What is the result of a \hlgc{filter()} command?

  \begin{itemize}
  \tightlist
  \item
    \hlgc{filter()} returns the rows where the condition is \hlgc{TRUE}
  \end{itemize}
\end{itemize}

\medskip Example using data frame object \hlgc{df\_school}, where each
observation is a high school

\begin{itemize}
\tightlist
\item
  Show all obs where the high school received 1 visit from UC Berkeley
  (110635) {[}output omitted{]}
\end{itemize}

\begin{Shaded}
\begin{Highlighting}[]
\KeywordTok{filter}\NormalTok{(df_school,visits_by_}\DecValTok{110635} \OperatorTok{==}\StringTok{ }\DecValTok{1}\NormalTok{)}
\end{Highlighting}
\end{Shaded}

\begin{itemize}
\tightlist
\item
  Must \textbf{assign} to create new object based on filter
\end{itemize}

\begin{Shaded}
\begin{Highlighting}[]
\NormalTok{berk_boulder <-}\StringTok{ }\KeywordTok{filter}\NormalTok{(df_school,visits_by_}\DecValTok{110635} \OperatorTok{==}\StringTok{ }\DecValTok{1}\NormalTok{, visits_by_}\DecValTok{126614}\OperatorTok{==}\DecValTok{1}\NormalTok{)}
\NormalTok{berk_boulder}
\end{Highlighting}
\end{Shaded}

\end{frame}

\begin{frame}[fragile]{Filter, character variables}

\begin{itemize}
\tightlist
\item
  Use quotes \hlgc{''} or \hlgc{""} to refer to character variables
\end{itemize}

\begin{Shaded}
\begin{Highlighting}[]
\CommentTok{#Berkeley}
\KeywordTok{filter}\NormalTok{(df_school,visits_by_}\DecValTok{110635} \OperatorTok{==}\StringTok{ }\DecValTok{1}\NormalTok{, school_type }\OperatorTok{==}\StringTok{ "private"}\NormalTok{, state_code }\OperatorTok{==}\StringTok{ "CA"}\NormalTok{)}
\CommentTok{#Bama}
\KeywordTok{filter}\NormalTok{(df_school,visits_by_}\DecValTok{100751} \OperatorTok{==}\StringTok{ }\DecValTok{1}\NormalTok{, school_type }\OperatorTok{==}\StringTok{ "private"}\NormalTok{, state_code }\OperatorTok{==}\StringTok{ "CA"}\NormalTok{) }

\CommentTok{#Berkeley and Bama}
\KeywordTok{filter}\NormalTok{(df_school,visits_by_}\DecValTok{100751} \OperatorTok{==}\StringTok{ }\DecValTok{1}\NormalTok{, visits_by_}\DecValTok{110635} \OperatorTok{==}\StringTok{ }\DecValTok{1}\NormalTok{, school_type }\OperatorTok{==}\StringTok{ "private"}\NormalTok{, state_code }\OperatorTok{==}\StringTok{ "CA"}\NormalTok{) }
\end{Highlighting}
\end{Shaded}

\end{frame}

\begin{frame}[fragile]{Logical operators for comparisons}

\begin{longtable}[]{@{}ll@{}}
\caption{Logical operators}\tabularnewline
\toprule
Symbol & Meaning\tabularnewline
\midrule
\endfirsthead
\toprule
Symbol & Meaning\tabularnewline
\midrule
\endhead
\texttt{==} & Equal to\tabularnewline
\texttt{!=} & Not equal to\tabularnewline
\texttt{\textgreater{}} & greater than\tabularnewline
\texttt{\textgreater{}=} & greater than or equal to\tabularnewline
\texttt{\textless{}} & less than\tabularnewline
\texttt{\textless{}=} & less than or equal to\tabularnewline
\texttt{\&} & AND\tabularnewline
\texttt{\textbar{}} & OR\tabularnewline
\texttt{\%in} & includes\tabularnewline
\bottomrule
\end{longtable}

\medskip 

\begin{figure}
\centering
\includegraphics[width=0.50000\textwidth]{transform-logical.png}
\caption{``Boolean'' operations, x=left circle, y=right circle, from
Wichkam (2018)}
\end{figure}

\end{frame}

\begin{frame}[fragile]{Filters and comparisons}

Schools visited by Bama (100751) and/or Berkeley (110635)

\begin{Shaded}
\begin{Highlighting}[]
\CommentTok{#berkeley and bama}
\KeywordTok{filter}\NormalTok{(df_school,visits_by_}\DecValTok{100751} \OperatorTok{>=}\StringTok{ }\DecValTok{1}\NormalTok{, visits_by_}\DecValTok{110635} \OperatorTok{>=}\StringTok{ }\DecValTok{1}\NormalTok{) }
\KeywordTok{filter}\NormalTok{(df_school,visits_by_}\DecValTok{100751} \OperatorTok{>=}\StringTok{ }\DecValTok{1} \OperatorTok{&}\StringTok{ }\NormalTok{visits_by_}\DecValTok{110635} \OperatorTok{>=}\StringTok{ }\DecValTok{1}\NormalTok{) }\CommentTok{# same same}
\CommentTok{#berkeley or bama}
\KeywordTok{filter}\NormalTok{(df_school,visits_by_}\DecValTok{100751} \OperatorTok{>=}\StringTok{ }\DecValTok{1} \OperatorTok{|}\StringTok{ }\NormalTok{visits_by_}\DecValTok{110635} \OperatorTok{>=}\StringTok{ }\DecValTok{1}\NormalTok{) }
\end{Highlighting}
\end{Shaded}

Apply \hlgc{count()} function on top of \hlgc{filter()} function to
count the number of observations that satisfy criteria

\begin{itemize}
\tightlist
\item
  Avoids printing individual observations
\end{itemize}

\begin{Shaded}
\begin{Highlighting}[]
\KeywordTok{count}\NormalTok{(}\KeywordTok{filter}\NormalTok{(df_school,visits_by_}\DecValTok{100751} \OperatorTok{>=}\StringTok{ }\DecValTok{1} \OperatorTok{&}\StringTok{ }\NormalTok{visits_by_}\DecValTok{110635} \OperatorTok{>=}\StringTok{ }\DecValTok{1}\NormalTok{))}
\CommentTok{#> # A tibble: 1 x 1}
\CommentTok{#>       n}
\CommentTok{#>   <int>}
\CommentTok{#> 1   247}
\KeywordTok{count}\NormalTok{(}\KeywordTok{filter}\NormalTok{(df_school,visits_by_}\DecValTok{100751} \OperatorTok{>=}\StringTok{ }\DecValTok{1} \OperatorTok{|}\StringTok{ }\NormalTok{visits_by_}\DecValTok{110635} \OperatorTok{>=}\StringTok{ }\DecValTok{1}\NormalTok{))}
\CommentTok{#> # A tibble: 1 x 1}
\CommentTok{#>       n}
\CommentTok{#>   <int>}
\CommentTok{#> 1  2763}
\end{Highlighting}
\end{Shaded}

\end{frame}

\begin{frame}[fragile]{Filters and comparisons, \textgreater{}=}

Number of public high schools that are at least 50\% Black in Alabama
compared to number of schools that received visit by Bama

\begin{Shaded}
\begin{Highlighting}[]
\CommentTok{#at least 50% black}
\KeywordTok{count}\NormalTok{(}\KeywordTok{filter}\NormalTok{(df_school, school_type }\OperatorTok{==}\StringTok{ "public"}\NormalTok{, pct_black }\OperatorTok{>=}\StringTok{ }\DecValTok{50}\NormalTok{, state_code }\OperatorTok{==}\StringTok{ "AL"}\NormalTok{))}
\CommentTok{#> # A tibble: 1 x 1}
\CommentTok{#>       n}
\CommentTok{#>   <int>}
\CommentTok{#> 1    86}
\KeywordTok{count}\NormalTok{(}\KeywordTok{filter}\NormalTok{(df_school, school_type }\OperatorTok{==}\StringTok{ "public"}\NormalTok{, pct_black }\OperatorTok{>=}\StringTok{ }\DecValTok{50}\NormalTok{, state_code }\OperatorTok{==}\StringTok{ "AL"}\NormalTok{, visits_by_}\DecValTok{100751} \OperatorTok{>=}\StringTok{ }\DecValTok{1}\NormalTok{))}
\CommentTok{#> # A tibble: 1 x 1}
\CommentTok{#>       n}
\CommentTok{#>   <int>}
\CommentTok{#> 1    21}

\CommentTok{#at least 50% white}
\KeywordTok{count}\NormalTok{(}\KeywordTok{filter}\NormalTok{(df_school, school_type }\OperatorTok{==}\StringTok{ "public"}\NormalTok{, pct_white }\OperatorTok{>=}\StringTok{ }\DecValTok{50}\NormalTok{, state_code }\OperatorTok{==}\StringTok{ "AL"}\NormalTok{))}
\CommentTok{#> # A tibble: 1 x 1}
\CommentTok{#>       n}
\CommentTok{#>   <int>}
\CommentTok{#> 1   238}
\KeywordTok{count}\NormalTok{(}\KeywordTok{filter}\NormalTok{(df_school, school_type }\OperatorTok{==}\StringTok{ "public"}\NormalTok{, pct_white }\OperatorTok{>=}\StringTok{ }\DecValTok{50}\NormalTok{, state_code }\OperatorTok{==}\StringTok{ "AL"}\NormalTok{, visits_by_}\DecValTok{100751} \OperatorTok{>=}\StringTok{ }\DecValTok{1}\NormalTok{))}
\CommentTok{#> # A tibble: 1 x 1}
\CommentTok{#>       n}
\CommentTok{#>   <int>}
\CommentTok{#> 1    82}
\end{Highlighting}
\end{Shaded}

\end{frame}

\begin{frame}[fragile]{Filters and comparisons, not equals (\hlgc{!=})}

Number of high schools visited by University of Colorado (126614) that
are not located in CO

\begin{Shaded}
\begin{Highlighting}[]

\CommentTok{#number of high schools visited by U Colorado}
\KeywordTok{count}\NormalTok{(}\KeywordTok{filter}\NormalTok{(df_school, visits_by_}\DecValTok{126614} \OperatorTok{>=}\StringTok{ }\DecValTok{1}\NormalTok{))}
\CommentTok{#> # A tibble: 1 x 1}
\CommentTok{#>       n}
\CommentTok{#>   <int>}
\CommentTok{#> 1  1056}

\CommentTok{#number of high schools visited by U Colorado not located in CO}
\KeywordTok{count}\NormalTok{(}\KeywordTok{filter}\NormalTok{(df_school, visits_by_}\DecValTok{126614} \OperatorTok{>=}\StringTok{ }\DecValTok{1}\NormalTok{, state_code }\OperatorTok{!=}\StringTok{ "CO"}\NormalTok{))}
\CommentTok{#> # A tibble: 1 x 1}
\CommentTok{#>       n}
\CommentTok{#>   <int>}
\CommentTok{#> 1   873}
\CommentTok{#number of high schools visited by U Colorado located in CO}
\CommentTok{#count(filter(df_school, visits_by_126614 >= 1, state_code == "CO"))}
\end{Highlighting}
\end{Shaded}

\end{frame}

\begin{frame}[fragile]{Filters and comparisons, \%in\% operator}

What if you wanted to count the number of schools visited by Bama
(100751) in a group of states?

\begin{Shaded}
\begin{Highlighting}[]
\KeywordTok{count}\NormalTok{(}\KeywordTok{filter}\NormalTok{(df_school,visits_by_}\DecValTok{100751} \OperatorTok{>=}\StringTok{ }\DecValTok{1}\NormalTok{, state_code }\OperatorTok{==}\StringTok{ "MA"} \OperatorTok{|}\StringTok{ }\NormalTok{state_code }\OperatorTok{==}\StringTok{ "VT"} \OperatorTok{|}\StringTok{ }\NormalTok{state_code }\OperatorTok{==}\StringTok{ "ME"}\NormalTok{))}
\CommentTok{#> # A tibble: 1 x 1}
\CommentTok{#>       n}
\CommentTok{#>   <int>}
\CommentTok{#> 1   108}
\end{Highlighting}
\end{Shaded}

Easier way to do this is with \texttt{\%in\%} operator

\begin{Shaded}
\begin{Highlighting}[]
\KeywordTok{count}\NormalTok{(}\KeywordTok{filter}\NormalTok{(df_school,visits_by_}\DecValTok{100751} \OperatorTok{>=}\StringTok{ }\DecValTok{1}\NormalTok{, state_code }\OperatorTok\StringTok{ }\KeywordTok{c}\NormalTok{(}\StringTok{"MA"}\NormalTok{,}\StringTok{"ME"}\NormalTok{,}\StringTok{"VT"}\NormalTok{)))}
\CommentTok{#> # A tibble: 1 x 1}
\CommentTok{#>       n}
\CommentTok{#>   <int>}
\CommentTok{#> 1   108}
\end{Highlighting}
\end{Shaded}

Select the private high schools that got either 2 or 3 visits from Bama

\begin{Shaded}
\begin{Highlighting}[]
\KeywordTok{count}\NormalTok{(}\KeywordTok{filter}\NormalTok{(df_school, visits_by_}\DecValTok{100751} \OperatorTok\StringTok{ }\DecValTok{2}\OperatorTok{:}\DecValTok{3}\NormalTok{, school_type }\OperatorTok{==}\StringTok{ "private"}\NormalTok{))}
\CommentTok{#> # A tibble: 1 x 1}
\CommentTok{#>       n}
\CommentTok{#>   <int>}
\CommentTok{#> 1   183}
\end{Highlighting}
\end{Shaded}

\end{frame}

\begin{frame}[fragile]{Identifying data type and possible values of
variable is helpful for filtering}

\begin{itemize}
\tightlist
\item
  \hlgc{class()} and \hlgc{str()} shows data type of a variable
\item
  \hlgc{table()} to show potential values of categorical variables
\end{itemize}

\begin{Shaded}
\begin{Highlighting}[]
\KeywordTok{class}\NormalTok{(df_event}\OperatorTok{$}\NormalTok{eventtype)}
\CommentTok{#> [1] "character"}
\KeywordTok{str}\NormalTok{(df_event}\OperatorTok{$}\NormalTok{eventtype)}
\CommentTok{#>  chr [1:17976] "other" "other" "public hs" "public hs" "public hs" ...}
\KeywordTok{table}\NormalTok{(df_event}\OperatorTok{$}\NormalTok{eventtype)}
\CommentTok{#> }
\CommentTok{#>              other other college/univ       PNP 4yr univ }
\CommentTok{#>               2107                 42                183 }
\CommentTok{#>         private hs         pub 2yr cc       pub 4yr univ }
\CommentTok{#>               3644                769                206 }
\CommentTok{#>          public hs }
\CommentTok{#>              11025}

\KeywordTok{class}\NormalTok{(df_event}\OperatorTok{$}\NormalTok{event_state)}
\CommentTok{#> [1] "factor"}
\KeywordTok{str}\NormalTok{(df_event}\OperatorTok{$}\NormalTok{event_state) }\CommentTok{# double quotes indicate character}
\CommentTok{#>  Factor w/ 48 levels "AL","AR","AZ",..: 19 38 30 33 33 30 6 33 33 33 ...}

\KeywordTok{class}\NormalTok{(df_event}\OperatorTok{$}\NormalTok{med_inc)}
\CommentTok{#> [1] "numeric"}
\KeywordTok{str}\NormalTok{(df_event}\OperatorTok{$}\NormalTok{med_inc)}
\CommentTok{#>  num [1:17976] 149193 33412 126919 140833 138843 ...}
\end{Highlighting}
\end{Shaded}

Now that we know \hlgc{eventtype} is a character, we can filter values

\begin{Shaded}
\begin{Highlighting}[]
\KeywordTok{count}\NormalTok{(}\KeywordTok{filter}\NormalTok{(df_event, eventtype }\OperatorTok{==}\StringTok{ "public hs"}\NormalTok{, event_state }\OperatorTok{==}\StringTok{"CA"}\NormalTok{))}
\CommentTok{#> # A tibble: 1 x 1}
\CommentTok{#>       n}
\CommentTok{#>   <int>}
\CommentTok{#> 1  1057}
\CommentTok{#below code would return an error because variables are character}
\CommentTok{#count(filter(df_event, eventtype == public hs, event_state ==CA))}
\end{Highlighting}
\end{Shaded}

\end{frame}

\begin{frame}{Exercises}

Use the data from df\_event, which has one observation for each
off-campus recruiting event a university attends

\begin{enumerate}
\def\labelenumi{\arabic{enumi}.}
\item
  Count the number of events attended by the University of Pittsburgh
  (Pitt) \hlgc{univ\_id == 215293}
\item
  Count the number of recruiting events by Pitt at public or private
  high schools
\item
  Count the number of recruiting events by Pitt at public or private
  high schools located in the state of PA
\item
  Count the number of recruiting events by Pitt at public high schools
  not located in PA where median income is less than 100,000
\item
  Count the number of recruiting events by Pitt at public high schools
  not located in PA where median income is greater than or equal to
  100,000
\item
  Count the number of out-of-state recruiting events by Pitt at private
  high schools or public high schools with median income of at least
  100,000
\end{enumerate}

\end{frame}

\section{Missing values}\label{missing-values}

\begin{frame}[fragile]{Missing values}

Missing values have the value \hlgc{NA}

\begin{itemize}
\tightlist
\item
  \hlgc{NA} is a special keyword, not the same as the character string
  \hlgc{"NA"}
\item
  use \hlgc{is.(na)} function to determine if a value is missing
\end{itemize}

\medskip

\begin{Shaded}
\begin{Highlighting}[]
\KeywordTok{is.na}\NormalTok{(}\DecValTok{5}\NormalTok{)}
\CommentTok{#> [1] FALSE}
\KeywordTok{is.na}\NormalTok{(}\OtherTok{NA}\NormalTok{)}
\CommentTok{#> [1] TRUE}
\KeywordTok{is.na}\NormalTok{(}\StringTok{"NA"}\NormalTok{)}
\CommentTok{#> [1] FALSE}

\NormalTok{nvector <-}\StringTok{ }\KeywordTok{c}\NormalTok{(}\DecValTok{10}\NormalTok{,}\DecValTok{5}\NormalTok{,}\OtherTok{NA}\NormalTok{)}
\NormalTok{nvector}
\CommentTok{#> [1] 10  5 NA}
\KeywordTok{is.na}\NormalTok{(nvector)}
\CommentTok{#> [1] FALSE FALSE  TRUE}

\NormalTok{svector <-}\StringTok{ }\KeywordTok{c}\NormalTok{(}\StringTok{"e"}\NormalTok{,}\StringTok{"f"}\NormalTok{,}\OtherTok{NA}\NormalTok{,}\StringTok{"NA"}\NormalTok{)}
\NormalTok{svector}
\CommentTok{#> [1] "e"  "f"  NA   "NA"}
\KeywordTok{is.na}\NormalTok{(svector)}
\CommentTok{#> [1] FALSE FALSE  TRUE FALSE}
\end{Highlighting}
\end{Shaded}

\end{frame}

\begin{frame}[fragile]{Missing values are ``contageous''}

What does ``contageous'' mean?

\begin{itemize}
\tightlist
\item
  operations involving a missing value will yield a missing value
\end{itemize}

\begin{Shaded}
\begin{Highlighting}[]
\DecValTok{7}\OperatorTok{>}\DecValTok{5}
\CommentTok{#> [1] TRUE}
\DecValTok{7}\OperatorTok{>}\OtherTok{NA}
\CommentTok{#> [1] NA}
\DecValTok{0}\OperatorTok{==}\OtherTok{NA}
\CommentTok{#> [1] NA}
\DecValTok{2}\OperatorTok{*}\KeywordTok{c}\NormalTok{(}\DecValTok{0}\NormalTok{,}\DecValTok{1}\NormalTok{,}\DecValTok{2}\NormalTok{,}\OtherTok{NA}\NormalTok{)}
\CommentTok{#> [1]  0  2  4 NA}
\OtherTok{NA}\OperatorTok{*}\KeywordTok{c}\NormalTok{(}\DecValTok{0}\NormalTok{,}\DecValTok{1}\NormalTok{,}\DecValTok{2}\NormalTok{,}\OtherTok{NA}\NormalTok{)}
\CommentTok{#> [1] NA NA NA NA}
\end{Highlighting}
\end{Shaded}

\end{frame}

\begin{frame}[fragile]{Function and missing values, the \hlgc{table()}
function}

Tip: command \hlgc{str(df\_event)} shows which variables have missing
values

\hlgc{table()} function useful for investigating categorical variables

\begin{itemize}
\tightlist
\item
  by default \hlgc{table()} ignores \hlgc{NA} values
\item
  the \hlgc{useNA} argument to include \hlgc{NA} values

  \begin{itemize}
  \tightlist
  \item
    from help file: ``useNA controls if the table includes counts of NA
    values: the allowed values correspond to never (''no``), only if the
    count is positive (''ifany``) and even for zero counts
    (''always``)''
  \end{itemize}
\end{itemize}

\begin{Shaded}
\begin{Highlighting}[]
\KeywordTok{table}\NormalTok{(df_event}\OperatorTok{$}\NormalTok{g12offered)}
\CommentTok{#> }
\CommentTok{#>     1 }
\CommentTok{#> 11025}
\KeywordTok{nrow}\NormalTok{(df_event)}
\CommentTok{#> [1] 17976}
\KeywordTok{table}\NormalTok{(df_event}\OperatorTok{$}\NormalTok{g12offered, }\DataTypeTok{useNA=}\StringTok{"always"}\NormalTok{)}
\CommentTok{#> }
\CommentTok{#>     1  <NA> }
\CommentTok{#> 11025  6951}
\end{Highlighting}
\end{Shaded}

Broader point:

\begin{itemize}
\tightlist
\item
  Most functions in R ignor missing values by default, but have the
  option to include missing values
\item
  When investigating data, generally a good idea to always include
  missing values
\end{itemize}

\end{frame}

\begin{frame}[fragile]{Filtering and missing values}

Wickham (2018) states:

\begin{itemize}
\tightlist
\item
  ``\hlgc{filter()} only includes rows where condition is TRUE; it
  excludes both \hlgc{FALSE} and \hlgc{NA} values. To preserve missing
  values, ask for them explicitly:''
\end{itemize}

\medskip Investigate var \hlgc{df\_event\textdollar fr\_lunch}, number
of free/reduced lunch students

\begin{itemize}
\tightlist
\item
  only available for visits to public high schools
\end{itemize}

\begin{Shaded}
\begin{Highlighting}[]
\CommentTok{#visits to public HS with less than 50 students on free/reduced lunch}
\KeywordTok{count}\NormalTok{(}\KeywordTok{filter}\NormalTok{(df_event,eventtype }\OperatorTok{==}\StringTok{ "public hs"}\NormalTok{, fr_lunch}\OperatorTok{<}\DecValTok{50}\NormalTok{))}
\CommentTok{#> # A tibble: 1 x 1}
\CommentTok{#>       n}
\CommentTok{#>   <int>}
\CommentTok{#> 1   890}
\CommentTok{#visits to public HS, where free/reduced lunch missing}
\KeywordTok{count}\NormalTok{(}\KeywordTok{filter}\NormalTok{(df_event,eventtype }\OperatorTok{==}\StringTok{ "public hs"}\NormalTok{, }\KeywordTok{is.na}\NormalTok{(fr_lunch)))}
\CommentTok{#> # A tibble: 1 x 1}
\CommentTok{#>       n}
\CommentTok{#>   <int>}
\CommentTok{#> 1    26}
\CommentTok{#visits to public HS, where free/reduced is less than 50 OR is missing}
\KeywordTok{count}\NormalTok{(}\KeywordTok{filter}\NormalTok{(df_event,eventtype }\OperatorTok{==}\StringTok{ "public hs"}\NormalTok{, fr_lunch}\OperatorTok{<}\DecValTok{50} \OperatorTok{|}\StringTok{ }\KeywordTok{is.na}\NormalTok{(fr_lunch)))}
\CommentTok{#> # A tibble: 1 x 1}
\CommentTok{#>       n}
\CommentTok{#>   <int>}
\CommentTok{#> 1   916}
\end{Highlighting}
\end{Shaded}

\end{frame}

\begin{frame}{Exercises, missing values}

\end{frame}

\section{Arrange rows}\label{arrange-rows}

\begin{frame}[fragile]{\hlgc{arrange()} function}

\hlgc{arrange()} function ``arranges'' rows in a data frame; said
different, it sorts observations

\medskip
Syntax: \hlgc{arrange(x,\ldots)}

\begin{itemize}
\tightlist
\item
  First argument, \hlgc{x}, is a data frame
\item
  Subsequent arguments are a ``comma separated list of unquoted variable
  names''
\end{itemize}

\begin{Shaded}
\begin{Highlighting}[]
\KeywordTok{arrange}\NormalTok{(df_event, event_date)}
\end{Highlighting}
\end{Shaded}

Data frame goes back to previous order unless you \textbf{assign} the
new order

\begin{Shaded}
\begin{Highlighting}[]
\NormalTok{df_event}
\NormalTok{df_event <-}\StringTok{ }\KeywordTok{arrange}\NormalTok{(df_event, event_date)}
\NormalTok{df_event}
\end{Highlighting}
\end{Shaded}

\end{frame}

\begin{frame}[fragile]{\hlgc{arrange()} function}

Ascending and descending order

\begin{itemize}
\tightlist
\item
  \hlgc{arrange()} sorts in \textbf{ascending} order by default
\item
  use \hlgc{desc()} to sort a column by descending order
\end{itemize}

\begin{Shaded}
\begin{Highlighting}[]
\KeywordTok{arrange}\NormalTok{(df_event, }\KeywordTok{desc}\NormalTok{(event_date))}
\end{Highlighting}
\end{Shaded}

Can sort by multiple variables

\begin{Shaded}
\begin{Highlighting}[]
\KeywordTok{arrange}\NormalTok{(df_event, univ_id, }\KeywordTok{desc}\NormalTok{(event_date), }\KeywordTok{desc}\NormalTok{(med_inc))}

\CommentTok{#sort by university and descending by size of 12th grade class; combine with select}
\KeywordTok{select}\NormalTok{(}\KeywordTok{arrange}\NormalTok{(df_event, univ_id, }\KeywordTok{desc}\NormalTok{(g12)),instnm,eventtype,event_date,g12)}
\end{Highlighting}
\end{Shaded}

\end{frame}

\begin{frame}[fragile]{\hlgc{arrange()}, missing values sorted at the
end}

Missing values automatically sorted at the end, regardless of whether
you sort ascending or descending

\begin{Shaded}
\begin{Highlighting}[]
\CommentTok{#by university, date, ascending school id}
\KeywordTok{select}\NormalTok{(}\KeywordTok{arrange}\NormalTok{(df_event, univ_id, }\KeywordTok{desc}\NormalTok{(event_date), school_id),instnm,event_date,eventtype,school_id)}
\CommentTok{#> # A tibble: 17,976 x 4}
\CommentTok{#>    instnm event_date eventtype  school_id   }
\CommentTok{#>    <chr>  <fct>      <chr>      <fct>       }
\CommentTok{#>  1 Bama   2017-12-18 private hs A9106483    }
\CommentTok{#>  2 Bama   2017-12-18 other      <NA>        }
\CommentTok{#>  3 Bama   2017-12-15 public hs  062927004516}
\CommentTok{#>  4 Bama   2017-12-15 public hs  484473005095}
\CommentTok{#>  5 Bama   2017-12-15 other      <NA>        }
\CommentTok{#>  6 Bama   2017-12-14 other      <NA>        }
\CommentTok{#>  7 Bama   2017-12-13 private hs 00071151    }
\CommentTok{#>  8 Bama   2017-12-13 public hs  063386005296}
\CommentTok{#>  9 Bama   2017-12-13 public hs  130387001439}
\CommentTok{#> 10 Bama   2017-12-13 other      <NA>        }
\CommentTok{#> # ... with 17,966 more rows}

\CommentTok{#by university, date, descending school id}
\KeywordTok{select}\NormalTok{(}\KeywordTok{arrange}\NormalTok{(df_event, univ_id, }\KeywordTok{desc}\NormalTok{(event_date), }\KeywordTok{desc}\NormalTok{(school_id)),instnm,event_date,eventtype,school_id)}
\CommentTok{#> # A tibble: 17,976 x 4}
\CommentTok{#>    instnm event_date eventtype  school_id   }
\CommentTok{#>    <chr>  <fct>      <chr>      <fct>       }
\CommentTok{#>  1 Bama   2017-12-18 private hs A9106483    }
\CommentTok{#>  2 Bama   2017-12-18 other      <NA>        }
\CommentTok{#>  3 Bama   2017-12-15 public hs  484473005095}
\CommentTok{#>  4 Bama   2017-12-15 public hs  062927004516}
\CommentTok{#>  5 Bama   2017-12-15 other      <NA>        }
\CommentTok{#>  6 Bama   2017-12-14 other      <NA>        }
\CommentTok{#>  7 Bama   2017-12-13 public hs  130387001439}
\CommentTok{#>  8 Bama   2017-12-13 public hs  063386005296}
\CommentTok{#>  9 Bama   2017-12-13 private hs 00071151    }
\CommentTok{#> 10 Bama   2017-12-13 other      <NA>        }
\CommentTok{#> # ... with 17,966 more rows}
\end{Highlighting}
\end{Shaded}

Can sort by \hlgc{is.na} to put missing values first

\begin{Shaded}
\begin{Highlighting}[]
\KeywordTok{select}\NormalTok{(}\KeywordTok{arrange}\NormalTok{(df_event, univ_id, }\KeywordTok{desc}\NormalTok{(event_date), }\KeywordTok{desc}\NormalTok{(}\KeywordTok{is.na}\NormalTok{(school_id))),instnm,event_date,eventtype,school_id)}
\CommentTok{#> # A tibble: 17,976 x 4}
\CommentTok{#>    instnm event_date eventtype  school_id   }
\CommentTok{#>    <chr>  <fct>      <chr>      <fct>       }
\CommentTok{#>  1 Bama   2017-12-18 other      <NA>        }
\CommentTok{#>  2 Bama   2017-12-18 private hs A9106483    }
\CommentTok{#>  3 Bama   2017-12-15 other      <NA>        }
\CommentTok{#>  4 Bama   2017-12-15 public hs  062927004516}
\CommentTok{#>  5 Bama   2017-12-15 public hs  484473005095}
\CommentTok{#>  6 Bama   2017-12-14 other      <NA>        }
\CommentTok{#>  7 Bama   2017-12-13 other      <NA>        }
\CommentTok{#>  8 Bama   2017-12-13 private hs 00071151    }
\CommentTok{#>  9 Bama   2017-12-13 public hs  063386005296}
\CommentTok{#> 10 Bama   2017-12-13 public hs  130387001439}
\CommentTok{#> # ... with 17,966 more rows}
\end{Highlighting}
\end{Shaded}

\end{frame}

\begin{frame}{Exercises, arranging}

\end{frame}

\end{document}
