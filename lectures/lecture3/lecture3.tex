\documentclass[8pt,ignorenonframetext,dvipsnames]{beamer}
\setbeamertemplate{caption}[numbered]
\setbeamertemplate{caption label separator}{: }
\setbeamercolor{caption name}{fg=normal text.fg}
\beamertemplatenavigationsymbolsempty
\usepackage{lmodern}
\usepackage{amssymb,amsmath}
\usepackage{ifxetex,ifluatex}
\usepackage{fixltx2e} % provides \textsubscript
\ifnum 0\ifxetex 1\fi\ifluatex 1\fi=0 % if pdftex
  \usepackage[T1]{fontenc}
  \usepackage[utf8]{inputenc}
\else % if luatex or xelatex
  \ifxetex
    \usepackage{mathspec}
  \else
    \usepackage{fontspec}
  \fi
  \defaultfontfeatures{Ligatures=TeX,Scale=MatchLowercase}
\fi
% use upquote if available, for straight quotes in verbatim environments
\IfFileExists{upquote.sty}{\usepackage{upquote}}{}
% use microtype if available
\IfFileExists{microtype.sty}{%
\usepackage{microtype}
\UseMicrotypeSet[protrusion]{basicmath} % disable protrusion for tt fonts
}{}
\newif\ifbibliography
\hypersetup{
            pdftitle={Lecture 3: Pipes and creating variables using mutate()},
            pdfauthor={Ozan Jaquette},
            colorlinks=true,
            linkcolor=Maroon,
            citecolor=Blue,
            urlcolor=blue,
            breaklinks=true}
\urlstyle{same}  % don't use monospace font for urls
\usepackage{color}
\usepackage{fancyvrb}
\newcommand{\VerbBar}{|}
\newcommand{\VERB}{\Verb[commandchars=\\\{\}]}
\DefineVerbatimEnvironment{Highlighting}{Verbatim}{commandchars=\\\{\}}
% Add ',fontsize=\small' for more characters per line
\usepackage{framed}
\definecolor{shadecolor}{RGB}{248,248,248}
\newenvironment{Shaded}{\begin{snugshade}}{\end{snugshade}}
\newcommand{\KeywordTok}[1]{\textcolor[rgb]{0.13,0.29,0.53}{\textbf{#1}}}
\newcommand{\DataTypeTok}[1]{\textcolor[rgb]{0.13,0.29,0.53}{#1}}
\newcommand{\DecValTok}[1]{\textcolor[rgb]{0.00,0.00,0.81}{#1}}
\newcommand{\BaseNTok}[1]{\textcolor[rgb]{0.00,0.00,0.81}{#1}}
\newcommand{\FloatTok}[1]{\textcolor[rgb]{0.00,0.00,0.81}{#1}}
\newcommand{\ConstantTok}[1]{\textcolor[rgb]{0.00,0.00,0.00}{#1}}
\newcommand{\CharTok}[1]{\textcolor[rgb]{0.31,0.60,0.02}{#1}}
\newcommand{\SpecialCharTok}[1]{\textcolor[rgb]{0.00,0.00,0.00}{#1}}
\newcommand{\StringTok}[1]{\textcolor[rgb]{0.31,0.60,0.02}{#1}}
\newcommand{\VerbatimStringTok}[1]{\textcolor[rgb]{0.31,0.60,0.02}{#1}}
\newcommand{\SpecialStringTok}[1]{\textcolor[rgb]{0.31,0.60,0.02}{#1}}
\newcommand{\ImportTok}[1]{#1}
\newcommand{\CommentTok}[1]{\textcolor[rgb]{0.56,0.35,0.01}{\textit{#1}}}
\newcommand{\DocumentationTok}[1]{\textcolor[rgb]{0.56,0.35,0.01}{\textbf{\textit{#1}}}}
\newcommand{\AnnotationTok}[1]{\textcolor[rgb]{0.56,0.35,0.01}{\textbf{\textit{#1}}}}
\newcommand{\CommentVarTok}[1]{\textcolor[rgb]{0.56,0.35,0.01}{\textbf{\textit{#1}}}}
\newcommand{\OtherTok}[1]{\textcolor[rgb]{0.56,0.35,0.01}{#1}}
\newcommand{\FunctionTok}[1]{\textcolor[rgb]{0.00,0.00,0.00}{#1}}
\newcommand{\VariableTok}[1]{\textcolor[rgb]{0.00,0.00,0.00}{#1}}
\newcommand{\ControlFlowTok}[1]{\textcolor[rgb]{0.13,0.29,0.53}{\textbf{#1}}}
\newcommand{\OperatorTok}[1]{\textcolor[rgb]{0.81,0.36,0.00}{\textbf{#1}}}
\newcommand{\BuiltInTok}[1]{#1}
\newcommand{\ExtensionTok}[1]{#1}
\newcommand{\PreprocessorTok}[1]{\textcolor[rgb]{0.56,0.35,0.01}{\textit{#1}}}
\newcommand{\AttributeTok}[1]{\textcolor[rgb]{0.77,0.63,0.00}{#1}}
\newcommand{\RegionMarkerTok}[1]{#1}
\newcommand{\InformationTok}[1]{\textcolor[rgb]{0.56,0.35,0.01}{\textbf{\textit{#1}}}}
\newcommand{\WarningTok}[1]{\textcolor[rgb]{0.56,0.35,0.01}{\textbf{\textit{#1}}}}
\newcommand{\AlertTok}[1]{\textcolor[rgb]{0.94,0.16,0.16}{#1}}
\newcommand{\ErrorTok}[1]{\textcolor[rgb]{0.64,0.00,0.00}{\textbf{#1}}}
\newcommand{\NormalTok}[1]{#1}
\usepackage{longtable,booktabs}
\usepackage{caption}
% These lines are needed to make table captions work with longtable:
\makeatletter
\def\fnum@table{\tablename~\thetable}
\makeatother

% Prevent slide breaks in the middle of a paragraph:
\widowpenalties 1 10000
\raggedbottom

\AtBeginPart{
  \let\insertpartnumber\relax
  \let\partname\relax
  \frame{\partpage}
}
\AtBeginSection{
  \ifbibliography
  \else
    \let\insertsectionnumber\relax
    \let\sectionname\relax
    \frame{\sectionpage}
  \fi
}
\AtBeginSubsection{
  \let\insertsubsectionnumber\relax
  \let\subsectionname\relax
  \frame{\subsectionpage}
}

\setlength{\parindent}{0pt}
\setlength{\parskip}{6pt plus 2pt minus 1pt}
\setlength{\emergencystretch}{3em}  % prevent overfull lines
\providecommand{\tightlist}{%
  \setlength{\itemsep}{0pt}\setlength{\parskip}{0pt}}
\setcounter{secnumdepth}{0}

%packages
\usepackage{graphicx}
\usepackage{rotating}
\usepackage{hyperref}

\usepackage{tikz} % used for text highlighting, amongst others
%title slide stuff
%\institute{Department of Education}
%\title{Managing and Manipulating Data Using R}

%
\setbeamertemplate{navigation symbols}{} % get rid of navigation icons:

%\setbeamertemplate{frametitle}{\thesection \hspace{0.2cm} \insertframetitle}
\setbeamertemplate{section in toc}[sections numbered]
%\setbeamertemplate{subsection in toc}[subsections numbered]
\setbeamertemplate{subsection in toc}{%
  \leavevmode\leftskip=3.2em\color{gray}\rlap{\hskip-2em\inserttocsectionnumber.\inserttocsubsectionnumber}\inserttocsubsection\par
}

%define colors
%\definecolor{uva_orange}{RGB}{216,141,42} % UVa orange (Rotunda orange)
\definecolor{mygray}{rgb}{0.95, 0.95, 0.95} % for highlighted text
	% grey is equal parts red, green, blue. higher values >> lighter grey
	%\definecolor{lightgraybo}{rgb}{0.83, 0.83, 0.83}

% new commands

%highlight text with very light grey
\newcommand*{\hlg}[1]{%
	\tikz[baseline=(X.base)] \node[rectangle, fill=mygray] (X) {#1};%
}
%, inner sep=0.3mm
%highlight text with very light grey and use font associated with code
\newcommand*{\hlgc}[1]{\texttt{\hlg{#1}}}

% Font
\usepackage[defaultfam,light,tabular,lining]{montserrat}
\usepackage[T1]{fontenc}
\renewcommand*\oldstylenums[1]{{\fontfamily{Montserrat-TOsF}\selectfont #1}}

% Change color of boldface text to darkgray
\renewcommand{\textbf}[1]{{\color{darkgray}\bfseries\fontfamily{Montserrat-TOsF}#1}}

% Bullet points
\setbeamertemplate{itemize item}{\color{BlueViolet}$\circ$}
\setbeamertemplate{itemize subitem}{\color{BrickRed}$\triangleright$}
\setbeamertemplate{itemize subsubitem}{$-$}

% Reduce space before lists
\addtobeamertemplate{itemize/enumerate body begin}{}{\vspace*{-8pt}}

\let\olditem\item
\renewcommand{\item}{%
  \olditem\vspace{4pt}
}

% decreasing space before and after level-2 bullet block
\addtobeamertemplate{itemize/enumerate subbody begin}{}{\vspace*{-3pt}}
\addtobeamertemplate{itemize/enumerate subbody end}{}{\vspace*{-3pt}}

% decreasing space before and after level-3 bullet block
\addtobeamertemplate{itemize/enumerate subsubbody begin}{}{\vspace*{-2pt}}
\addtobeamertemplate{itemize/enumerate subsubbody end}{}{\vspace*{-2pt}}

%Section numbering
\setbeamertemplate{section page}{%
    \begingroup
        \begin{beamercolorbox}[sep=10pt,center,rounded=true,shadow=true]{section title}
        \usebeamerfont{section title}\thesection~\insertsection\par
        \end{beamercolorbox}
    \endgroup
}

\setbeamertemplate{subsection page}{%
    \begingroup
        \begin{beamercolorbox}[sep=6pt,center,rounded=true,shadow=true]{subsection title}
        \usebeamerfont{subsection title}\thesection.\thesubsection~\insertsubsection\par
        \end{beamercolorbox}
    \endgroup
}

%modifying back ticks to add grey background
\let\OldTexttt\texttt
\renewcommand{\texttt}[1]{\OldTexttt{\hlg{#1}}}

\title{Lecture 3: Pipes and creating variables using \texttt{mutate()}}
\subtitle{EDUC 263: Managing and Manipulating Data Using R}
\author{Ozan Jaquette}
\date{}

\begin{document}
\frame{\titlepage}

\section{Introduction}\label{introduction}

\begin{frame}{What we will do today}

\tableofcontents

\end{frame}

\begin{frame}[fragile]{Libraries we will use today}

``Load'' the package we will use today (output omitted)

\begin{itemize}
\tightlist
\item
  \textbf{you must run this code chunk}
\end{itemize}

\begin{Shaded}
\begin{Highlighting}[]
\KeywordTok{library}\NormalTok{(tidyverse)}
\end{Highlighting}
\end{Shaded}

If package not yet installed, then must install before you load. Install
in ``console'' rather than .Rmd file

\begin{itemize}
\tightlist
\item
  Generic syntax: \texttt{install.packages("package\_name")}
\item
  Install ``tidyverse'': \texttt{install.packages("tidyverse")}
\end{itemize}

Note: when we load package, name of package is not in quotes; but when
we install package, name of package is in quotes:

\begin{itemize}
\tightlist
\item
  \texttt{install.packages("tidyverse")}
\item
  \texttt{library(tidyverse)}
\end{itemize}

\end{frame}

\subsection{Finish lecture 2, filter and arrange (i.e.,
sort)}\label{finish-lecture-2-filter-and-arrange-i.e.-sort}

\begin{frame}[fragile]{Load data for lecture 2}

Data on off-campus recruiting events by public universities

\begin{Shaded}
\begin{Highlighting}[]
\KeywordTok{rm}\NormalTok{(}\DataTypeTok{list =} \KeywordTok{ls}\NormalTok{()) }\CommentTok{# remove all objects}
\CommentTok{#load dataset with one obs per recruiting event}
\KeywordTok{load}\NormalTok{(}\KeywordTok{url}\NormalTok{(}\StringTok{"https://github.com/ozanj/rclass/raw/master/data/recruiting/recruit_event_somevars.RData"}\NormalTok{))}
\CommentTok{#load dataset with one obs per high school}
\KeywordTok{load}\NormalTok{(}\KeywordTok{url}\NormalTok{(}\StringTok{"https://github.com/ozanj/rclass/raw/master/data/recruiting/recruit_school_somevars.RData"}\NormalTok{))}
\end{Highlighting}
\end{Shaded}

Object \texttt{df\_event}

\begin{itemize}
\tightlist
\item
  Off-campus recruiting project; one obs per university, recruiting
  event
\end{itemize}

Object \texttt{df\_school}

\begin{itemize}
\tightlist
\item
  Off-campus recruiting project; one obs per high school (visited and
  non-visited)
\end{itemize}

\medskip Work through lecture 2 slides on \texttt{filter()} and
\texttt{arrange()}

\end{frame}

\subsection{Data for lecture 3}\label{data-for-lecture-3}

\begin{frame}[fragile]{Lecture 3 data: prospects purchased by Western
Washington U.}

The ``Student list'' business

\begin{itemize}
\tightlist
\item
  Universities identify/target ``prospects'' by buying ``student lists''
  from College Board/ACT (e.g., \$.40 per prospect)
\item
  Prospect lists contain contact info (e.g., address, email), academic
  achievement, socioeconomic, demographic characteristics
\item
  Universities choose which prospects to purchase by filtering on
  criteria like zip-code, GPA, test score range, etc.
\end{itemize}

\begin{Shaded}
\begin{Highlighting}[]
\CommentTok{#load prospect list data}
\KeywordTok{load}\NormalTok{(}\KeywordTok{url}\NormalTok{(}\StringTok{"https://github.com/ozanj/rclass/raw/master/data/prospect_list/wwlist_merged.RData"}\NormalTok{))}
\end{Highlighting}
\end{Shaded}

Object \texttt{wwlist}

\begin{itemize}
\tightlist
\item
  De-identified list of prospective students purchased by Western
  Washington University from College Board
\item
  We collected these data using FOIA request

  \begin{itemize}
  \tightlist
  \item
    ASIDE: Become an expert on collecting data via FOIA requests and you
    will become a superstar!
  \end{itemize}
\end{itemize}

\end{frame}

\begin{frame}[fragile]{Lecture 3 data: prospects purchased by Western
Washington U.}

Observations on \texttt{wwlist}

\begin{itemize}
\tightlist
\item
  each observation represents a prospective student
\end{itemize}

\begin{Shaded}
\begin{Highlighting}[]
\KeywordTok{typeof}\NormalTok{(wwlist)}
\CommentTok{#> [1] "list"}
\KeywordTok{dim}\NormalTok{(wwlist)}
\CommentTok{#> [1] 268396     31}
\end{Highlighting}
\end{Shaded}

Variables on \texttt{wwlist}

\begin{itemize}
\tightlist
\item
  some vars provide de-identified data on individual prospects

  \begin{itemize}
  \tightlist
  \item
    e.g., \texttt{psat\_range}, \texttt{state}, \texttt{sex},
    \texttt{ethn\_code}
  \end{itemize}
\item
  some vars provide data about zip-code student lives in

  \begin{itemize}
  \tightlist
  \item
    e.g., \texttt{med\_inc}, \texttt{pop\_total}, \texttt{pop\_black}
  \end{itemize}
\item
  some vars provide data about school student enrolled in

  \begin{itemize}
  \tightlist
  \item
    e.g., \texttt{fr\_lunch} is number of students on free/reduced lunch
  \item
    note: bad merge between prospect-level data and school-level data
  \end{itemize}
\end{itemize}

\begin{Shaded}
\begin{Highlighting}[]
\KeywordTok{names}\NormalTok{(wwlist)}
\KeywordTok{str}\NormalTok{(wwlist)}
\end{Highlighting}
\end{Shaded}

\end{frame}

\section{Pipes}\label{pipes}

\begin{frame}[fragile]{What are ``pipes'', \%\textgreater{}\%}

\textbf{Pipes} are a means of perfoming multiple steps in a single line
of code

\begin{itemize}
\tightlist
\item
  Pipes are part of \textbf{tidyverse} suite of packages, not
  \textbf{base R}
\item
  When writing code, the pipe symbol is \texttt{\%\textgreater{}\%}
\item
  Basic flow of using pipes in code:

  \begin{itemize}
  \tightlist
  \item
    \texttt{object\ \%\textgreater{}\%\ some\_function\ \%\textgreater{}\%\ some\_function,\ \textbackslash{}ldots}\\
  \end{itemize}
\item
  Pipes work from left to right:

  \begin{itemize}
  \tightlist
  \item
    The object/result from left of \texttt{\%\textgreater{}\%} pipe
    symbol is the input of function to the right of the
    \texttt{\%\textgreater{}\%} pipe symbol
  \item
    In turn, the resulting output becomes the input of the function to
    the right of the next \texttt{\%\textgreater{}\%} pipe symbol
  \end{itemize}
\end{itemize}

Intuitive mnemonic device for understanding pipes

\begin{itemize}
\tightlist
\item
  whenever you see a pipe \texttt{\%\textgreater{}\%} think of the words
  ``\textbf{and then\ldots{}}''
\item
  Example:
  \texttt{wwlist\ \%\textgreater{}\%\ filter(firstgen\ ==\ "Y")}

  \begin{itemize}
  \tightlist
  \item
    in words: start with object \texttt{wwlist} \textbf{and then} filter
    first generation students
  \end{itemize}
\end{itemize}

\end{frame}

\begin{frame}[fragile]{Do task with and without pipes}

Task:

\begin{itemize}
\tightlist
\item
  Using object \texttt{wwlist} print data for ``first-generation''
  prospects (\texttt{firstgen\ ==\ "Y"})
\end{itemize}

\begin{Shaded}
\begin{Highlighting}[]
\KeywordTok{filter}\NormalTok{(wwlist, firstgen }\OperatorTok{==}\StringTok{ "Y"}\NormalTok{) }\CommentTok{# without pipes}
\NormalTok{wwlist }\OperatorTok\StringTok{ }\KeywordTok{filter}\NormalTok{(firstgen }\OperatorTok{==}\StringTok{ "Y"}\NormalTok{) }\CommentTok{# with pipes}
\end{Highlighting}
\end{Shaded}

Comparing the two approaches:

\begin{itemize}
\tightlist
\item
  In the ``without pipes'' approach, the object is the first argument
  \texttt{filter()} function
\item
  In the ``pipes'' approach, you don't specify the object as the first
  argument of \texttt{filter()}

  \begin{itemize}
  \tightlist
  \item
    Why? Because \texttt{\%\textgreater{}\%} ``pipes'' the object to the
    left of the \texttt{\%\textgreater{}\%} operator into the function
    to the right of the \texttt{\%\textgreater{}\%} operator
  \end{itemize}
\end{itemize}

Main takeaway:

\begin{itemize}
\tightlist
\item
  When writing code using pipes, functions to right of
  \texttt{\%\textgreater{}\%} pipe operator should not explicitly name
  object that is the input to the function.
\item
  Rather, object to the left of \texttt{\%\textgreater{}\%} pipe
  operator is automatically the input.
\end{itemize}

\end{frame}

\begin{frame}[fragile]{More intuition on the pipe operator,
\texttt{\%\textgreater{}\%}}

The pipe operator ``pipes'' (verb) an object from left of
\texttt{\%\textgreater{}\%} operator into the function to the right of
the \%\textgreater{}\% operator

Example:

\begin{Shaded}
\begin{Highlighting}[]
\KeywordTok{str}\NormalTok{(wwlist) }\CommentTok{# without pipe}

\NormalTok{wwlist }\OperatorTok\StringTok{ }\KeywordTok{str}\NormalTok{() }\CommentTok{# with pipe}
\end{Highlighting}
\end{Shaded}

\end{frame}

\begin{frame}[fragile]{Do task with and without pipes}

\textbf{Task}: Using object \texttt{wwlist}, print data for
``first-gen'' prospects for selected variables {[}output omitted{]}

\begin{Shaded}
\begin{Highlighting}[]
\CommentTok{#Without pipes}
\KeywordTok{select}\NormalTok{(}\KeywordTok{filter}\NormalTok{(wwlist, firstgen }\OperatorTok{==}\StringTok{ "Y"}\NormalTok{), state, hs_city, sex)}
\CommentTok{#With pipes}
\NormalTok{wwlist }\OperatorTok\StringTok{ }\KeywordTok{filter}\NormalTok{(firstgen }\OperatorTok{==}\StringTok{ "Y"}\NormalTok{) }\OperatorTok\StringTok{ }\KeywordTok{select}\NormalTok{(state, hs_city, sex)}
\end{Highlighting}
\end{Shaded}

Comparing the two approaches:

\begin{itemize}
\tightlist
\item
  In the ``without pipes'' approach, code is written ``inside out''

  \begin{itemize}
  \tightlist
  \item
    The first step in the task -- identifying the object -- is the
    innermost part of code
  \item
    The last step in task -- selecting variables to print -- is the
    outermost part of code
  \end{itemize}
\item
  In ``pipes'' approach the left-to-right order of code matches how we
  think about the task

  \begin{itemize}
  \tightlist
  \item
    First, we start with an object \textbf{\emph{and then}}
    (\texttt{\%\textgreater{}\%}) we use \texttt{filter()} to isolate
    first-gen students \textbf{\emph{and then}}
    (\texttt{\%\textgreater{}\%}) we select which variables to print
  \end{itemize}
\end{itemize}

Note the object ``piped'' into \texttt{select()} from \texttt{filter()}

\begin{Shaded}
\begin{Highlighting}[]
\NormalTok{wwlist }\OperatorTok\StringTok{ }\KeywordTok{filter}\NormalTok{(firstgen }\OperatorTok{==}\StringTok{ "Y"}\NormalTok{) }\OperatorTok\StringTok{ }\KeywordTok{str}\NormalTok{()}
\end{Highlighting}
\end{Shaded}

\end{frame}

\begin{frame}[fragile]{Do task with and without pipes}

Task:

\begin{itemize}
\tightlist
\item
  Count the number ``first-generation'' prospects from the state of
  Washington
\end{itemize}

Without pipes

\begin{Shaded}
\begin{Highlighting}[]
\KeywordTok{count}\NormalTok{(}\KeywordTok{filter}\NormalTok{(wwlist, firstgen }\OperatorTok{==}\StringTok{ "Y"}\NormalTok{, state }\OperatorTok{==}\StringTok{ "WA"}\NormalTok{))}
\CommentTok{#> # A tibble: 1 x 1}
\CommentTok{#>       n}
\CommentTok{#>   <int>}
\CommentTok{#> 1 32428}
\end{Highlighting}
\end{Shaded}

With pipes

\begin{Shaded}
\begin{Highlighting}[]
\NormalTok{wwlist }\OperatorTok\StringTok{ }\KeywordTok{filter}\NormalTok{(firstgen }\OperatorTok{==}\StringTok{ "Y"}\NormalTok{, state }\OperatorTok{==}\StringTok{ "WA"}\NormalTok{) }\OperatorTok\StringTok{ }\KeywordTok{count}\NormalTok{()}
\CommentTok{#> # A tibble: 1 x 1}
\CommentTok{#>       n}
\CommentTok{#>   <int>}
\CommentTok{#> 1 32428}
\end{Highlighting}
\end{Shaded}

\end{frame}

\begin{frame}[fragile]{Do task with and without pipes}

\textbf{Task}: Create frequency table of \texttt{school\_type} for non
first-gen prospects from WA

without pipes

\begin{Shaded}
\begin{Highlighting}[]
\NormalTok{wwlist_temp <-}\StringTok{ }\KeywordTok{filter}\NormalTok{(wwlist, firstgen }\OperatorTok{==}\StringTok{ "N"}\NormalTok{, state }\OperatorTok{==}\StringTok{ "WA"}\NormalTok{)}
\KeywordTok{table}\NormalTok{(wwlist_temp}\OperatorTok{$}\NormalTok{school_type, }\DataTypeTok{useNA =} \StringTok{"always"}\NormalTok{)}
\CommentTok{#> }
\CommentTok{#> private  public    <NA> }
\CommentTok{#>      11   46146   12489}
\KeywordTok{rm}\NormalTok{(wwlist_temp) }\CommentTok{# cuz we don't need after creating table}
\end{Highlighting}
\end{Shaded}

With pipes

\begin{Shaded}
\begin{Highlighting}[]
\NormalTok{wwlist }\OperatorTok\StringTok{ }\KeywordTok{filter}\NormalTok{(firstgen }\OperatorTok{==}\StringTok{ "N"}\NormalTok{, state }\OperatorTok{==}\StringTok{ "WA"}\NormalTok{) }\OperatorTok\StringTok{ }\KeywordTok{count}\NormalTok{(school_type)}
\CommentTok{#> # A tibble: 3 x 2}
\CommentTok{#>   school_type     n}
\CommentTok{#>   <chr>       <int>}
\CommentTok{#> 1 private        11}
\CommentTok{#> 2 public      46146}
\CommentTok{#> 3 <NA>        12489}
\end{Highlighting}
\end{Shaded}

\end{frame}

\begin{frame}{Comparing the two approachces}

Comparison of two approaches

\begin{itemize}
\tightlist
\item
  without pipes, task requires multiple lines of code; this is quite
  common

  \begin{itemize}
  \tightlist
  \item
    first line creates object; second line analyzes object
  \end{itemize}
\item
  with pipes, task can be completed in one line of code and you aren't
  left with objects you don't care about
\end{itemize}

\end{frame}

\begin{frame}[fragile]{Student exercises with pipes}

\begin{enumerate}
\def\labelenumi{\arabic{enumi}.}
\item
  Using object \texttt{wwlist} select the following variables (state,
  firstgen, ethn\_code) and assign them to object \texttt{wwlist\_temp}.
\item
  Using the object you just created \texttt{wwlist\_temp}, create a
  frequency table of \texttt{ethn\_code} for first-gen prospects from
  California.
\item
  \textbf{Bonus}: Try doing question 1 and 2 together. Use original
  object \texttt{wwlist}, but do not assign to a new object.
\end{enumerate}

Once finished you can \texttt{rm(wwlist\_temp)}

\end{frame}

\begin{frame}[fragile]{Solution to exercises with pipes}

\begin{enumerate}
\def\labelenumi{\arabic{enumi}.}
\tightlist
\item
  Using object \texttt{wwlist} select the following variables (state,
  firstgen, ethn\_code) and assign them to object \texttt{wwlist\_small}
\end{enumerate}

\begin{Shaded}
\begin{Highlighting}[]
\NormalTok{wwlist_temp <-}\StringTok{ }\NormalTok{wwlist }\OperatorTok
\StringTok{  }\KeywordTok{select}\NormalTok{(state, firstgen, ethn_code) }
\end{Highlighting}
\end{Shaded}

\end{frame}

\begin{frame}[fragile]{Solution to exercises with pipes}

\begin{enumerate}
\def\labelenumi{\arabic{enumi}.}
\setcounter{enumi}{1}
\tightlist
\item
  Using the object you just created \texttt{wwlist\_temp}, create a
  frequency table of \texttt{ethn\_code} for first-gen prospects from
  California.
\end{enumerate}

\begin{Shaded}
\begin{Highlighting}[]
\CommentTok{#names(wwlist)}
\NormalTok{wwlist_temp }\OperatorTok
\StringTok{  }\KeywordTok{filter}\NormalTok{(firstgen }\OperatorTok{==}\StringTok{ "Y"}\NormalTok{, state }\OperatorTok{==}\StringTok{ "CA"}\NormalTok{) }\OperatorTok\StringTok{ }\KeywordTok{count}\NormalTok{(ethn_code)}
\CommentTok{#> # A tibble: 11 x 2}
\CommentTok{#>    ethn_code                                               n}
\CommentTok{#>    <fct>                                               <int>}
\CommentTok{#>  1 American Indian or Alaska Native                        4}
\CommentTok{#>  2 Asian or Native Hawaiian or Other Pacific Islander     81}
\CommentTok{#>  3 Asian or Native Hawaiian or Other Pacific IslanderH     5}
\CommentTok{#>  4 Black or African American                              10}
\CommentTok{#>  5 Cuban                                                   1}
\CommentTok{#>  6 Mexican/Mexican American                              643}
\CommentTok{#>  7 Not reported                                          113}
\CommentTok{#>  8 Other-2 or more                                      4197}
\CommentTok{#>  9 Other Spanish/Hispanic                                179}
\CommentTok{#> 10 Puerto Rican                                            8}
\CommentTok{#> 11 White                                                2933}
\end{Highlighting}
\end{Shaded}

\end{frame}

\begin{frame}[fragile]{Solution to exercises with pipes}

\begin{enumerate}
\def\labelenumi{\arabic{enumi}.}
\setcounter{enumi}{2}
\tightlist
\item
  \textbf{Bonus}: Try doing question 1 and 2 together.
\end{enumerate}

\begin{Shaded}
\begin{Highlighting}[]
\NormalTok{wwlist }\OperatorTok
\StringTok{  }\KeywordTok{select}\NormalTok{(state, firstgen, ethn_code) }\OperatorTok
\StringTok{  }\KeywordTok{filter}\NormalTok{(firstgen }\OperatorTok{==}\StringTok{ "Y"}\NormalTok{, state }\OperatorTok{==}\StringTok{ "CA"}\NormalTok{) }\OperatorTok\StringTok{ }
\StringTok{  }\KeywordTok{count}\NormalTok{(ethn_code)}
\CommentTok{#> # A tibble: 11 x 2}
\CommentTok{#>    ethn_code                                               n}
\CommentTok{#>    <fct>                                               <int>}
\CommentTok{#>  1 American Indian or Alaska Native                        4}
\CommentTok{#>  2 Asian or Native Hawaiian or Other Pacific Islander     81}
\CommentTok{#>  3 Asian or Native Hawaiian or Other Pacific IslanderH     5}
\CommentTok{#>  4 Black or African American                              10}
\CommentTok{#>  5 Cuban                                                   1}
\CommentTok{#>  6 Mexican/Mexican American                              643}
\CommentTok{#>  7 Not reported                                          113}
\CommentTok{#>  8 Other-2 or more                                      4197}
\CommentTok{#>  9 Other Spanish/Hispanic                                179}
\CommentTok{#> 10 Puerto Rican                                            8}
\CommentTok{#> 11 White                                                2933}
\CommentTok{#rm(wwlist_temp)}
\end{Highlighting}
\end{Shaded}

\begin{Shaded}
\begin{Highlighting}[]
\KeywordTok{rm}\NormalTok{(wwlist_temp)}
\end{Highlighting}
\end{Shaded}

\end{frame}

\section{Creating variables using
mutate}\label{creating-variables-using-mutate}

\begin{frame}[fragile]{Our plan for learning how to create new
variables}

Recall that \texttt{dplyr} package within \texttt{tidyverse} provide a
set of functions that can be described as ``verbs'':
\textbf{subsetting}, \textbf{sorting}, and \textbf{transforming}

\begin{longtable}[]{@{}ll@{}}
\toprule
What we've done & Where we're going\tabularnewline
\midrule
\endhead
\textbf{Subsetting data} & \textbf{Transforming data}\tabularnewline
- \texttt{select()} variables & - \texttt{mutate()} creates new
variables\tabularnewline
- \texttt{filter()} observations & - \texttt{summarize()} calculates
across rows\tabularnewline
\textbf{Sorting data} & - \texttt{group\_by()} to calculate across rows
within groups\tabularnewline
- \texttt{arrange()} &\tabularnewline
\bottomrule
\end{longtable}

\textbf{Today}

\begin{itemize}
\tightlist
\item
  we'll use \texttt{mutate()} to create new variables based on
  calculations across columns within a row
\end{itemize}

\textbf{Next week}

\begin{itemize}
\tightlist
\item
  we'll combine \texttt{mutate()} with \texttt{summarize()} and
  \texttt{group\_by()} to create variables based on calculations across
  rows
\end{itemize}

\end{frame}

\begin{frame}[fragile]{Create new data frame based on
\texttt{df\_school\_all}}

Data frame \texttt{df\_school\_all} has one obs per US high school and
then variables identifying number of visits by particular universities

\begin{Shaded}
\begin{Highlighting}[]
\KeywordTok{load}\NormalTok{(}\KeywordTok{url}\NormalTok{(}\StringTok{"https://github.com/ozanj/rclass/raw/master/data/recruiting/recruit_school_allvars.RData"}\NormalTok{))}
\KeywordTok{names}\NormalTok{(df_school_all)}
\CommentTok{#>  [1] "state_code"         "school_type"        "ncessch"           }
\CommentTok{#>  [4] "name"               "address"            "city"              }
\CommentTok{#>  [7] "zip_code"           "pct_white"          "pct_black"         }
\CommentTok{#> [10] "pct_hispanic"       "pct_asian"          "pct_amerindian"    }
\CommentTok{#> [13] "pct_other"          "num_fr_lunch"       "total_students"    }
\CommentTok{#> [16] "num_took_math"      "num_prof_math"      "num_took_rla"      }
\CommentTok{#> [19] "num_prof_rla"       "avgmedian_inc_2564" "latitude"          }
\CommentTok{#> [22] "longitude"          "visits_by_196097"   "visits_by_186380"  }
\CommentTok{#> [25] "visits_by_215293"   "visits_by_201885"   "visits_by_181464"  }
\CommentTok{#> [28] "visits_by_139959"   "visits_by_218663"   "visits_by_100751"  }
\CommentTok{#> [31] "visits_by_199193"   "visits_by_110635"   "visits_by_110653"  }
\CommentTok{#> [34] "visits_by_126614"   "visits_by_155317"   "visits_by_106397"  }
\CommentTok{#> [37] "visits_by_149222"   "visits_by_166629"   "total_visits"      }
\CommentTok{#> [40] "inst_196097"        "inst_186380"        "inst_215293"       }
\CommentTok{#> [43] "inst_201885"        "inst_181464"        "inst_139959"       }
\CommentTok{#> [46] "inst_218663"        "inst_100751"        "inst_199193"       }
\CommentTok{#> [49] "inst_110635"        "inst_110653"        "inst_126614"       }
\CommentTok{#> [52] "inst_155317"        "inst_106397"        "inst_149222"       }
\CommentTok{#> [55] "inst_166629"}
\end{Highlighting}
\end{Shaded}

\end{frame}

\begin{frame}[fragile]{Create new data frame based on
\texttt{df\_school\_all}}

Let's create new version of this data frame, called \texttt{school\_v2},
which we'll use to introduce how to create new variables

\begin{Shaded}
\begin{Highlighting}[]
\NormalTok{school_v2 <-}\StringTok{ }\NormalTok{df_school_all }\OperatorTok\StringTok{ }
\StringTok{  }\KeywordTok{select}\NormalTok{(}\OperatorTok{-}\KeywordTok{contains}\NormalTok{(}\StringTok{"inst_"}\NormalTok{)) }\OperatorTok\StringTok{ }\CommentTok{# remove vars that start with "inst_"}
\StringTok{  }\KeywordTok{rename}\NormalTok{(}
    \DataTypeTok{visits_by_berkeley =}\NormalTok{ visits_by_}\DecValTok{110635}\NormalTok{,}
    \DataTypeTok{visits_by_boulder =}\NormalTok{ visits_by_}\DecValTok{126614}\NormalTok{,}
    \DataTypeTok{visits_by_bama =}\NormalTok{ visits_by_}\DecValTok{100751}\NormalTok{,}
    \DataTypeTok{visits_by_stonybrook =}\NormalTok{ visits_by_}\DecValTok{196097}\NormalTok{,}
    \DataTypeTok{visits_by_rutgers =}\NormalTok{ visits_by_}\DecValTok{186380}\NormalTok{,}
    \DataTypeTok{visits_by_pitt =}\NormalTok{ visits_by_}\DecValTok{215293}\NormalTok{,}
    \DataTypeTok{visits_by_cinci =}\NormalTok{ visits_by_}\DecValTok{201885}\NormalTok{,}
    \DataTypeTok{visits_by_nebraska =}\NormalTok{ visits_by_}\DecValTok{181464}\NormalTok{,}
    \DataTypeTok{visits_by_georgia =}\NormalTok{ visits_by_}\DecValTok{139959}\NormalTok{,}
    \DataTypeTok{visits_by_scarolina =}\NormalTok{ visits_by_}\DecValTok{218663}\NormalTok{,}
    \DataTypeTok{visits_by_ncstate =}\NormalTok{ visits_by_}\DecValTok{199193}\NormalTok{,}
    \DataTypeTok{visits_by_irvine =}\NormalTok{ visits_by_}\DecValTok{110653}\NormalTok{,}
    \DataTypeTok{visits_by_kansas =}\NormalTok{ visits_by_}\DecValTok{155317}\NormalTok{,}
    \DataTypeTok{visits_by_arkansas =}\NormalTok{ visits_by_}\DecValTok{106397}\NormalTok{,}
    \DataTypeTok{visits_by_sillinois =}\NormalTok{ visits_by_}\DecValTok{149222}\NormalTok{,}
    \DataTypeTok{visits_by_umass =}\NormalTok{ visits_by_}\DecValTok{166629}\NormalTok{,}
    \DataTypeTok{num_took_read =}\NormalTok{ num_took_rla,}
    \DataTypeTok{num_prof_read =}\NormalTok{ num_prof_rla,}
    \DataTypeTok{med_inc =}\NormalTok{ avgmedian_inc_}\DecValTok{2564}\NormalTok{)}

\KeywordTok{names}\NormalTok{(school_v2)}
\end{Highlighting}
\end{Shaded}

\end{frame}

\subsection{Introduce mutate()
function}\label{introduce-mutate-function}

\begin{frame}[fragile]{Introduce \texttt{mutate()} function}

\texttt{mutate()} is \textbf{tidyverse} approach to creating variables
(not \textbf{Base R} approach)

Description of \texttt{mutate()}

\begin{itemize}
\tightlist
\item
  \texttt{mutate()} creates new columns (variables) that are functions
  of existing columns
\item
  After creating a new variable using \texttt{mutate()}, every row of
  data is retained
\item
  \texttt{mutate()} works best with pipes \texttt{\%\textgreater{}\%}
\end{itemize}

\textbf{Task}:

\begin{itemize}
\tightlist
\item
  Using data frame \texttt{school\_v2} create new variable that measures
  the pct of students on free/reduced lunch (output omitted)
\end{itemize}

\begin{Shaded}
\begin{Highlighting}[]
\NormalTok{school_sml <-}\StringTok{ }\NormalTok{school_v2 }\OperatorTok\StringTok{ }\CommentTok{# create new dataset with fewer vars; not necessary to do this}
\StringTok{  }\KeywordTok{select}\NormalTok{(ncessch, school_type, num_fr_lunch, total_students)}

\NormalTok{school_sml }\OperatorTok\StringTok{ }
\StringTok{  }\KeywordTok{mutate}\NormalTok{(}\DataTypeTok{pct_fr_lunch =}\NormalTok{ num_fr_lunch}\OperatorTok{/}\NormalTok{total_students) }\CommentTok{# create new var}

\KeywordTok{rm}\NormalTok{(school_sml)}
\end{Highlighting}
\end{Shaded}

\end{frame}

\begin{frame}[fragile]{Syntax for \texttt{mutate()}}

Let's spend a couple minutes looking at help file for \texttt{mutate()}

\begin{Shaded}
\begin{Highlighting}[]
\NormalTok{?mutate}
\end{Highlighting}
\end{Shaded}

\textbf{Usage (i.e., syntax)}

\begin{itemize}
\tightlist
\item
  \texttt{mutate(.data,...)}
\end{itemize}

\textbf{Arguments}

\begin{itemize}
\tightlist
\item
  \texttt{.data}: a data frame

  \begin{itemize}
  \tightlist
  \item
    if using \texttt{mutate()} after pipe operator
    \texttt{\%\textgreater{}\%}, then this argument can be omitted

    \begin{itemize}
    \tightlist
    \item
      Why? Because data frame object to left of
      \texttt{\%\textgreater{}\%} ``piped in'' to first argument of
      \texttt{mutate()}
    \end{itemize}
  \end{itemize}
\item
  \texttt{...}: expressions used to create new variables

  \begin{itemize}
  \tightlist
  \item
    Can create multiple variables at once
  \end{itemize}
\end{itemize}

\textbf{Value}

\begin{itemize}
\tightlist
\item
  returns an object that contains the original input data frame and new
  variables that were created by \texttt{mutate()}
\end{itemize}

\textbf{Useful functions (i.e., ``helper functions'')}

\begin{itemize}
\tightlist
\item
  These are standalone functions can be called \emph{within}
  \texttt{mutate()}

  \begin{itemize}
  \tightlist
  \item
    e.g., \texttt{if\_else()}, \texttt{recode()}, \texttt{case\_when()}
  \end{itemize}
\item
  will show examples of this in subsequent slides
\end{itemize}

\end{frame}

\begin{frame}[fragile]{Introduce \texttt{mutate()} function}

New variable not retained unless we \textbf{assign}
\texttt{\textless{}-} it to an object (existing or new)

\textbf{\texttt{mutate()} without assignment}

\begin{Shaded}
\begin{Highlighting}[]
\NormalTok{school_v2 }\OperatorTok\StringTok{ }\KeywordTok{mutate}\NormalTok{(}\DataTypeTok{pct_fr_lunch =}\NormalTok{ num_fr_lunch}\OperatorTok{/}\NormalTok{total_students)}

\KeywordTok{names}\NormalTok{(school_v2)}
\end{Highlighting}
\end{Shaded}

\textbf{\texttt{mutate()} with assignment}

\begin{Shaded}
\begin{Highlighting}[]
\NormalTok{school_v2_temp <-}\StringTok{ }\NormalTok{school_v2 }\OperatorTok\StringTok{ }
\StringTok{  }\KeywordTok{mutate}\NormalTok{(}\DataTypeTok{pct_fr_lunch =}\NormalTok{ num_fr_lunch}\OperatorTok{/}\NormalTok{total_students) }

\KeywordTok{names}\NormalTok{(school_v2_temp)}
\KeywordTok{rm}\NormalTok{(school_v2_temp)}
\end{Highlighting}
\end{Shaded}

\end{frame}

\begin{frame}[fragile]{Aside: Base R approach to creating new variables}

Task:

\begin{itemize}
\tightlist
\item
  Create measure of percent of students on free-reduced lunch
\end{itemize}

\textbf{dplyr/tidyverse approach}

\begin{Shaded}
\begin{Highlighting}[]
\NormalTok{school_v2_temp <-}\StringTok{ }\NormalTok{school_v2 }\OperatorTok\StringTok{ }
\StringTok{  }\KeywordTok{mutate}\NormalTok{(}\DataTypeTok{pct_fr_lunch =}\NormalTok{ num_fr_lunch}\OperatorTok{/}\NormalTok{total_students) }
\end{Highlighting}
\end{Shaded}

\textbf{Base R approach}

\begin{Shaded}
\begin{Highlighting}[]
\NormalTok{school_v2_temp <-}\StringTok{ }\NormalTok{school_v2 }\CommentTok{# create copy of dataset; not necessary}

\NormalTok{school_v2_temp}\OperatorTok{$}\NormalTok{pct_fr_lunch <-}\StringTok{ }\NormalTok{school_v2_temp}\OperatorTok{$}\NormalTok{num_fr_lunch}\OperatorTok{/}\NormalTok{school_v2_temp}\OperatorTok{$}\NormalTok{total_students}

\KeywordTok{names}\NormalTok{(school_v2_temp)}
\KeywordTok{rm}\NormalTok{(school_v2_temp)}
\end{Highlighting}
\end{Shaded}

Good to know both Base R and tidyverse approaches; sometimes you need to
use one or the other

\begin{itemize}
\tightlist
\item
  But overwhelming to learn both approaches at once
\item
  We'll focus mostly on learning tidyverse approaches
\item
  But I'll try to work-in opportunities to learn Base R approach
\end{itemize}

\end{frame}

\begin{frame}[fragile]{\texttt{mutate()} can create multiple variables
at once}

\texttt{mutate()} can create multiple variables at once

\begin{Shaded}
\begin{Highlighting}[]
\NormalTok{school_v2 }\OperatorTok\StringTok{ }
\StringTok{  }\KeywordTok{mutate}\NormalTok{(}\DataTypeTok{pct_fr_lunch =}\NormalTok{ num_fr_lunch}\OperatorTok{/}\NormalTok{total_students,}
         \DataTypeTok{pct_prof_math=}\NormalTok{ num_prof_math}\OperatorTok{/}\NormalTok{num_took_math) }\OperatorTok
\StringTok{  }\KeywordTok{select}\NormalTok{(num_fr_lunch, total_students, pct_fr_lunch, }
\NormalTok{         num_prof_math, num_took_math, pct_prof_math)}
\end{Highlighting}
\end{Shaded}

Or we could write code this way:

\begin{Shaded}
\begin{Highlighting}[]
\NormalTok{school_v2 }\OperatorTok\StringTok{ }
\StringTok{  }\KeywordTok{select}\NormalTok{(num_fr_lunch, total_students, num_prof_math, num_took_math) }\OperatorTok
\StringTok{  }\KeywordTok{mutate}\NormalTok{(}\DataTypeTok{pct_fr_lunch =}\NormalTok{ num_fr_lunch}\OperatorTok{/}\NormalTok{total_students,}
         \DataTypeTok{pct_prof_math=}\NormalTok{ num_prof_math}\OperatorTok{/}\NormalTok{num_took_math) }
\end{Highlighting}
\end{Shaded}

\end{frame}

\begin{frame}[fragile]{Student exercise using mutate()}

\begin{enumerate}
\def\labelenumi{\arabic{enumi}.}
\item
  Using the object \texttt{school\_v2}, select the following variables
  (num\_prof\_math, num\_took\_math, num\_prof\_read, num\_took\_read)
  and create a measure of percent proficient in math
  \texttt{pct\_prof\_math} and percent proficient in reading
  \texttt{pct\_prof\_read}.
\item
  Now using the code for question 1, filter schools where at least 50\%
  of students are proficient in math \textbf{\&} reading.
\item
  If you have time, count the number of schools from question 2.
\end{enumerate}

\end{frame}

\begin{frame}[fragile]{Solutions for exercise using mutate()}

\begin{enumerate}
\def\labelenumi{\arabic{enumi}.}
\tightlist
\item
  Using the object \texttt{school\_v2}, select the following variables
  (num\_prof\_math, num\_took\_math, num\_prof\_read, num\_took\_read)
  and create a measure of percent proficient in math and percent
  proficient in reading.
\end{enumerate}

\begin{Shaded}
\begin{Highlighting}[]
\NormalTok{school_v2 }\OperatorTok
\StringTok{  }\KeywordTok{select}\NormalTok{(num_prof_math, num_took_math, num_prof_read, num_took_read) }\OperatorTok
\StringTok{  }\KeywordTok{mutate}\NormalTok{(}\DataTypeTok{pct_prof_math =}\NormalTok{ num_prof_math}\OperatorTok{/}\NormalTok{num_took_math,}
         \DataTypeTok{pct_prof_read =}\NormalTok{ num_prof_read}\OperatorTok{/}\NormalTok{num_took_read) }
\CommentTok{#> # A tibble: 21,301 x 6}
\CommentTok{#>    num_prof_math num_took_math num_prof_read num_took_read pct_prof_math}
\CommentTok{#>            <dbl>         <dbl>         <dbl>         <dbl>         <dbl>}
\CommentTok{#>  1         24.8            146         25.0            147         0.17 }
\CommentTok{#>  2          1.7             17          1.7             17         0.10 }
\CommentTok{#>  3          3.5             14          3.5             14         0.25 }
\CommentTok{#>  4          3               30          3               30         0.1  }
\CommentTok{#>  5          2.8             28          2.8             28         0.10 }
\CommentTok{#>  6          2.5             25          2.4             24         0.1  }
\CommentTok{#>  7          1.55            62          1.55            62         0.025}
\CommentTok{#>  8          2.1             21          2.2             22         0.1  }
\CommentTok{#>  9          2.3             23          2.3             23         0.10 }
\CommentTok{#> 10          1.9             19          1.9             19         0.10 }
\CommentTok{#> # ... with 21,291 more rows, and 1 more variable: pct_prof_read <dbl>}
\end{Highlighting}
\end{Shaded}

\end{frame}

\begin{frame}[fragile]{Solutions for exercise using mutate()}

\begin{enumerate}
\def\labelenumi{\arabic{enumi}.}
\setcounter{enumi}{1}
\tightlist
\item
  Now using the code for question 1, filter schools where at least 50\%
  of students are proficient in math \textbf{\&} reading.
\end{enumerate}

\begin{Shaded}
\begin{Highlighting}[]
\NormalTok{school_v2 }\OperatorTok
\StringTok{  }\KeywordTok{select}\NormalTok{(num_prof_math, num_took_math, num_prof_read, num_took_read) }\OperatorTok
\StringTok{  }\KeywordTok{mutate}\NormalTok{(}\DataTypeTok{pct_prof_math =}\NormalTok{ num_prof_math}\OperatorTok{/}\NormalTok{num_took_math,}
         \DataTypeTok{pct_prof_read =}\NormalTok{ num_prof_read}\OperatorTok{/}\NormalTok{num_took_read) }\OperatorTok
\StringTok{  }\KeywordTok{filter}\NormalTok{(pct_prof_math }\OperatorTok{>=}\StringTok{ }\FloatTok{0.5} \OperatorTok{&}\StringTok{ }\NormalTok{pct_prof_read }\OperatorTok{>=}\StringTok{ }\FloatTok{0.5}\NormalTok{) }
\CommentTok{#> # A tibble: 7,760 x 6}
\CommentTok{#>    num_prof_math num_took_math num_prof_read num_took_read pct_prof_math}
\CommentTok{#>            <dbl>         <dbl>         <dbl>         <dbl>         <dbl>}
\CommentTok{#>  1         135.            260         149.            261         0.520}
\CommentTok{#>  2         299.            475         418             475         0.63 }
\CommentTok{#>  3         213.            410         332.            410         0.52 }
\CommentTok{#>  4          54.6           105          96.6           105         0.52 }
\CommentTok{#>  5         111.            121         118.            121         0.92 }
\CommentTok{#>  6        1057.           1994        1477.           2204         0.530}
\CommentTok{#>  7         100.            103         125.            128         0.975}
\CommentTok{#>  8          56.4            99          84.4           148         0.570}
\CommentTok{#>  9         445.            586         392.            594         0.76 }
\CommentTok{#> 10          56.0            59          53.1            61         0.95 }
\CommentTok{#> # ... with 7,750 more rows, and 1 more variable: pct_prof_read <dbl>}
\end{Highlighting}
\end{Shaded}

\end{frame}

\begin{frame}[fragile]{Solutions for exercise using mutate()}

\begin{enumerate}
\def\labelenumi{\arabic{enumi}.}
\setcounter{enumi}{2}
\tightlist
\item
  If you have time, count the number of schools from question 2.
\end{enumerate}

\begin{Shaded}
\begin{Highlighting}[]
\NormalTok{school_v2 }\OperatorTok
\StringTok{  }\KeywordTok{select}\NormalTok{(num_prof_math, num_took_math, num_prof_read, num_took_read) }\OperatorTok
\StringTok{  }\KeywordTok{mutate}\NormalTok{(}\DataTypeTok{pct_prof_math =}\NormalTok{ num_prof_math}\OperatorTok{/}\NormalTok{num_took_math,}
         \DataTypeTok{pct_prof_read =}\NormalTok{ num_prof_read}\OperatorTok{/}\NormalTok{num_took_read) }\OperatorTok
\StringTok{  }\KeywordTok{filter}\NormalTok{(pct_prof_math }\OperatorTok{>=}\StringTok{ }\FloatTok{0.5} \OperatorTok{&}\StringTok{ }\NormalTok{pct_prof_read }\OperatorTok{>=}\StringTok{ }\FloatTok{0.5}\NormalTok{) }\OperatorTok
\StringTok{  }\KeywordTok{count}\NormalTok{()}
\CommentTok{#> # A tibble: 1 x 1}
\CommentTok{#>       n}
\CommentTok{#>   <int>}
\CommentTok{#> 1  7760}
\end{Highlighting}
\end{Shaded}

\end{frame}

\subsection{Using ifelse() function within
mutate()}\label{using-ifelse-function-within-mutate}

\begin{frame}[fragile]{Using \texttt{ifelse()} function within
\texttt{mutate()}}

\begin{Shaded}
\begin{Highlighting}[]
\NormalTok{?if_else}
\end{Highlighting}
\end{Shaded}

\textbf{Description}

\begin{itemize}
\tightlist
\item
  if \texttt{condition} \texttt{TRUE}, assign a value; if
  \texttt{condition} \texttt{FALSE} assign a value
\end{itemize}

\textbf{Usage (i.e., syntax)}

\begin{itemize}
\tightlist
\item
  \texttt{if\_else(logical\ condition,\ true,\ false,\ missing\ =\ NULL)}
\end{itemize}

\textbf{Arguments}

\begin{itemize}
\tightlist
\item
  \texttt{logical\ condition}: a condition that evaluates to
  \texttt{TRUE} or \texttt{FALSE}
\item
  \texttt{true}: value to assign if condition \texttt{TRUE}
\item
  \texttt{false}: value to assign if condition \texttt{FALSE}
\end{itemize}

\textbf{Value}

\begin{itemize}
\tightlist
\item
  ``Where condition is TRUE, the matching value from true, where it's
  FALSE, the matching value from false, otherwise NA.''
\item
  missing values from ``input'' var are assigned missing values in
  ``output var'', unless you specify otherwise
\end{itemize}

\textbf{Example}: Create 0/1 indicator of whether high school got at
least one visit from Berkeley

\begin{Shaded}
\begin{Highlighting}[]
\NormalTok{school_v2 }\OperatorTok\StringTok{ }
\StringTok{  }\KeywordTok{mutate}\NormalTok{(}\DataTypeTok{got_visit_berkeley =} \KeywordTok{ifelse}\NormalTok{(visits_by_berkeley}\OperatorTok{>}\DecValTok{0}\NormalTok{,}\DecValTok{1}\NormalTok{,}\DecValTok{0}\NormalTok{)) }\OperatorTok
\StringTok{  }\KeywordTok{count}\NormalTok{(got_visit_berkeley)}
\end{Highlighting}
\end{Shaded}

\end{frame}

\begin{frame}[fragile]{Using \texttt{ifelse()} function within
\texttt{mutate()}}

Task

\begin{itemize}
\tightlist
\item
  Create 0/1 indicator if school has median income greater than
  \$100,000
\end{itemize}

Usually a good idea to investigate ``input'' variables before creating
analysis vars

\begin{Shaded}
\begin{Highlighting}[]
\NormalTok{school_v2 }\OperatorTok\StringTok{ }\KeywordTok{count}\NormalTok{(med_inc) }\CommentTok{# this isn't very helpful}

\NormalTok{school_v2 }\OperatorTok\StringTok{ }\KeywordTok{filter}\NormalTok{(}\KeywordTok{is.na}\NormalTok{(med_inc)) }\OperatorTok\StringTok{ }\KeywordTok{count}\NormalTok{(med_inc) }
\CommentTok{# shows number of obs w/ missing med_inc}
\end{Highlighting}
\end{Shaded}

Create variable

\begin{Shaded}
\begin{Highlighting}[]
\NormalTok{school_v2 }\OperatorTok\StringTok{ }\KeywordTok{select}\NormalTok{(med_inc) }\OperatorTok\StringTok{ }
\StringTok{  }\KeywordTok{mutate}\NormalTok{(}\DataTypeTok{inc_gt_100k=} \KeywordTok{ifelse}\NormalTok{(med_inc}\OperatorTok{>}\DecValTok{100000}\NormalTok{,}\DecValTok{1}\NormalTok{,}\DecValTok{0}\NormalTok{)) }\OperatorTok
\StringTok{  }\KeywordTok{count}\NormalTok{(inc_gt_100k) }\CommentTok{# note how NA values of med_inc treated}
\CommentTok{#> # A tibble: 3 x 2}
\CommentTok{#>   inc_gt_100k     n}
\CommentTok{#>         <dbl> <int>}
\CommentTok{#> 1           0 18632}
\CommentTok{#> 2           1  2045}
\CommentTok{#> 3          NA   624}
\end{Highlighting}
\end{Shaded}

\end{frame}

\begin{frame}[fragile]{Using \texttt{ifelse()} function within
\texttt{mutate()}}

\textbf{Task}

\begin{itemize}
\tightlist
\item
  Create 0/1 indicator variable \texttt{nonmiss\_math} which indicates
  whether school has non-missing values for the variable
  \texttt{num\_took\_math}

  \begin{itemize}
  \tightlist
  \item
    note: \texttt{num\_took\_math} refers to number of students at
    school that took state math proficiency test
  \end{itemize}
\end{itemize}

Usually a good to investigate ``input'' variables before creating
analysis vars

\begin{Shaded}
\begin{Highlighting}[]
\NormalTok{school_v2 }\OperatorTok\StringTok{ }\KeywordTok{count}\NormalTok{(num_took_math) }\CommentTok{# this isn't very helpful}
\NormalTok{school_v2 }\OperatorTok\StringTok{ }\KeywordTok{filter}\NormalTok{(}\KeywordTok{is.na}\NormalTok{(num_took_math)) }\OperatorTok\StringTok{ }\KeywordTok{count}\NormalTok{(num_took_math) }\CommentTok{# shows number of obs w/ missing med_inc}
\end{Highlighting}
\end{Shaded}

Create variable

\begin{Shaded}
\begin{Highlighting}[]
\NormalTok{school_v2 }\OperatorTok\StringTok{ }\KeywordTok{select}\NormalTok{(num_took_math) }\OperatorTok\StringTok{ }
\StringTok{  }\KeywordTok{mutate}\NormalTok{(}\DataTypeTok{nonmiss_math=} \KeywordTok{ifelse}\NormalTok{(}\OperatorTok{!}\KeywordTok{is.na}\NormalTok{(num_took_math),}\DecValTok{1}\NormalTok{,}\DecValTok{0}\NormalTok{)) }\OperatorTok
\StringTok{  }\KeywordTok{count}\NormalTok{(nonmiss_math) }\CommentTok{# note how NA values treated}
\CommentTok{#> # A tibble: 2 x 2}
\CommentTok{#>   nonmiss_math     n}
\CommentTok{#>          <dbl> <int>}
\CommentTok{#> 1            0  4103}
\CommentTok{#> 2            1 17198}
\end{Highlighting}
\end{Shaded}

\end{frame}

\begin{frame}[fragile]{Student exercises \texttt{ifelse()}}

\begin{enumerate}
\def\labelenumi{\arabic{enumi}.}
\tightlist
\item
  Using the object \texttt{school\_v2}, create 0/1 indicator variable
  \texttt{in\_state\_berkeley} that equals \texttt{1} if the high school
  is in the same state as UC Berkeley (i.e.,
  \texttt{state\_code=="CA"}).\\
\item
  Create 0/1 indicator \texttt{berkeley\_and\_irvine} of whether a
  school got at least one visit from UC Berkeley \textbf{AND} from UC
  Irvine.\\
\item
  Create 0/1 indicator \texttt{berkeley\_or\_irvine} of whether a school
  got at least one visit from UC Berkeley \textbf{OR} from UC Irvine.
\end{enumerate}

\end{frame}

\begin{frame}[fragile]{Exercise\texttt{ifelse()} solutions}

\begin{enumerate}
\def\labelenumi{\arabic{enumi}.}
\tightlist
\item
  Using the object \texttt{school\_v2}, create 0/1 indicator variable
  \texttt{in\_state\_berkeley} that equals \texttt{1} if the high school
  is in the same state as UC Berkeley (i.e.,
  \texttt{state\_code=="CA"}).
\end{enumerate}

\begin{Shaded}
\begin{Highlighting}[]
\NormalTok{school_v2 }\OperatorTok\StringTok{ }\KeywordTok{mutate}\NormalTok{(}\DataTypeTok{in_state_berkeley=}\KeywordTok{ifelse}\NormalTok{(state_code}\OperatorTok{==}\StringTok{"CA"}\NormalTok{,}\DecValTok{1}\NormalTok{,}\DecValTok{0}\NormalTok{)) }\OperatorTok
\StringTok{  }\KeywordTok{count}\NormalTok{(in_state_berkeley)}
\end{Highlighting}
\end{Shaded}

\end{frame}

\begin{frame}[fragile]{Exercise\texttt{ifelse()} solutions}

\begin{enumerate}
\def\labelenumi{\arabic{enumi}.}
\setcounter{enumi}{1}
\tightlist
\item
  Create 0/1 indicator \texttt{berkeley\_and\_irvine} of whether a
  school got at least one visit from UC Berkeley \textbf{AND} from UC
  Irvine.
\end{enumerate}

\begin{Shaded}
\begin{Highlighting}[]
\NormalTok{school_v2 }\OperatorTok\StringTok{ }
\StringTok{  }\KeywordTok{mutate}\NormalTok{(}\DataTypeTok{berkeley_and_irvine=}\KeywordTok{ifelse}\NormalTok{(visits_by_berkeley}\OperatorTok{>}\DecValTok{0} \OperatorTok{&}\StringTok{ }\NormalTok{visits_by_irvine}\OperatorTok{>}\DecValTok{0}\NormalTok{,}\DecValTok{1}\NormalTok{,}\DecValTok{0}\NormalTok{)) }\OperatorTok
\StringTok{  }\KeywordTok{count}\NormalTok{(berkeley_and_irvine)}
\end{Highlighting}
\end{Shaded}

\end{frame}

\begin{frame}[fragile]{Exercise\texttt{ifelse()} solutions}

\begin{enumerate}
\def\labelenumi{\arabic{enumi}.}
\setcounter{enumi}{2}
\tightlist
\item
  Create 0/1 indicator \texttt{berkeley\_or\_irvine} of whether a school
  got at least one visit from UC Berkeley \textbf{OR} from UC Irvine.
\end{enumerate}

\begin{Shaded}
\begin{Highlighting}[]
\NormalTok{school_v2 }\OperatorTok\StringTok{ }
\StringTok{  }\KeywordTok{mutate}\NormalTok{(}\DataTypeTok{berkeley_or_irvine=}\KeywordTok{ifelse}\NormalTok{(visits_by_berkeley}\OperatorTok{>}\DecValTok{0} \OperatorTok{|}\StringTok{ }\NormalTok{visits_by_irvine}\OperatorTok{>}\DecValTok{0}\NormalTok{,}\DecValTok{1}\NormalTok{,}\DecValTok{0}\NormalTok{)) }\OperatorTok
\StringTok{  }\KeywordTok{count}\NormalTok{(berkeley_or_irvine)}
\end{Highlighting}
\end{Shaded}

\end{frame}

\subsection{Using recode() function within
mutate()}\label{using-recode-function-within-mutate}

\begin{frame}[fragile]{Using \texttt{recode()} function within
\texttt{mutate()}}

\textbf{Description}: Recode values of a variable

\textbf{Usage (i.e., syntax)}

\begin{itemize}
\tightlist
\item
  recode(.x, \ldots{}, .default = NULL, .missing = NULL)
\end{itemize}

\textbf{Arguments} {[}see help file for further details{]}

\begin{itemize}
\tightlist
\item
  \texttt{.x} A vector (e.g., variable) to modify
\item
  \texttt{...} Specifications for recode, of the form
  \texttt{current\_value\ =\ new\ recoded\ value}
\item
  \texttt{.default}: If supplied, all values not otherwise matched given
  this value.
\item
  \texttt{.missing}: If supplied, any missing values in .x replaced by
  this value.
\end{itemize}

Example: Using data frame \texttt{wwlist}, create new 0/1 indicator
\texttt{public\_school} from variable \texttt{school\_type}

\begin{Shaded}
\begin{Highlighting}[]
\KeywordTok{str}\NormalTok{(wwlist}\OperatorTok{$}\NormalTok{school_type)}
\NormalTok{wwlist }\OperatorTok\StringTok{ }\KeywordTok{count}\NormalTok{(school_type)}

\NormalTok{wwlist_temp <-}\StringTok{ }\NormalTok{wwlist }\OperatorTok\StringTok{ }\KeywordTok{select}\NormalTok{(school_type) }\OperatorTok\StringTok{ }
\StringTok{  }\KeywordTok{mutate}\NormalTok{(}\DataTypeTok{public_school =} \KeywordTok{recode}\NormalTok{(school_type,}\StringTok{"public"}\NormalTok{ =}\StringTok{ }\DecValTok{1}\NormalTok{, }\StringTok{"private"}\NormalTok{ =}\StringTok{ }\DecValTok{0}\NormalTok{))}

\NormalTok{wwlist_temp }\OperatorTok\StringTok{ }\KeywordTok{head}\NormalTok{(}\DataTypeTok{n=}\DecValTok{10}\NormalTok{)}
\KeywordTok{str}\NormalTok{(wwlist_temp}\OperatorTok{$}\NormalTok{public_school)}
\NormalTok{wwlist_temp }\OperatorTok\StringTok{ }\KeywordTok{count}\NormalTok{(public_school)}
\KeywordTok{rm}\NormalTok{(wwlist_temp)}
\end{Highlighting}
\end{Shaded}

\end{frame}

\begin{frame}[fragile]{Using \texttt{recode()} function within
\texttt{mutate()}}

Recoding \texttt{school\_type} could have been accomplished using
\texttt{if\_else()}

\begin{itemize}
\tightlist
\item
  Use \texttt{recode()} when new variable has more than two categories
\end{itemize}

\textbf{Task}: Create \texttt{school\_catv2} based on
\texttt{school\_category} with these categories:

\begin{itemize}
\tightlist
\item
  ``regular''; ``alternative''; ``special''; ``vocational''
\end{itemize}

Investigate input var

\begin{Shaded}
\begin{Highlighting}[]
\KeywordTok{str}\NormalTok{(wwlist}\OperatorTok{$}\NormalTok{school_category)}
\NormalTok{wwlist }\OperatorTok\StringTok{ }\KeywordTok{count}\NormalTok{(school_category)}
\end{Highlighting}
\end{Shaded}

Recode

\begin{Shaded}
\begin{Highlighting}[]
\NormalTok{wwlist_temp <-}\StringTok{ }\NormalTok{wwlist }\OperatorTok\StringTok{ }\KeywordTok{select}\NormalTok{(school_category) }\OperatorTok\StringTok{ }
\StringTok{  }\KeywordTok{mutate}\NormalTok{(}\DataTypeTok{school_catv2 =} \KeywordTok{recode}\NormalTok{(school_category,}
    \StringTok{"Alternative Education School"}\NormalTok{ =}\StringTok{ "alternative"}\NormalTok{,}
    \StringTok{"Alternative/other"}\NormalTok{ =}\StringTok{ "alternative"}\NormalTok{,}
    \StringTok{"Regular elementary or secondary"}\NormalTok{ =}\StringTok{ "regular"}\NormalTok{,}
    \StringTok{"Regular School"}\NormalTok{ =}\StringTok{ "regular"}\NormalTok{,}
    \StringTok{"Special Education School"}\NormalTok{ =}\StringTok{ "special"}\NormalTok{,}
    \StringTok{"Special program emphasis"}\NormalTok{ =}\StringTok{ "special"}\NormalTok{,}
    \StringTok{"Vocational Education School"}\NormalTok{ =}\StringTok{ "vocational"}\NormalTok{)}
\NormalTok{  )}
\KeywordTok{str}\NormalTok{(wwlist_temp}\OperatorTok{$}\NormalTok{school_catv2)}
\NormalTok{wwlist_temp }\OperatorTok\StringTok{ }\KeywordTok{count}\NormalTok{(school_catv2)}
\NormalTok{wwlist }\OperatorTok\StringTok{ }\KeywordTok{count}\NormalTok{(school_category)}
\KeywordTok{rm}\NormalTok{(wwlist_temp)}
\end{Highlighting}
\end{Shaded}

\end{frame}

\begin{frame}[fragile]{Using \texttt{recode()} within \texttt{mutate()}
{[}do in pairs/groups{]}}

\textbf{Task}: Create \texttt{school\_catv2} based on
\texttt{school\_category} with these categories:

\begin{itemize}
\tightlist
\item
  ``regular''; ``alternative''; ``special''; ``vocational''
\item
  This time use the \texttt{.missing} argument to recode \texttt{NAs} to
  ``unknown''
\end{itemize}

\begin{Shaded}
\begin{Highlighting}[]
\NormalTok{wwlist_temp <-}\StringTok{ }\NormalTok{wwlist }\OperatorTok\StringTok{ }\KeywordTok{select}\NormalTok{(school_category) }\OperatorTok\StringTok{ }
\StringTok{  }\KeywordTok{mutate}\NormalTok{(}\DataTypeTok{school_catv2 =} \KeywordTok{recode}\NormalTok{(school_category,}
    \StringTok{"Alternative Education School"}\NormalTok{ =}\StringTok{ "alternative"}\NormalTok{,}
    \StringTok{"Alternative/other"}\NormalTok{ =}\StringTok{ "alternative"}\NormalTok{,}
    \StringTok{"Regular elementary or secondary"}\NormalTok{ =}\StringTok{ "regular"}\NormalTok{,}
    \StringTok{"Regular School"}\NormalTok{ =}\StringTok{ "regular"}\NormalTok{,}
    \StringTok{"Special Education School"}\NormalTok{ =}\StringTok{ "special"}\NormalTok{,}
    \StringTok{"Special program emphasis"}\NormalTok{ =}\StringTok{ "special"}\NormalTok{,}
    \StringTok{"Vocational Education School"}\NormalTok{ =}\StringTok{ "vocational"}\NormalTok{,}
    \DataTypeTok{.missing =} \StringTok{"unknown"}\NormalTok{)}
\NormalTok{  )}
\KeywordTok{str}\NormalTok{(wwlist_temp}\OperatorTok{$}\NormalTok{school_catv2)}
\NormalTok{wwlist_temp }\OperatorTok\StringTok{ }\KeywordTok{count}\NormalTok{(school_catv2)}
\NormalTok{wwlist }\OperatorTok\StringTok{ }\KeywordTok{count}\NormalTok{(school_category)}
\KeywordTok{rm}\NormalTok{(wwlist_temp)}
\end{Highlighting}
\end{Shaded}

\end{frame}

\begin{frame}[fragile]{Using \texttt{recode()} within \texttt{mutate()}
{[}do in pairs/groups{]}}

\textbf{Task}: Create \texttt{school\_catv2} based on
\texttt{school\_category} with these categories:

\begin{itemize}
\tightlist
\item
  ``regular''; ``alternative''; ``special''; ``vocational''
\item
  This time use the \texttt{.default} argument to assign the value
  ``regular''
\end{itemize}

\begin{Shaded}
\begin{Highlighting}[]
\NormalTok{wwlist_temp <-}\StringTok{ }\NormalTok{wwlist }\OperatorTok\StringTok{ }\KeywordTok{select}\NormalTok{(school_category) }\OperatorTok\StringTok{ }
\StringTok{  }\KeywordTok{mutate}\NormalTok{(}\DataTypeTok{school_catv2 =} \KeywordTok{recode}\NormalTok{(school_category,}
    \StringTok{"Alternative Education School"}\NormalTok{ =}\StringTok{ "alternative"}\NormalTok{,}
    \StringTok{"Alternative/other"}\NormalTok{ =}\StringTok{ "alternative"}\NormalTok{,}
    \StringTok{"Special Education School"}\NormalTok{ =}\StringTok{ "special"}\NormalTok{,}
    \StringTok{"Special program emphasis"}\NormalTok{ =}\StringTok{ "special"}\NormalTok{,}
    \StringTok{"Vocational Education School"}\NormalTok{ =}\StringTok{ "vocational"}\NormalTok{,}
    \DataTypeTok{.default =} \StringTok{"regular"}\NormalTok{)}
\NormalTok{  )}
\KeywordTok{str}\NormalTok{(wwlist_temp}\OperatorTok{$}\NormalTok{school_catv2)}
\NormalTok{wwlist_temp }\OperatorTok\StringTok{ }\KeywordTok{count}\NormalTok{(school_catv2)}
\NormalTok{wwlist }\OperatorTok\StringTok{ }\KeywordTok{count}\NormalTok{(school_category)}
\KeywordTok{rm}\NormalTok{(wwlist_temp)}
\end{Highlighting}
\end{Shaded}

\end{frame}

\begin{frame}[fragile]{Using \texttt{recode()} within \texttt{mutate()}
{[}do in pairs/groups{]}}

\textbf{Task}: Create \texttt{school\_catv2} based on
\texttt{school\_category} with these categories:

\begin{itemize}
\tightlist
\item
  This time create a numeric variable rather than character:

  \begin{itemize}
  \tightlist
  \item
    \texttt{1} for ``regular''; \texttt{2} for ``alternative'';
    \texttt{3} for ``special''; \texttt{4} for ``vocational''
  \end{itemize}
\end{itemize}

\begin{Shaded}
\begin{Highlighting}[]
\NormalTok{wwlist_temp <-}\StringTok{ }\NormalTok{wwlist }\OperatorTok\StringTok{ }\KeywordTok{select}\NormalTok{(school_category) }\OperatorTok\StringTok{ }
\StringTok{  }\KeywordTok{mutate}\NormalTok{(}\DataTypeTok{school_catv2 =} \KeywordTok{recode}\NormalTok{(school_category,}
    \StringTok{"Alternative Education School"}\NormalTok{ =}\StringTok{ }\DecValTok{2}\NormalTok{,}
    \StringTok{"Alternative/other"}\NormalTok{ =}\StringTok{ }\DecValTok{2}\NormalTok{,}
    \StringTok{"Regular elementary or secondary"}\NormalTok{ =}\StringTok{ }\DecValTok{1}\NormalTok{,}
    \StringTok{"Regular School"}\NormalTok{ =}\StringTok{ }\DecValTok{1}\NormalTok{,}
    \StringTok{"Special Education School"}\NormalTok{ =}\StringTok{ }\DecValTok{3}\NormalTok{,}
    \StringTok{"Special program emphasis"}\NormalTok{ =}\StringTok{ }\DecValTok{3}\NormalTok{,}
    \StringTok{"Vocational Education School"}\NormalTok{ =}\StringTok{ }\DecValTok{4}\NormalTok{)}
\NormalTok{  )}
\KeywordTok{str}\NormalTok{(wwlist_temp}\OperatorTok{$}\NormalTok{school_catv2)}
\NormalTok{wwlist_temp }\OperatorTok\StringTok{ }\KeywordTok{count}\NormalTok{(school_catv2)}
\NormalTok{wwlist }\OperatorTok\StringTok{ }\KeywordTok{count}\NormalTok{(school_category)}
\KeywordTok{rm}\NormalTok{(wwlist_temp)}
\end{Highlighting}
\end{Shaded}

\end{frame}

\begin{frame}[fragile]{Student exercise using \texttt{recode()} within
\texttt{mutate()}}

\begin{enumerate}
\def\labelenumi{\arabic{enumi}.}
\tightlist
\item
  Using object \texttt{df\_event}, create \texttt{event\_typev2} based
  on \texttt{event\_type} with these categories:
\end{enumerate}

\begin{itemize}
\tightlist
\item
  \texttt{1} for ``2yr college''; \texttt{2} for ``4yr college'';
  \texttt{3} for ``other''; \texttt{4} for ``private hs''; \texttt{5}
  for ``public hs''\\
\item
  Assign new object \texttt{df\_event\_temp}
\end{itemize}

\begin{enumerate}
\def\labelenumi{\arabic{enumi}.}
\setcounter{enumi}{1}
\tightlist
\item
  This time use the \texttt{.default} argument to assign the value
  \texttt{5} for ``public hs''
\end{enumerate}

\end{frame}

\begin{frame}[fragile]{Exercise using \texttt{recode()} within
\texttt{mutate()} solutions}

\begin{Shaded}
\begin{Highlighting}[]
\CommentTok{#load(url("https://github.com/ozanj/rclass/raw/master/data/recruiting/recruit_event_somevars.RData"))}
\CommentTok{#names(df_event)}
\KeywordTok{str}\NormalTok{(df_event}\OperatorTok{$}\NormalTok{event_type)}
\CommentTok{#>  chr [1:18680] "public hs" "public hs" "public hs" "public hs" ...}
\NormalTok{df_event }\OperatorTok\StringTok{ }\KeywordTok{count}\NormalTok{(event_type)}
\CommentTok{#> # A tibble: 5 x 2}
\CommentTok{#>   event_type      n}
\CommentTok{#>   <chr>       <int>}
\CommentTok{#> 1 2yr college   951}
\CommentTok{#> 2 4yr college   531}
\CommentTok{#> 3 other        2001}
\CommentTok{#> 4 private hs   3774}
\CommentTok{#> 5 public hs   11423}
\end{Highlighting}
\end{Shaded}

\begin{enumerate}
\def\labelenumi{\arabic{enumi}.}
\tightlist
\item
  Using object \texttt{df\_event}, create \texttt{event\_typev2} based
  on \texttt{event\_type} with these categories:
\end{enumerate}

\begin{itemize}
\tightlist
\item
  \texttt{1} for ``2yr college''; \texttt{2} for ``4yr college'';
  \texttt{3} for ``other''; \texttt{4} for ``private hs''; \texttt{5}
  for ``public hs''\\
  Assign new object \texttt{df\_event\_temp}
\end{itemize}

\begin{Shaded}
\begin{Highlighting}[]
\NormalTok{df_event_temp <-}\StringTok{ }\NormalTok{df_event }\OperatorTok\StringTok{ }
\StringTok{  }\KeywordTok{select}\NormalTok{(event_type) }\OperatorTok
\StringTok{  }\KeywordTok{mutate}\NormalTok{(}\DataTypeTok{event_typev2 =} \KeywordTok{recode}\NormalTok{(event_type,}
                              \StringTok{"2yr college"}\NormalTok{ =}\StringTok{ }\DecValTok{1}\NormalTok{,}
                              \StringTok{"4yr college"}\NormalTok{ =}\StringTok{ }\DecValTok{2}\NormalTok{,}
                              \StringTok{"other"}\NormalTok{ =}\StringTok{ }\DecValTok{3}\NormalTok{,}
                              \StringTok{"private hs"}\NormalTok{ =}\StringTok{ }\DecValTok{4}\NormalTok{,}
                              \StringTok{"public hs"}\NormalTok{ =}\StringTok{ }\DecValTok{5}\NormalTok{)}
\NormalTok{         )}
\KeywordTok{str}\NormalTok{(df_event_temp}\OperatorTok{$}\NormalTok{event_typev2)}
\CommentTok{#>  num [1:18680] 5 5 5 5 5 4 4 5 4 5 ...}
\NormalTok{df_event_temp }\OperatorTok\StringTok{ }\KeywordTok{count}\NormalTok{(event_typev2)}
\CommentTok{#> # A tibble: 5 x 2}
\CommentTok{#>   event_typev2     n}
\CommentTok{#>          <dbl> <int>}
\CommentTok{#> 1            1   951}
\CommentTok{#> 2            2   531}
\CommentTok{#> 3            3  2001}
\CommentTok{#> 4            4  3774}
\CommentTok{#> 5            5 11423}
\NormalTok{df_event }\OperatorTok\StringTok{ }\KeywordTok{count}\NormalTok{(event_type)}
\CommentTok{#> # A tibble: 5 x 2}
\CommentTok{#>   event_type      n}
\CommentTok{#>   <chr>       <int>}
\CommentTok{#> 1 2yr college   951}
\CommentTok{#> 2 4yr college   531}
\CommentTok{#> 3 other        2001}
\CommentTok{#> 4 private hs   3774}
\CommentTok{#> 5 public hs   11423}
\end{Highlighting}
\end{Shaded}

\end{frame}

\begin{frame}[fragile]{Exercise using \texttt{recode()} within
\texttt{mutate()} solutions}

\begin{enumerate}
\def\labelenumi{\arabic{enumi}.}
\setcounter{enumi}{1}
\tightlist
\item
  This time use the \texttt{.default} argument to assign the value
  \texttt{5} for ``public hs''
\end{enumerate}

\begin{Shaded}
\begin{Highlighting}[]
\NormalTok{df_event }\OperatorTok\StringTok{ }\KeywordTok{select}\NormalTok{(event_type) }\OperatorTok\StringTok{ }
\StringTok{  }\KeywordTok{mutate}\NormalTok{(}\DataTypeTok{event_typev2 =} \KeywordTok{recode}\NormalTok{(event_type,}
    \StringTok{"2yr college"}\NormalTok{ =}\StringTok{ }\DecValTok{1}\NormalTok{,}
    \StringTok{"4yr college"}\NormalTok{ =}\StringTok{ }\DecValTok{2}\NormalTok{,}
    \StringTok{"other"}\NormalTok{ =}\StringTok{ }\DecValTok{3}\NormalTok{,}
    \StringTok{"private hs"}\NormalTok{ =}\StringTok{ }\DecValTok{4}\NormalTok{,}
    \DataTypeTok{.default =} \DecValTok{5}\NormalTok{)}
\NormalTok{  )}
\CommentTok{#> # A tibble: 18,680 x 2}
\CommentTok{#>    event_type event_typev2}
\CommentTok{#>    <chr>             <dbl>}
\CommentTok{#>  1 public hs             5}
\CommentTok{#>  2 public hs             5}
\CommentTok{#>  3 public hs             5}
\CommentTok{#>  4 public hs             5}
\CommentTok{#>  5 public hs             5}
\CommentTok{#>  6 private hs            4}
\CommentTok{#>  7 private hs            4}
\CommentTok{#>  8 public hs             5}
\CommentTok{#>  9 private hs            4}
\CommentTok{#> 10 public hs             5}
\CommentTok{#> # ... with 18,670 more rows}
\KeywordTok{str}\NormalTok{(df_event_temp}\OperatorTok{$}\NormalTok{event_typev2)}
\CommentTok{#>  num [1:18680] 5 5 5 5 5 4 4 5 4 5 ...}
\NormalTok{df_event_temp }\OperatorTok\StringTok{ }\KeywordTok{count}\NormalTok{(event_typev2)}
\CommentTok{#> # A tibble: 5 x 2}
\CommentTok{#>   event_typev2     n}
\CommentTok{#>          <dbl> <int>}
\CommentTok{#> 1            1   951}
\CommentTok{#> 2            2   531}
\CommentTok{#> 3            3  2001}
\CommentTok{#> 4            4  3774}
\CommentTok{#> 5            5 11423}
\NormalTok{df_event }\OperatorTok\StringTok{ }\KeywordTok{count}\NormalTok{(event_type)}
\CommentTok{#> # A tibble: 5 x 2}
\CommentTok{#>   event_type      n}
\CommentTok{#>   <chr>       <int>}
\CommentTok{#> 1 2yr college   951}
\CommentTok{#> 2 4yr college   531}
\CommentTok{#> 3 other        2001}
\CommentTok{#> 4 private hs   3774}
\CommentTok{#> 5 public hs   11423}
\end{Highlighting}
\end{Shaded}

\end{frame}

\subsection{Using case\_when() function within
mutate()}\label{using-case_when-function-within-mutate}

\begin{frame}[fragile]{Using \texttt{case\_when()} function within
\texttt{mutate()}}

\textbf{Description} Useful when the variable you want to create is more
complicated than variables that can be created using \texttt{ifelse()}
or \texttt{recode()}

\begin{itemize}
\tightlist
\item
  For example, useful when new variable is a function of multiple
  ``input'' variables
\end{itemize}

\textbf{Usage (i.e., syntax)}: \texttt{case\_when(...)}

\textbf{Arguments} {[}from help file; see help file for more details{]}

\begin{itemize}
\tightlist
\item
  \texttt{...}: A sequence of two-sided formulas.

  \begin{itemize}
  \tightlist
  \item
    The left hand side (LHS) determines which values match this case.

    \begin{itemize}
    \tightlist
    \item
      LHS must evaluate to a logical vector.\\
    \end{itemize}
  \item
    The right hand side (RHS) provides the replacement value.
  \end{itemize}
\end{itemize}

\textbf{Example task}: Using data frame \texttt{wwlist} and input vars
\texttt{state} and \texttt{firstgen}, create a 4-category var with
following categories:

\begin{itemize}
\tightlist
\item
  ``instate\_firstgen''; ``instate\_nonfirstgen'';
  ``outstate\_firstgen''; ``outstate\_nonfirstgen''
\end{itemize}

\begin{Shaded}
\begin{Highlighting}[]
\NormalTok{wwlist_temp <-}\StringTok{ }\NormalTok{wwlist }\OperatorTok\StringTok{ }\KeywordTok{select}\NormalTok{(state,firstgen) }\OperatorTok
\StringTok{  }\KeywordTok{mutate}\NormalTok{(}\DataTypeTok{state_gen =} \KeywordTok{case_when}\NormalTok{(}
\NormalTok{    state }\OperatorTok{==}\StringTok{ "WA"} \OperatorTok{&}\StringTok{ }\NormalTok{firstgen }\OperatorTok{==}\StringTok{"Y"} \OperatorTok{~}\StringTok{ "instate_firstgen"}\NormalTok{,}
\NormalTok{    state }\OperatorTok{==}\StringTok{ "WA"} \OperatorTok{&}\StringTok{ }\NormalTok{firstgen }\OperatorTok{==}\StringTok{"N"} \OperatorTok{~}\StringTok{ "instate_nonfirstgen"}\NormalTok{,}
\NormalTok{    state }\OperatorTok{!=}\StringTok{ "WA"} \OperatorTok{&}\StringTok{ }\NormalTok{firstgen }\OperatorTok{==}\StringTok{"Y"} \OperatorTok{~}\StringTok{ "outstate_firstgen"}\NormalTok{,}
\NormalTok{    state }\OperatorTok{!=}\StringTok{ "WA"} \OperatorTok{&}\StringTok{ }\NormalTok{firstgen }\OperatorTok{==}\StringTok{"N"} \OperatorTok{~}\StringTok{ "outstate_nonfirstgen"}\NormalTok{)}
\NormalTok{  )}
\KeywordTok{str}\NormalTok{(wwlist_temp}\OperatorTok{$}\NormalTok{state_gen)}
\NormalTok{wwlist_temp }\OperatorTok\StringTok{ }\KeywordTok{count}\NormalTok{(state_gen)}
\end{Highlighting}
\end{Shaded}

\end{frame}

\begin{frame}[fragile]{Using \texttt{case\_when()} function within
\texttt{mutate()}}

\textbf{Task}: Using data frame \texttt{wwlist} and input vars
\texttt{state} and \texttt{firstgen}, create a 4-category var with
following categories:

\begin{itemize}
\tightlist
\item
  ``instate\_firstgen''; ``instate\_nonfirstgen'';
  ``outstate\_firstgen''; ``outstate\_nonfirstgen''
\end{itemize}

Let's take a closer look at how values of inputs are coded into values
of outputs

\begin{Shaded}
\begin{Highlighting}[]
\NormalTok{wwlist }\OperatorTok\StringTok{ }\KeywordTok{select}\NormalTok{(state,firstgen) }\OperatorTok\StringTok{ }\KeywordTok{str}\NormalTok{()}
\KeywordTok{count}\NormalTok{(wwlist,state)}
\KeywordTok{count}\NormalTok{(wwlist,firstgen)}

\NormalTok{wwlist_temp <-}\StringTok{ }\NormalTok{wwlist }\OperatorTok\StringTok{ }\KeywordTok{select}\NormalTok{(state,firstgen) }\OperatorTok
\StringTok{  }\KeywordTok{mutate}\NormalTok{(}\DataTypeTok{state_gen =} \KeywordTok{case_when}\NormalTok{(}
\NormalTok{    state }\OperatorTok{==}\StringTok{ "WA"} \OperatorTok{&}\StringTok{ }\NormalTok{firstgen }\OperatorTok{==}\StringTok{"Y"} \OperatorTok{~}\StringTok{ "instate_firstgen"}\NormalTok{,}
\NormalTok{    state }\OperatorTok{==}\StringTok{ "WA"} \OperatorTok{&}\StringTok{ }\NormalTok{firstgen }\OperatorTok{==}\StringTok{"N"} \OperatorTok{~}\StringTok{ "instate_nonfirstgen"}\NormalTok{,}
\NormalTok{    state }\OperatorTok{!=}\StringTok{ "WA"} \OperatorTok{&}\StringTok{ }\NormalTok{firstgen }\OperatorTok{==}\StringTok{"Y"} \OperatorTok{~}\StringTok{ "outstate_firstgen"}\NormalTok{,}
\NormalTok{    state }\OperatorTok{!=}\StringTok{ "WA"} \OperatorTok{&}\StringTok{ }\NormalTok{firstgen }\OperatorTok{==}\StringTok{"N"} \OperatorTok{~}\StringTok{ "outstate_nonfirstgen"}\NormalTok{)}
\NormalTok{  )}

\NormalTok{wwlist_temp }\OperatorTok\StringTok{ }\KeywordTok{count}\NormalTok{(state_gen)}
\NormalTok{wwlist_temp }\OperatorTok\StringTok{ }\KeywordTok{filter}\NormalTok{(}\KeywordTok{is.na}\NormalTok{(state)) }\OperatorTok\StringTok{ }\KeywordTok{count}\NormalTok{(state_gen)}
\NormalTok{wwlist_temp }\OperatorTok\StringTok{ }\KeywordTok{filter}\NormalTok{(}\KeywordTok{is.na}\NormalTok{(firstgen)) }\OperatorTok\StringTok{ }\KeywordTok{count}\NormalTok{(state_gen)}
\end{Highlighting}
\end{Shaded}

\end{frame}

\begin{frame}[fragile]{Student exercise using \texttt{case\_when()}
within \texttt{mutate()}}

\begin{enumerate}
\def\labelenumi{\arabic{enumi}.}
\tightlist
\item
  Using the object \texttt{school\_v2} and input vars
  \texttt{school\_type}, and \texttt{state\_code} , create a 4-category
  var with following categories:\\
\end{enumerate}

\begin{itemize}
\tightlist
\item
  ``instate\_public''; ``instate\_private''; ``outstate\_public'';
  ``outstate\_private''
\end{itemize}

\end{frame}

\begin{frame}[fragile]{Exercise using \texttt{case\_when()} within
\texttt{mutate()} solution}

Investigate

\begin{Shaded}
\begin{Highlighting}[]
\NormalTok{school_v2 }\OperatorTok\StringTok{ }\KeywordTok{select}\NormalTok{(state_code,school_type) }\OperatorTok\StringTok{ }\KeywordTok{str}\NormalTok{()}
\CommentTok{#> Classes 'tbl_df', 'tbl' and 'data.frame':    21301 obs. of  2 variables:}
\CommentTok{#>  $ state_code : chr  "AK" "AK" "AK" "AK" ...}
\CommentTok{#>  $ school_type: chr  "public" "public" "public" "public" ...}
\KeywordTok{count}\NormalTok{(school_v2,state_code)}
\CommentTok{#> # A tibble: 51 x 2}
\CommentTok{#>    state_code     n}
\CommentTok{#>    <chr>      <int>}
\CommentTok{#>  1 AK            84}
\CommentTok{#>  2 AL           440}
\CommentTok{#>  3 AR           266}
\CommentTok{#>  4 AZ           455}
\CommentTok{#>  5 CA          1770}
\CommentTok{#>  6 CO           350}
\CommentTok{#>  7 CT           252}
\CommentTok{#>  8 DC            46}
\CommentTok{#>  9 DE            51}
\CommentTok{#> 10 FL           792}
\CommentTok{#> # ... with 41 more rows}
\KeywordTok{count}\NormalTok{(school_v2,school_type)}
\CommentTok{#> # A tibble: 2 x 2}
\CommentTok{#>   school_type     n}
\CommentTok{#>   <chr>       <int>}
\CommentTok{#> 1 private      3822}
\CommentTok{#> 2 public      17479}
\end{Highlighting}
\end{Shaded}

\end{frame}

\begin{frame}[fragile]{Exercise using \texttt{case\_when()} within
\texttt{mutate()} solution}

\begin{enumerate}
\def\labelenumi{\arabic{enumi}.}
\tightlist
\item
  Using the object \texttt{school\_v2} and input vars
  \texttt{school\_type}, and \texttt{state\_code} , create a 4-category
  var with following categories:\\
\end{enumerate}

\begin{itemize}
\tightlist
\item
  ``instate\_public''; ``instate\_private''; ``outstate\_public'';
  ``outstate\_private''
\end{itemize}

\begin{Shaded}
\begin{Highlighting}[]
\NormalTok{school_v2_temp <-}\StringTok{ }\NormalTok{school_v2 }\OperatorTok\StringTok{ }\KeywordTok{select}\NormalTok{(state_code,school_type) }\OperatorTok
\StringTok{  }\KeywordTok{mutate}\NormalTok{(}\DataTypeTok{state_type =} \KeywordTok{case_when}\NormalTok{(}
\NormalTok{    state_code }\OperatorTok{==}\StringTok{ "CA"} \OperatorTok{&}\StringTok{ }\NormalTok{school_type }\OperatorTok{==}\StringTok{ "public"}  \OperatorTok{~}\StringTok{ "instate_public"}\NormalTok{,}
\NormalTok{    state_code }\OperatorTok{==}\StringTok{ "CA"} \OperatorTok{&}\StringTok{ }\NormalTok{school_type }\OperatorTok{==}\StringTok{ "private"} \OperatorTok{~}\StringTok{ "instate_private"}\NormalTok{,}
\NormalTok{    state_code }\OperatorTok{!=}\StringTok{ "CA"} \OperatorTok{&}\StringTok{ }\NormalTok{school_type }\OperatorTok{==}\StringTok{ "public"} \OperatorTok{~}\StringTok{ "outstate_public"}\NormalTok{,}
\NormalTok{    state_code }\OperatorTok{!=}\StringTok{ "CA"} \OperatorTok{&}\StringTok{ }\NormalTok{school_type }\OperatorTok{==}\StringTok{ "private"} \OperatorTok{~}\StringTok{ "outstate_private"}\NormalTok{)}
\NormalTok{  )}

\NormalTok{school_v2_temp }\OperatorTok\StringTok{ }\KeywordTok{count}\NormalTok{(state_type)}
\CommentTok{#> # A tibble: 4 x 2}
\CommentTok{#>   state_type           n}
\CommentTok{#>   <chr>            <int>}
\CommentTok{#> 1 instate_private    366}
\CommentTok{#> 2 instate_public    1404}
\CommentTok{#> 3 outstate_private  3456}
\CommentTok{#> 4 outstate_public  16075}
\CommentTok{#school_v2_temp %>% filter(is.na(state_code)) %>% count(state_type) #no missing}
\CommentTok{#school_v2_temp %>% filter(is.na(school_type)) %>% count(state_type) #no missing}
\end{Highlighting}
\end{Shaded}

\end{frame}

\begin{frame}[fragile]{Mutate to create indicator variables}

We often create dichotomous (0/1) indicator variables of whether
something happened (or whether something is TRUE)

\begin{itemize}
\tightlist
\item
  Variables that are of substantive interest to project

  \begin{itemize}
  \tightlist
  \item
    e.g., did student graduate from college
  \end{itemize}
\item
  Variables that help you investigate data, check quality

  \begin{itemize}
  \tightlist
  \item
    e.g., indicator of whether an observation is missing/non-missing for
    a particular variable
  \end{itemize}
\end{itemize}

Let's conduct some investigations of \texttt{df\_school}, which has one
observation for each high school

\end{frame}

\begin{frame}[fragile]{Creating indicators for \texttt{df\_schoolv2}
data frame}

Create TRUE/FALSE indicator that median household income greater than
\$50,000

\begin{Shaded}
\begin{Highlighting}[]
\NormalTok{school_v2_temp <-}\StringTok{ }\NormalTok{school_v2 }\OperatorTok\StringTok{ }\KeywordTok{mutate}\NormalTok{(}\DataTypeTok{incgt50k =}\NormalTok{ med_inc}\OperatorTok{>}\DecValTok{50000}\NormalTok{)}

\NormalTok{school_v2_temp }\OperatorTok\StringTok{ }\KeywordTok{select}\NormalTok{(med_inc, incgt50k) }\OperatorTok\StringTok{ }\KeywordTok{head}\NormalTok{(}\DataTypeTok{n=}\DecValTok{3}\NormalTok{)}
\CommentTok{#> # A tibble: 3 x 2}
\CommentTok{#>   med_inc incgt50k}
\CommentTok{#>     <dbl> <lgl>   }
\CommentTok{#> 1   76160 TRUE    }
\CommentTok{#> 2   76160 TRUE    }
\CommentTok{#> 3      NA NA}

\NormalTok{school_v2_temp }\OperatorTok\StringTok{ }\KeywordTok{filter}\NormalTok{(}\KeywordTok{is.na}\NormalTok{(med_inc)) }\OperatorTok\StringTok{ }\KeywordTok{count}\NormalTok{(incgt50k)}
\CommentTok{#> # A tibble: 1 x 2}
\CommentTok{#>   incgt50k     n}
\CommentTok{#>   <lgl>    <int>}
\CommentTok{#> 1 NA         624}
\end{Highlighting}
\end{Shaded}

Important takeaway:

\begin{itemize}
\tightlist
\item
  Variable created by \texttt{mutate()} equals \texttt{NA} for obs if
  input variable to \texttt{mutate()} is missing for that obs. This is a
  good thing!
\end{itemize}

\end{frame}

\subsection{Alternative approach to creating 0/1
indicators}\label{alternative-approach-to-creating-01-indicators}

\begin{frame}[fragile]{Creating indicators for \texttt{school\_v2} data
frame}

PATRICIA - I WROTE THESE BELOW SLIDES IN SUMMER BEFORE I KNEW ABOUT
\texttt{ifelse()}; YOU MAKE A RECOMMENDATION TO ME ABOUT WHETHER THESE
SLIDES SHOULD BE CUT Create TRUE/FALSE indicator that school is less
than 50 percent white

\begin{Shaded}
\begin{Highlighting}[]
\NormalTok{school_v2_temp <-}\StringTok{ }\NormalTok{school_v2 }\OperatorTok\StringTok{ }\KeywordTok{mutate}\NormalTok{(}\DataTypeTok{lt50pctwhite =}\NormalTok{ pct_white}\OperatorTok{<}\DecValTok{50}\NormalTok{)}
\NormalTok{school_v2_temp }\OperatorTok\StringTok{ }\KeywordTok{select}\NormalTok{(pct_white,lt50pctwhite) }\OperatorTok\StringTok{ }\KeywordTok{head}\NormalTok{(}\DataTypeTok{n=}\DecValTok{3}\NormalTok{)}
\CommentTok{#> # A tibble: 3 x 2}
\CommentTok{#>   pct_white lt50pctwhite}
\CommentTok{#>       <dbl> <lgl>       }
\CommentTok{#> 1      11.8 TRUE        }
\CommentTok{#> 2       0   TRUE        }
\CommentTok{#> 3       0   TRUE}
\KeywordTok{str}\NormalTok{(school_v2_temp}\OperatorTok{$}\NormalTok{lt50pctwhite)}
\CommentTok{#>  logi [1:21301] TRUE TRUE TRUE TRUE TRUE TRUE ...}
\end{Highlighting}
\end{Shaded}

Create 0/1 integer indicator rather than logical indicator

\begin{Shaded}
\begin{Highlighting}[]
\NormalTok{df_schoolv2_temp <-}\StringTok{ }\NormalTok{school_v2 }\OperatorTok\StringTok{ }\KeywordTok{mutate}\NormalTok{(}\DataTypeTok{lt50pctwhite =} \KeywordTok{as.integer}\NormalTok{(pct_white}\OperatorTok{<}\DecValTok{50}\NormalTok{))}
\NormalTok{df_schoolv2_temp }\OperatorTok\StringTok{ }\KeywordTok{select}\NormalTok{(pct_white,lt50pctwhite) }\OperatorTok\StringTok{ }\KeywordTok{head}\NormalTok{(}\DataTypeTok{n=}\DecValTok{3}\NormalTok{)}
\CommentTok{#> # A tibble: 3 x 2}
\CommentTok{#>   pct_white lt50pctwhite}
\CommentTok{#>       <dbl>        <int>}
\CommentTok{#> 1      11.8            1}
\CommentTok{#> 2       0              1}
\CommentTok{#> 3       0              1}
\KeywordTok{str}\NormalTok{(df_schoolv2_temp}\OperatorTok{$}\NormalTok{lt50pctwhite)}
\CommentTok{#>  int [1:21301] 1 1 1 1 1 1 1 1 1 1 ...}
\end{Highlighting}
\end{Shaded}

\end{frame}

\begin{frame}[fragile]{Student exercises}

0/1 indicators of whether school received visit from each university

\begin{Shaded}
\begin{Highlighting}[]
\NormalTok{school_v2 }\OperatorTok\StringTok{ }\KeywordTok{count}\NormalTok{(visits_by_berkeley)}
\CommentTok{#> # A tibble: 4 x 2}
\CommentTok{#>   visits_by_berkeley     n}
\CommentTok{#>                <int> <int>}
\CommentTok{#> 1                  0 20732}
\CommentTok{#> 2                  1   528}
\CommentTok{#> 3                  2    36}
\CommentTok{#> 4                  3     5}
\NormalTok{school_v2_temp <-}\StringTok{ }\NormalTok{school_v2 }\OperatorTok\StringTok{ }\KeywordTok{mutate}\NormalTok{(}\DataTypeTok{yesvis_berkeley =}\NormalTok{ visits_by_berkeley}\OperatorTok{>}\DecValTok{0}\NormalTok{)}

\NormalTok{school_v2_temp }\OperatorTok\StringTok{ }\KeywordTok{filter}\NormalTok{(visits_by_berkeley}\OperatorTok{>}\DecValTok{0}\NormalTok{) }\OperatorTok\StringTok{ }\KeywordTok{select}\NormalTok{(visits_by_berkeley,visits_by_berkeley)}
\CommentTok{#> # A tibble: 569 x 1}
\CommentTok{#>    visits_by_berkeley}
\CommentTok{#>                 <int>}
\CommentTok{#>  1                  2}
\CommentTok{#>  2                  2}
\CommentTok{#>  3                  2}
\CommentTok{#>  4                  1}
\CommentTok{#>  5                  1}
\CommentTok{#>  6                  1}
\CommentTok{#>  7                  1}
\CommentTok{#>  8                  1}
\CommentTok{#>  9                  1}
\CommentTok{#> 10                  1}
\CommentTok{#> # ... with 559 more rows}
\NormalTok{school_v2_temp }\OperatorTok\StringTok{ }\KeywordTok{filter}\NormalTok{(visits_by_berkeley}\OperatorTok{==}\DecValTok{0}\NormalTok{) }\OperatorTok\StringTok{ }\KeywordTok{select}\NormalTok{(visits_by_berkeley,yesvis_berkeley)}
\CommentTok{#> # A tibble: 20,732 x 2}
\CommentTok{#>    visits_by_berkeley yesvis_berkeley}
\CommentTok{#>                 <int> <lgl>          }
\CommentTok{#>  1                  0 FALSE          }
\CommentTok{#>  2                  0 FALSE          }
\CommentTok{#>  3                  0 FALSE          }
\CommentTok{#>  4                  0 FALSE          }
\CommentTok{#>  5                  0 FALSE          }
\CommentTok{#>  6                  0 FALSE          }
\CommentTok{#>  7                  0 FALSE          }
\CommentTok{#>  8                  0 FALSE          }
\CommentTok{#>  9                  0 FALSE          }
\CommentTok{#> 10                  0 FALSE          }
\CommentTok{#> # ... with 20,722 more rows}
\end{Highlighting}
\end{Shaded}

\end{frame}

\end{document}
