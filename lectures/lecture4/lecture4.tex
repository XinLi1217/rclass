\documentclass[8pt,ignorenonframetext,dvipsnames]{beamer}
\setbeamertemplate{caption}[numbered]
\setbeamertemplate{caption label separator}{: }
\setbeamercolor{caption name}{fg=normal text.fg}
\beamertemplatenavigationsymbolsempty
\usepackage{lmodern}
\usepackage{amssymb,amsmath}
\usepackage{ifxetex,ifluatex}
\usepackage{fixltx2e} % provides \textsubscript
\ifnum 0\ifxetex 1\fi\ifluatex 1\fi=0 % if pdftex
  \usepackage[T1]{fontenc}
  \usepackage[utf8]{inputenc}
\else % if luatex or xelatex
  \ifxetex
    \usepackage{mathspec}
  \else
    \usepackage{fontspec}
  \fi
  \defaultfontfeatures{Ligatures=TeX,Scale=MatchLowercase}
\fi
% use upquote if available, for straight quotes in verbatim environments
\IfFileExists{upquote.sty}{\usepackage{upquote}}{}
% use microtype if available
\IfFileExists{microtype.sty}{%
\usepackage{microtype}
\UseMicrotypeSet[protrusion]{basicmath} % disable protrusion for tt fonts
}{}
\newif\ifbibliography
\hypersetup{
            pdftitle={Managing and Manipulating Data Using R},
            pdfauthor={Ozan Jaquette},
            pdfborder={0 0 0},
            breaklinks=true}
\urlstyle{same}  % don't use monospace font for urls
\usepackage{color}
\usepackage{fancyvrb}
\newcommand{\VerbBar}{|}
\newcommand{\VERB}{\Verb[commandchars=\\\{\}]}
\DefineVerbatimEnvironment{Highlighting}{Verbatim}{commandchars=\\\{\}}
% Add ',fontsize=\small' for more characters per line
\usepackage{framed}
\definecolor{shadecolor}{RGB}{248,248,248}
\newenvironment{Shaded}{\begin{snugshade}}{\end{snugshade}}
\newcommand{\KeywordTok}[1]{\textcolor[rgb]{0.13,0.29,0.53}{\textbf{#1}}}
\newcommand{\DataTypeTok}[1]{\textcolor[rgb]{0.13,0.29,0.53}{#1}}
\newcommand{\DecValTok}[1]{\textcolor[rgb]{0.00,0.00,0.81}{#1}}
\newcommand{\BaseNTok}[1]{\textcolor[rgb]{0.00,0.00,0.81}{#1}}
\newcommand{\FloatTok}[1]{\textcolor[rgb]{0.00,0.00,0.81}{#1}}
\newcommand{\ConstantTok}[1]{\textcolor[rgb]{0.00,0.00,0.00}{#1}}
\newcommand{\CharTok}[1]{\textcolor[rgb]{0.31,0.60,0.02}{#1}}
\newcommand{\SpecialCharTok}[1]{\textcolor[rgb]{0.00,0.00,0.00}{#1}}
\newcommand{\StringTok}[1]{\textcolor[rgb]{0.31,0.60,0.02}{#1}}
\newcommand{\VerbatimStringTok}[1]{\textcolor[rgb]{0.31,0.60,0.02}{#1}}
\newcommand{\SpecialStringTok}[1]{\textcolor[rgb]{0.31,0.60,0.02}{#1}}
\newcommand{\ImportTok}[1]{#1}
\newcommand{\CommentTok}[1]{\textcolor[rgb]{0.56,0.35,0.01}{\textit{#1}}}
\newcommand{\DocumentationTok}[1]{\textcolor[rgb]{0.56,0.35,0.01}{\textbf{\textit{#1}}}}
\newcommand{\AnnotationTok}[1]{\textcolor[rgb]{0.56,0.35,0.01}{\textbf{\textit{#1}}}}
\newcommand{\CommentVarTok}[1]{\textcolor[rgb]{0.56,0.35,0.01}{\textbf{\textit{#1}}}}
\newcommand{\OtherTok}[1]{\textcolor[rgb]{0.56,0.35,0.01}{#1}}
\newcommand{\FunctionTok}[1]{\textcolor[rgb]{0.00,0.00,0.00}{#1}}
\newcommand{\VariableTok}[1]{\textcolor[rgb]{0.00,0.00,0.00}{#1}}
\newcommand{\ControlFlowTok}[1]{\textcolor[rgb]{0.13,0.29,0.53}{\textbf{#1}}}
\newcommand{\OperatorTok}[1]{\textcolor[rgb]{0.81,0.36,0.00}{\textbf{#1}}}
\newcommand{\BuiltInTok}[1]{#1}
\newcommand{\ExtensionTok}[1]{#1}
\newcommand{\PreprocessorTok}[1]{\textcolor[rgb]{0.56,0.35,0.01}{\textit{#1}}}
\newcommand{\AttributeTok}[1]{\textcolor[rgb]{0.77,0.63,0.00}{#1}}
\newcommand{\RegionMarkerTok}[1]{#1}
\newcommand{\InformationTok}[1]{\textcolor[rgb]{0.56,0.35,0.01}{\textbf{\textit{#1}}}}
\newcommand{\WarningTok}[1]{\textcolor[rgb]{0.56,0.35,0.01}{\textbf{\textit{#1}}}}
\newcommand{\AlertTok}[1]{\textcolor[rgb]{0.94,0.16,0.16}{#1}}
\newcommand{\ErrorTok}[1]{\textcolor[rgb]{0.64,0.00,0.00}{\textbf{#1}}}
\newcommand{\NormalTok}[1]{#1}
\usepackage{longtable,booktabs}
\usepackage{caption}
% These lines are needed to make table captions work with longtable:
\makeatletter
\def\fnum@table{\tablename~\thetable}
\makeatother

% Prevent slide breaks in the middle of a paragraph:
\widowpenalties 1 10000
\raggedbottom

\AtBeginPart{
  \let\insertpartnumber\relax
  \let\partname\relax
  \frame{\partpage}
}
\AtBeginSection{
  \ifbibliography
  \else
    \let\insertsectionnumber\relax
    \let\sectionname\relax
    \frame{\sectionpage}
  \fi
}
\AtBeginSubsection{
  \let\insertsubsectionnumber\relax
  \let\subsectionname\relax
  \frame{\subsectionpage}
}

\setlength{\parindent}{0pt}
\setlength{\parskip}{6pt plus 2pt minus 1pt}
\setlength{\emergencystretch}{3em}  % prevent overfull lines
\providecommand{\tightlist}{%
  \setlength{\itemsep}{0pt}\setlength{\parskip}{0pt}}
\setcounter{secnumdepth}{0}

%packages
\usepackage{graphicx}
\usepackage{rotating}
\usepackage{hyperref}

\usepackage{tikz} % used for text highlighting, amongst others
%title slide stuff
%\institute{Department of Education}
%\title{Managing and Manipulating Data Using R}

%
\setbeamertemplate{navigation symbols}{} % get rid of navigation icons:

%\setbeamertemplate{frametitle}{\thesection \hspace{0.2cm} \insertframetitle}
\setbeamertemplate{section in toc}[sections numbered]
\setbeamertemplate{subsection in toc}[subsections numbered]

%define colors
%\definecolor{uva_orange}{RGB}{216,141,42} % UVa orange (Rotunda orange)
\definecolor{mygray}{rgb}{0.95, 0.95, 0.95} % for highlighted text
	% grey is equal parts red, green, blue. higher values >> lighter grey
	%\definecolor{lightgraybo}{rgb}{0.83, 0.83, 0.83}

% new commands

%highlight text with very light grey
\newcommand*{\hlg}[1]{%
	\tikz[baseline=(X.base)] \node[rectangle, fill=mygray] (X) {#1};%
}
%, inner sep=0.3mm
%highlight text with very light grey and use font associated with code
\newcommand*{\hlgc}[1]{\texttt{\hlg{#1}}}

% Font
\usepackage[defaultfam,light,tabular,lining]{montserrat}
\usepackage[T1]{fontenc}
\renewcommand*\oldstylenums[1]{{\fontfamily{Montserrat-TOsF}\selectfont #1}}

% Change color of boldface text to darkgray
\renewcommand{\textbf}[1]{{\color{darkgray}\bfseries\fontfamily{Montserrat-TOsF}#1}}

% Bullet points
\setbeamertemplate{itemize item}{\color{BlueViolet}$\circ$}
\setbeamertemplate{itemize subitem}{\color{BrickRed}$\triangleright$}
\setbeamertemplate{itemize subsubitem}{$-$}

% Reduce space before lists
\addtobeamertemplate{itemize/enumerate body begin}{}{\vspace*{-8pt}}

\title{Managing and Manipulating Data Using R}
\subtitle{Lecture 4}
\author{Ozan Jaquette}
\date{}

\begin{document}
\frame{\titlepage}

\begin{frame}
\tableofcontents[hideallsubsections]
\end{frame}

\begin{frame}

\end{frame}

\section{Introduction}\label{introduction}

\begin{frame}[fragile]{Libraries we will use today}

\begin{Shaded}
\begin{Highlighting}[]
\KeywordTok{library}\NormalTok{(tidyverse)}
\CommentTok{#> -- Attaching packages ------------------------------------------------------- tidyverse 1.2.1 --}
\CommentTok{#> v ggplot2 3.0.0     v purrr   0.2.5}
\CommentTok{#> v tibble  1.4.2     v dplyr   0.7.6}
\CommentTok{#> v tidyr   0.8.1     v stringr 1.3.1}
\CommentTok{#> v readr   1.1.1     v forcats 0.3.0}
\CommentTok{#> -- Conflicts ---------------------------------------------------------- tidyverse_conflicts() --}
\CommentTok{#> x dplyr::filter() masks stats::filter()}
\CommentTok{#> x dplyr::lag()    masks stats::lag()}
\end{Highlighting}
\end{Shaded}

\end{frame}

\begin{frame}[fragile]{Data we will use today}

Data on off-campus recruiting events by public universities

\begin{itemize}
\tightlist
\item
  Object \hlgc{df\_event}

  \begin{itemize}
  \tightlist
  \item
    One observation per university, recruiting event
  \end{itemize}
\item
  Object \hlgc{df\_event}

  \begin{itemize}
  \tightlist
  \item
    One observation per high school (visited and non-visited)
  \end{itemize}
\end{itemize}

\begin{Shaded}
\begin{Highlighting}[]
\KeywordTok{rm}\NormalTok{(}\DataTypeTok{list =} \KeywordTok{ls}\NormalTok{()) }\CommentTok{# remove all objects}

\CommentTok{#load dataset with one obs per recruiting event}
\KeywordTok{load}\NormalTok{(}\StringTok{"../../data/recruiting/recruit_event_somevars.Rdata"}\NormalTok{)}

\CommentTok{#load dataset with one obs per high school}
\KeywordTok{load}\NormalTok{(}\StringTok{"../../data/recruiting/recruit_school_somevars.Rdata"}\NormalTok{)}

\KeywordTok{load}\NormalTok{(}\StringTok{"../../data/prospect_list/western_washington_college_board_list.RData"}\NormalTok{)}
\end{Highlighting}
\end{Shaded}

\end{frame}

\begin{frame}{Processing across observations, introduction}

Creation of analysis datasets often requires calculations across obs

Examples:

\begin{itemize}
\tightlist
\item
  You have a dataset with one observation per student-term and want to
  create a variable of credits attempted per term
\item
  You have a dataset with one observation per student-term and want to
  create a variable of GPA for the semester or cumulative GPA for all
  semesters
\item
  Number of off-campus recruiting events university makes to each state
\item
  Average household income at visited versus non-visited high schools
\end{itemize}

\end{frame}

\begin{frame}[fragile]{Processing across variables vs.~processing across
observations}

Visits by UC Berkely to public high schools

\begin{verbatim}
#> # A tibble: 5 x 6
#>   school_id    state tot_stu_pub fr_lunch pct_fr_lunch med_inc
#>   <chr>        <chr>       <dbl>    <dbl>        <dbl>   <dbl>
#> 1 340882002126 NJ           1846       29       0.0157 178732 
#> 2 340147000250 NJ           1044       50       0.0479  62288 
#> 3 340561003796 NJ           1505      298       0.198  100684.
#> 4 340165005124 NJ           1900       43       0.0226 160476.
#> 5 341341003182 NJ           1519      130       0.0856 144346
\end{verbatim}

\begin{itemize}
\tightlist
\item
  So far, we have focused on ``processing across variables''

  \begin{itemize}
  \tightlist
  \item
    Performing calculations across columns (i.e., vars), typically
    within a row (i.e., observation)
  \item
    Example: percent free-reduced lunch
  \end{itemize}
\item
  Processing across obs (focus of today's lecture)

  \begin{itemize}
  \tightlist
  \item
    Performing calculations across rows (i.e., obs), often within a
    column (i.e., variable)
  \item
    Example: Average household income of visited high schools, by state
  \end{itemize}
\end{itemize}

\end{frame}

\section{\texorpdfstring{\texttt{group\_by()} and
\texttt{summarise()}}{group\_by() and summarise()}}\label{group_by-and-summarise}

\begin{frame}{group\_by()}

\hlgc{group\_by()} converts a data frame object into groups. After
grouping, functions performed on data frame are performed ``by group''

\begin{itemize}
\tightlist
\item
  part of \textbf{dplyr} package within \textbf{tidyverse}; not part of
  \textbf{Base R}
\item
  works best with pipes \hlgc{\%>\%} and \hlgc{summarise()} function
  {[}described below{]}
\end{itemize}

Basic syntax:

\begin{itemize}
\tightlist
\item
  \hlgc{group\_by(object, vars to group by separated by commas)}
\end{itemize}

\medskip Typically, ``group\_by'' variables are character, factor, or
integer variables

\medskip Possible ``group by'' variables in \hlgc{df\_event} data

\begin{itemize}
\tightlist
\item
  university
\item
  event type (e.g., public HS, private HS, hotel)
\item
  state
\end{itemize}

\end{frame}

\begin{frame}[fragile]{\texttt{group\_by()}}

Group \hlgc{df\_event} data by university, event type, and event state

\begin{itemize}
\tightlist
\item
  group\_by doesn't do much by itself; just prints data
\end{itemize}

\begin{Shaded}
\begin{Highlighting}[]
\KeywordTok{group_by}\NormalTok{(df_event, univ_id, event_type, event_state)}

\NormalTok{df_event }\OperatorTok\StringTok{ }\KeywordTok{group_by}\NormalTok{(univ_id, event_type, event_state) }\CommentTok{# using pipes}
\end{Highlighting}
\end{Shaded}

Grouping is not retained unless you \textbf{assign} it

\begin{Shaded}
\begin{Highlighting}[]
\KeywordTok{class}\NormalTok{(df_event)}
\CommentTok{#> [1] "tbl_df"     "tbl"        "data.frame"}
\NormalTok{df_event_grp <-}\StringTok{ }\NormalTok{df_event }\OperatorTok\StringTok{ }\KeywordTok{group_by}\NormalTok{(univ_id, event_type, event_state) }\CommentTok{# using pipes}
\KeywordTok{class}\NormalTok{(df_event_grp)}
\CommentTok{#> [1] "grouped_df" "tbl_df"     "tbl"        "data.frame"}
\end{Highlighting}
\end{Shaded}

Use \hlgc{ungroup(object)} to un-group grouped data

\begin{Shaded}
\begin{Highlighting}[]
\NormalTok{df_event_grp <-}\StringTok{ }\KeywordTok{ungroup}\NormalTok{(df_event_grp)}
\KeywordTok{class}\NormalTok{(df_event_grp)}
\CommentTok{#> [1] "tbl_df"     "tbl"        "data.frame"}
\KeywordTok{rm}\NormalTok{(df_event_grp)}
\end{Highlighting}
\end{Shaded}

\end{frame}

\begin{frame}[fragile]{\texttt{summarise()}}

\hlgc{summarise()} function performs calculations across rows of a data
frame and then collapses the data frame to a single row

Basic syntax {[}see documentation{]}:

\begin{itemize}
\tightlist
\item
  \hlgc{summarise(object, summarise functions separated by commas)}
\item
  summarise functions include: \texttt{n()}, \texttt{mean()},
  \texttt{first()}, etc.
\end{itemize}

Simple example (output omitted)

\begin{Shaded}
\begin{Highlighting}[]
\KeywordTok{summarise}\NormalTok{(df_event, }\DataTypeTok{num_events=}\KeywordTok{n}\NormalTok{())}
\NormalTok{df_event }\OperatorTok\StringTok{ }\KeywordTok{summarise}\NormalTok{(}\DataTypeTok{num_events=}\KeywordTok{n}\NormalTok{()) }\CommentTok{# using pipes}
\end{Highlighting}
\end{Shaded}

Object created by summarise() not retained unless you \textbf{assign} it

\begin{Shaded}
\begin{Highlighting}[]
\NormalTok{event_temp <-}\StringTok{ }\NormalTok{df_event }\OperatorTok\StringTok{ }\KeywordTok{summarise}\NormalTok{(}\DataTypeTok{num_events=}\KeywordTok{n}\NormalTok{(), }
  \DataTypeTok{mean_inc=}\KeywordTok{mean}\NormalTok{(med_inc, }\DataTypeTok{na.rm =} \OtherTok{TRUE}\NormalTok{))}

\NormalTok{event_temp}
\CommentTok{#> # A tibble: 1 x 2}
\CommentTok{#>   num_events mean_inc}
\CommentTok{#>        <int>    <dbl>}
\CommentTok{#> 1      17976   88774.}
\KeywordTok{rm}\NormalTok{(event_temp)}
\end{Highlighting}
\end{Shaded}

I'll explain \texttt{na.rm\ =\ TRUE} later

\end{frame}

\begin{frame}[fragile]{Combining \texttt{summarise()} and
\texttt{group\_by}}

\texttt{summarise()} on ungrouped vs.~grouped data:

\begin{itemize}
\tightlist
\item
  By itself, \hlgc{summarise()} performs calculations across all rows of
  data frame then collapses the data frame to a single row
\item
  When data frame is grouped, \hlgc{summarise()} performs calculations
  across rows within a group and then collapses to a single row for each
  group
\end{itemize}

Number of recruiting events for each university

\begin{Shaded}
\begin{Highlighting}[]
\NormalTok{df_event }\OperatorTok\StringTok{ }\KeywordTok{group_by}\NormalTok{(instnm) }\OperatorTok\StringTok{ }\KeywordTok{summarise}\NormalTok{(}\DataTypeTok{num_events=}\KeywordTok{n}\NormalTok{())}
\end{Highlighting}
\end{Shaded}

Number of recruiting events by event\_type for each university

\begin{Shaded}
\begin{Highlighting}[]
\NormalTok{df_event }\OperatorTok\StringTok{ }\KeywordTok{group_by}\NormalTok{(instnm, event_type) }\OperatorTok\StringTok{ }\KeywordTok{summarise}\NormalTok{(}\DataTypeTok{num_events=}\KeywordTok{n}\NormalTok{())}
\end{Highlighting}
\end{Shaded}

Number of events and avg. pct White by event\_type for each university

\begin{Shaded}
\begin{Highlighting}[]
\NormalTok{df_event }\OperatorTok\StringTok{ }\KeywordTok{group_by}\NormalTok{(instnm, event_type) }\OperatorTok\StringTok{ }
\StringTok{  }\KeywordTok{summarise}\NormalTok{(}\DataTypeTok{num_events=}\KeywordTok{n}\NormalTok{(),}
    \DataTypeTok{mean_pct_white=}\KeywordTok{mean}\NormalTok{(pct_white_zip, }\DataTypeTok{na.rm =} \OtherTok{TRUE}\NormalTok{)}
\NormalTok{  )}
\end{Highlighting}
\end{Shaded}

\end{frame}

\begin{frame}[fragile]{Combining \texttt{summarise()} and
\texttt{group\_by}}

Recruiting events by UC Berkeley

\begin{Shaded}
\begin{Highlighting}[]
\NormalTok{df_event }\OperatorTok\StringTok{ }\KeywordTok{filter}\NormalTok{(univ_id }\OperatorTok{==}\StringTok{ }\DecValTok{110635}\NormalTok{) }\OperatorTok\StringTok{ }
\StringTok{  }\KeywordTok{group_by}\NormalTok{(event_type) }\OperatorTok\StringTok{ }\KeywordTok{summarise}\NormalTok{(}\DataTypeTok{num_events=}\KeywordTok{n}\NormalTok{())}
\end{Highlighting}
\end{Shaded}

Let's create a dataset of recruiting events at UC Berkeley

The 0/1 variable \texttt{event\_inst} equals 1 if event is in same state
as the university

\begin{Shaded}
\begin{Highlighting}[]
\CommentTok{#event_berk %>% group_by(event_type, event_inst) %>% select(pid, event_date, event_type, event_state, event_inst)}
\NormalTok{event_berk }\OperatorTok\StringTok{ }\KeywordTok{arrange}\NormalTok{(event_date) }\OperatorTok\StringTok{ }\KeywordTok{select}\NormalTok{(pid, event_date, event_type, event_state, event_inst) }\OperatorTok\StringTok{ }\KeywordTok{slice}\NormalTok{(}\DecValTok{1}\OperatorTok{:}\DecValTok{8}\NormalTok{)}
\CommentTok{#> # A tibble: 8 x 5}
\CommentTok{#>     pid event_date event_type  event_state event_inst}
\CommentTok{#>   <int> <date>     <fct>       <chr>       <chr>     }
\CommentTok{#> 1 13100 2017-04-11 other       HI          Out-State }
\CommentTok{#> 2 13089 2017-04-14 public hs   GA          Out-State }
\CommentTok{#> 3 13088 2017-04-23 private hs  CT          Out-State }
\CommentTok{#> 4 13086 2017-04-23 other       CA          In-State  }
\CommentTok{#> 5 13091 2017-04-24 private hs  NY          Out-State }
\CommentTok{#> 6 13087 2017-04-24 public hs   CA          In-State  }
\CommentTok{#> 7 13092 2017-04-25 other       NY          Out-State }
\CommentTok{#> 8 13099 2017-04-25 2yr college CA          In-State}
\end{Highlighting}
\end{Shaded}

\end{frame}

\begin{frame}[fragile]{\texttt{summarise()}: Counts}

The count function \texttt{n()} takes no arguments and returns the size
of the current group

\begin{Shaded}
\begin{Highlighting}[]
\NormalTok{event_berk }\OperatorTok\StringTok{ }\KeywordTok{group_by}\NormalTok{(event_type, event_inst) }\OperatorTok\StringTok{ }
\StringTok{  }\KeywordTok{summarise}\NormalTok{(}\DataTypeTok{num_events=}\KeywordTok{n}\NormalTok{())}
\end{Highlighting}
\end{Shaded}

Object not retained unless we \textbf{assign}

\begin{Shaded}
\begin{Highlighting}[]
\NormalTok{berk_temp <-}\StringTok{ }\NormalTok{event_berk }\OperatorTok\StringTok{ }\KeywordTok{group_by}\NormalTok{(event_type, event_inst) }\OperatorTok\StringTok{ }
\StringTok{  }\KeywordTok{summarise}\NormalTok{(}\DataTypeTok{num_events=}\KeywordTok{n}\NormalTok{())}
\NormalTok{berk_temp}
\KeywordTok{typeof}\NormalTok{(berk_temp)}
\KeywordTok{str}\NormalTok{(berk_temp)}
\end{Highlighting}
\end{Shaded}

Because counts are so important, \texttt{dplyr} package includes
separate \texttt{count()} function that can be called outside
\texttt{summarise()} function

\begin{Shaded}
\begin{Highlighting}[]
\NormalTok{event_berk }\OperatorTok\StringTok{ }\KeywordTok{group_by}\NormalTok{(event_type, event_inst) }\OperatorTok\StringTok{ }\KeywordTok{count}\NormalTok{()}
\NormalTok{event_berk }\OperatorTok\StringTok{ }\KeywordTok{group_by}\NormalTok{(event_type) }\OperatorTok\StringTok{ }\KeywordTok{count}\NormalTok{(event_inst) }\CommentTok{# same}

\NormalTok{berk_temp2 <-}\StringTok{ }\NormalTok{event_berk }\OperatorTok\StringTok{ }\KeywordTok{group_by}\NormalTok{(event_type, event_inst) }\OperatorTok\StringTok{ }\KeywordTok{count}\NormalTok{()}
\NormalTok{berk_temp }\OperatorTok{==}\StringTok{ }\NormalTok{berk_temp2}
\KeywordTok{rm}\NormalTok{(berk_temp,berk_temp2)}
\end{Highlighting}
\end{Shaded}

\end{frame}

\begin{frame}[fragile]{\texttt{summarise()}: count with logical vectors
and \texttt{sum()}}

Logical vectors have values \texttt{TRUE} and \texttt{FALSE}.

\begin{itemize}
\tightlist
\item
  When used with numeric functions, \texttt{TRUE} converted to 1 and
  \texttt{FALSE} to 0.
\end{itemize}

\texttt{sum()} is a numeric function that returns the sum of values

\begin{Shaded}
\begin{Highlighting}[]
\KeywordTok{sum}\NormalTok{(}\KeywordTok{c}\NormalTok{(}\DecValTok{5}\NormalTok{,}\DecValTok{10}\NormalTok{))}
\CommentTok{#> [1] 15}
\KeywordTok{sum}\NormalTok{(}\KeywordTok{c}\NormalTok{(}\OtherTok{TRUE}\NormalTok{,}\OtherTok{TRUE}\NormalTok{,}\OtherTok{FALSE}\NormalTok{,}\OtherTok{FALSE}\NormalTok{))}
\CommentTok{#> [1] 2}
\end{Highlighting}
\end{Shaded}

\texttt{is.na()} returns \texttt{TRUE} if value is \texttt{NA} and
otherwise returns \texttt{FALSE}

\begin{Shaded}
\begin{Highlighting}[]
\KeywordTok{is.na}\NormalTok{(}\KeywordTok{c}\NormalTok{(}\DecValTok{5}\NormalTok{,}\OtherTok{NA}\NormalTok{,}\DecValTok{4}\NormalTok{,}\OtherTok{NA}\NormalTok{))}
\CommentTok{#> [1] FALSE  TRUE FALSE  TRUE}
\end{Highlighting}
\end{Shaded}

Application: How many missing/non-missing obs in variable
{[}\textbf{very important}{]}

\begin{Shaded}
\begin{Highlighting}[]
\NormalTok{event_berk }\OperatorTok\StringTok{ }\KeywordTok{group_by}\NormalTok{(event_type) }\OperatorTok\StringTok{ }
\StringTok{  }\KeywordTok{summarise}\NormalTok{(}
    \DataTypeTok{n_events =} \KeywordTok{n}\NormalTok{(),}
    \DataTypeTok{n_miss_inc =} \KeywordTok{sum}\NormalTok{(}\KeywordTok{is.na}\NormalTok{(med_inc)),}
    \DataTypeTok{n_nonmiss_inc =} \KeywordTok{sum}\NormalTok{(}\OperatorTok{!}\KeywordTok{is.na}\NormalTok{(med_inc)),}
    \DataTypeTok{n_nonmiss_fr_lunch =} \KeywordTok{sum}\NormalTok{(}\OperatorTok{!}\KeywordTok{is.na}\NormalTok{(fr_lunch))}
\NormalTok{  )}
\end{Highlighting}
\end{Shaded}

\end{frame}

\begin{frame}[fragile]{\texttt{summarise()}: means}

The \texttt{mean()} function within \texttt{summarise()} calculates
means, separately for each group

\begin{Shaded}
\begin{Highlighting}[]
\NormalTok{event_berk }\OperatorTok\StringTok{ }\KeywordTok{group_by}\NormalTok{(event_inst, event_type) }\OperatorTok\StringTok{ }\KeywordTok{summarise}\NormalTok{(}
  \DataTypeTok{n_events=}\KeywordTok{n}\NormalTok{(),}
  \DataTypeTok{mean_inc=}\KeywordTok{mean}\NormalTok{(med_inc, }\DataTypeTok{na.rm =} \OtherTok{TRUE}\NormalTok{),}
  \DataTypeTok{mean_pct_white=}\KeywordTok{mean}\NormalTok{(pct_white_zip, }\DataTypeTok{na.rm =} \OtherTok{TRUE}\NormalTok{)) }\OperatorTok\StringTok{ }\KeywordTok{head}\NormalTok{(}\DecValTok{5}\NormalTok{)}
\CommentTok{#> # A tibble: 5 x 5}
\CommentTok{#> # Groups:   event_inst [1]}
\CommentTok{#>   event_inst event_type  n_events mean_inc mean_pct_white}
\CommentTok{#>   <chr>      <fct>          <int>    <dbl>          <dbl>}
\CommentTok{#> 1 In-State   public hs        260   87146.           39.8}
\CommentTok{#> 2 In-State   private hs        36   94133.           48.0}
\CommentTok{#> 3 In-State   2yr college      107   79144.           40.5}
\CommentTok{#> 4 In-State   4yr college       12  148587.           55.1}
\CommentTok{#> 5 In-State   other             50   73218.           35.9}
\end{Highlighting}
\end{Shaded}

I'll talk about \texttt{na.rm\ =\ TRUE} on next slide

\end{frame}

\begin{frame}[fragile]{\texttt{summarise()}: means and \texttt{NA}
values}

The default behavior of ``aggregation functions'' (e.g.,
\texttt{summarise()}) is if the \textbf{input} has any missing value
(\texttt{NA}) than the output will be missing.

\texttt{na.rm} (in words ``remove \texttt{NA}'') is an option available
in many functions.

\begin{itemize}
\tightlist
\item
  \texttt{na.rm\ =\ FALSE} {[}the default for \texttt{mean()}{]}

  \begin{itemize}
  \tightlist
  \item
    Do not remove missing values from input before calculating
  \item
    Therefore, missing values in input will cause output to be missing
  \end{itemize}
\item
  \texttt{na.rm\ =\ TRUE}

  \begin{itemize}
  \tightlist
  \item
    Remove missing values from input before calculating
  \item
    Therefore, missing values in input will not cause output to be
    missing
  \end{itemize}
\end{itemize}

\begin{Shaded}
\begin{Highlighting}[]
\CommentTok{#na.rm = FALSE; the default setting}
\NormalTok{event_berk }\OperatorTok\StringTok{ }\KeywordTok{group_by}\NormalTok{(event_inst, event_type) }\OperatorTok\StringTok{ }\KeywordTok{summarise}\NormalTok{(}
  \DataTypeTok{n_events=}\KeywordTok{n}\NormalTok{(),}
  \DataTypeTok{n_miss_inc =} \KeywordTok{sum}\NormalTok{(}\KeywordTok{is.na}\NormalTok{(med_inc)),}
  \DataTypeTok{mean_inc=}\KeywordTok{mean}\NormalTok{(med_inc, }\DataTypeTok{na.rm =} \OtherTok{FALSE}\NormalTok{),}
  \DataTypeTok{n_miss_frlunch =} \KeywordTok{sum}\NormalTok{(}\KeywordTok{is.na}\NormalTok{(fr_lunch)),}
  \DataTypeTok{mean_fr_lunch=}\KeywordTok{mean}\NormalTok{(fr_lunch, }\DataTypeTok{na.rm =} \OtherTok{FALSE}\NormalTok{))}
\CommentTok{#na.rm = TRUE}
\NormalTok{event_berk }\OperatorTok\StringTok{ }\KeywordTok{group_by}\NormalTok{(event_inst, event_type) }\OperatorTok\StringTok{ }\KeywordTok{summarise}\NormalTok{(}
  \DataTypeTok{n_events=}\KeywordTok{n}\NormalTok{(),}
  \DataTypeTok{n_miss_inc =} \KeywordTok{sum}\NormalTok{(}\KeywordTok{is.na}\NormalTok{(med_inc)),}
  \DataTypeTok{mean_inc=}\KeywordTok{mean}\NormalTok{(med_inc, }\DataTypeTok{na.rm =} \OtherTok{TRUE}\NormalTok{),}
  \DataTypeTok{n_miss_frlunch =} \KeywordTok{sum}\NormalTok{(}\KeywordTok{is.na}\NormalTok{(fr_lunch)),}
  \DataTypeTok{mean_fr_lunch=}\KeywordTok{mean}\NormalTok{(fr_lunch, }\DataTypeTok{na.rm =} \OtherTok{TRUE}\NormalTok{))}
\end{Highlighting}
\end{Shaded}

\end{frame}

\begin{frame}{Student exercise}

\end{frame}

\begin{frame}[fragile]{\texttt{summarise()}: counts with logical
vectors, part II}

Application: count number in a group that satisfy some condition

Task: For each combination of \texttt{event\_type} and
\texttt{event\_inst}, how many visits to communities that are majority
Latinx or Black?

\begin{Shaded}
\begin{Highlighting}[]
\CommentTok{#event_berk %>% select(pct_black_zip, pct_hispanic_zip)}
\NormalTok{event_berk }\OperatorTok\StringTok{ }\KeywordTok{group_by}\NormalTok{ (event_inst, event_type) }\OperatorTok\StringTok{ }\KeywordTok{summarise}\NormalTok{(}
  \DataTypeTok{n_events=}\KeywordTok{n}\NormalTok{(), }\CommentTok{# number of events by group}
  \DataTypeTok{n_nonmiss_latbl =} \KeywordTok{sum}\NormalTok{(}\OperatorTok{!}\KeywordTok{is.na}\NormalTok{(pct_black_zip) }\OperatorTok{&}\StringTok{ }\OperatorTok{!}\KeywordTok{is.na}\NormalTok{(pct_hispanic_zip)), }\CommentTok{# w/ nonmissings values for pct_black and pct latinx}
  \DataTypeTok{n_majority_latbl=} \KeywordTok{sum}\NormalTok{(pct_black_zip}\OperatorTok{+}\StringTok{ }\NormalTok{pct_hispanic_zip}\OperatorTok{>}\DecValTok{50}\NormalTok{, }\DataTypeTok{na.rm =} \OtherTok{TRUE}\NormalTok{)) }\CommentTok{# number of events at majority black/latino communities}
\CommentTok{#> # A tibble: 10 x 5}
\CommentTok{#> # Groups:   event_inst [?]}
\CommentTok{#>    event_inst event_type  n_events n_nonmiss_latbl n_majority_latbl}
\CommentTok{#>    <chr>      <fct>          <int>           <int>            <int>}
\CommentTok{#>  1 In-State   public hs        260             259               88}
\CommentTok{#>  2 In-State   private hs        36              36                7}
\CommentTok{#>  3 In-State   2yr college      107             102               27}
\CommentTok{#>  4 In-State   4yr college       12              10                0}
\CommentTok{#>  5 In-State   other             50              49               17}
\CommentTok{#>  6 Out-State  public hs        184             184               27}
\CommentTok{#>  7 Out-State  private hs       135             133               20}
\CommentTok{#>  8 Out-State  2yr college        1               1                0}
\CommentTok{#>  9 Out-State  4yr college        3               2                0}
\CommentTok{#> 10 Out-State  other             90              87               20}
\end{Highlighting}
\end{Shaded}

\end{frame}

\begin{frame}[fragile]{\texttt{summarise()}: proportions with logical
values}

Application: count proportion of obs in group that satisfy some
condition

\begin{itemize}
\tightlist
\item
  Synatx:
  \texttt{group\_by(vars)\ \%\textgreater{}\%\ summarise(prop\ =\ mean(TRUE/FALSE\ conditon))}
\end{itemize}

Task:

\begin{itemize}
\tightlist
\item
  separately for in-state/out-of-state, what proportion of visits to
  public high schools are to communities with median income greater than
  \$100,000?
\end{itemize}

Steps:

\begin{enumerate}
\def\labelenumi{\arabic{enumi}.}
\tightlist
\item
  Filter public HS visits
\item
  group by in-state vs.~out-of-state
\item
  Create measure
\end{enumerate}

\begin{Shaded}
\begin{Highlighting}[]
\NormalTok{event_berk }\OperatorTok\StringTok{ }\KeywordTok{filter}\NormalTok{(event_type }\OperatorTok{==}\StringTok{ "public hs"}\NormalTok{) }\OperatorTok\StringTok{ }\CommentTok{# filter public hs visits}
\StringTok{  }\KeywordTok{group_by}\NormalTok{ (event_inst) }\OperatorTok\StringTok{ }\CommentTok{# group by in-state vs. out-of-state}
\StringTok{  }\KeywordTok{summarise}\NormalTok{(}
    \DataTypeTok{n_events=}\KeywordTok{n}\NormalTok{(), }\CommentTok{# number of events by group}
    \DataTypeTok{n_nonmiss_inc =} \KeywordTok{sum}\NormalTok{(}\OperatorTok{!}\KeywordTok{is.na}\NormalTok{(med_inc)), }\CommentTok{# w/ nonmissings values median inc,}
    \DataTypeTok{p_incgt100k =} \KeywordTok{mean}\NormalTok{(med_inc}\OperatorTok{>}\DecValTok{100000}\NormalTok{, }\DataTypeTok{na.rm=}\OtherTok{TRUE}\NormalTok{)) }\CommentTok{# proportion visits to $100K+ commmunities }
\CommentTok{#> # A tibble: 2 x 4}
\CommentTok{#>   event_inst n_events n_nonmiss_inc p_incgt100k}
\CommentTok{#>   <chr>         <int>         <int>       <dbl>}
\CommentTok{#> 1 In-State        260           257       0.272}
\CommentTok{#> 2 Out-State       184           184       0.516}
\end{Highlighting}
\end{Shaded}

What if we forgot to put \texttt{na.rm=TRUE}?

\end{frame}

\begin{frame}{Student exercise}

\end{frame}

\begin{frame}[fragile]{\texttt{summarise()}: Other functions}

Common functions to use with \texttt{summarise}:

\begin{longtable}[]{@{}ll@{}}
\toprule
Function & Description\tabularnewline
\midrule
\endhead
\texttt{n} & count\tabularnewline
\texttt{n\_distinct} & count unique values\tabularnewline
\texttt{mean} & mean\tabularnewline
\texttt{median} & median\tabularnewline
\texttt{max} & largest value\tabularnewline
\texttt{min} & smallest value\tabularnewline
\texttt{sd} & standard deviation\tabularnewline
\texttt{sum} & sum of values\tabularnewline
\texttt{first} & first value\tabularnewline
\texttt{last} & last value\tabularnewline
\texttt{nth} & nth value\tabularnewline
\texttt{any} & condition true for at least one value?\tabularnewline
\bottomrule
\end{longtable}

\emph{Note: These functions can also be used on their own or with
\texttt{mutate()}}

\end{frame}

\begin{frame}[fragile]{\texttt{summarise()}: Other functions}

Maximum value in a group

\begin{Shaded}
\begin{Highlighting}[]
\KeywordTok{max}\NormalTok{(}\KeywordTok{c}\NormalTok{(}\DecValTok{10}\NormalTok{,}\DecValTok{50}\NormalTok{,}\DecValTok{8}\NormalTok{))}
\CommentTok{#> [1] 50}
\end{Highlighting}
\end{Shaded}

Task: For each combination of in-state/out-of-state and event type, what
is the maximum value of \texttt{med\_inc}?

\begin{Shaded}
\begin{Highlighting}[]
\NormalTok{event_berk }\OperatorTok\StringTok{ }\KeywordTok{group_by}\NormalTok{(event_type, event_inst) }\OperatorTok\StringTok{ }
\StringTok{  }\KeywordTok{summarise}\NormalTok{(}\DataTypeTok{max_inc =} \KeywordTok{max}\NormalTok{(med_inc))}
\CommentTok{#> # A tibble: 10 x 3}
\CommentTok{#> # Groups:   event_type [?]}
\CommentTok{#>    event_type  event_inst max_inc}
\CommentTok{#>    <fct>       <chr>        <dbl>}
\CommentTok{#>  1 public hs   In-State       NA }
\CommentTok{#>  2 public hs   Out-State  223556.}
\CommentTok{#>  3 private hs  In-State   250001 }
\CommentTok{#>  4 private hs  Out-State      NA }
\CommentTok{#>  5 2yr college In-State       NA }
\CommentTok{#>  6 2yr college Out-State  153070.}
\CommentTok{#>  7 4yr college In-State       NA }
\CommentTok{#>  8 4yr college Out-State      NA }
\CommentTok{#>  9 other       In-State       NA }
\CommentTok{#> 10 other       Out-State      NA}
\end{Highlighting}
\end{Shaded}

What did we do wrong here?

\end{frame}

\begin{frame}[fragile]{\texttt{summarise()}: Other functions}

Isolate first/last/nth observation in a group

\begin{Shaded}
\begin{Highlighting}[]
\NormalTok{x <-}\StringTok{ }\KeywordTok{c}\NormalTok{(}\DecValTok{10}\NormalTok{,}\DecValTok{15}\NormalTok{,}\DecValTok{20}\NormalTok{,}\DecValTok{25}\NormalTok{,}\DecValTok{30}\NormalTok{)}
\KeywordTok{first}\NormalTok{(x)}
\KeywordTok{last}\NormalTok{(x)}
\KeywordTok{nth}\NormalTok{(x,}\DecValTok{1}\NormalTok{)}
\KeywordTok{nth}\NormalTok{(x,}\DecValTok{3}\NormalTok{)}
\KeywordTok{nth}\NormalTok{(x,}\DecValTok{10}\NormalTok{)}
\end{Highlighting}
\end{Shaded}

Task: after sorting \texttt{event\_berk} by {[}\texttt{arrange()}{]} by
\texttt{event\_type} and \texttt{event\_datetime\_start}, what is the
value of \texttt{event\_date} for:

\begin{itemize}
\tightlist
\item
  first event for each event type?
\item
  the last eventfor each event type?
\item
  the 10th event for each event type?
\end{itemize}

\begin{Shaded}
\begin{Highlighting}[]
\NormalTok{event_berk }\OperatorTok\StringTok{ }\KeywordTok{arrange}\NormalTok{(event_type, event_datetime_start) }\OperatorTok
\StringTok{  }\KeywordTok{group_by}\NormalTok{(event_type) }\OperatorTok
\StringTok{  }\KeywordTok{summarise}\NormalTok{(}
    \DataTypeTok{n_events =} \KeywordTok{n}\NormalTok{(),}
    \DataTypeTok{date_first=} \KeywordTok{first}\NormalTok{(event_date),}
    \DataTypeTok{date_last=} \KeywordTok{last}\NormalTok{(event_date),}
    \DataTypeTok{date_50th=} \KeywordTok{nth}\NormalTok{(event_date, }\DecValTok{50}\NormalTok{)}
\NormalTok{  )}
\end{Highlighting}
\end{Shaded}

\end{frame}

\begin{frame}{Student exercise}

something that involves whether visits adhered to a certain pattern?
e.g., visited org of type 1 and then org of type 2 in succession?

\end{frame}

\begin{frame}[fragile]{Attach aggregate measures to your data frame}

We can attach aggregate measures to a data frame by using group\_by
without summarise()

Example task:

\begin{itemize}
\tightlist
\item
  Using \texttt{event\_berk} data frame, create (1) a measure of average
  income across all events and (2) a measure of average income for each
  event type
\end{itemize}

Create measure of average income across all events

\begin{Shaded}
\begin{Highlighting}[]
\NormalTok{event_berk_temp <-}\StringTok{ }\NormalTok{event_berk }\OperatorTok\StringTok{ }
\StringTok{  }\KeywordTok{arrange}\NormalTok{(event_date) }\OperatorTok\StringTok{ }\CommentTok{# sort by event_date (optional)}
\StringTok{  }\KeywordTok{select}\NormalTok{(event_date, event_type,med_inc) }\OperatorTok\StringTok{ }\CommentTok{# select vars to be retained (optioanl) }
\StringTok{  }\KeywordTok{mutate}\NormalTok{(}\DataTypeTok{avg_inc =} \KeywordTok{mean}\NormalTok{(med_inc, }\DataTypeTok{na.rm=}\OtherTok{TRUE}\NormalTok{)) }\CommentTok{# create avg. inc measure}

\KeywordTok{dim}\NormalTok{(event_berk_temp)}
\NormalTok{event_berk_temp }\OperatorTok\StringTok{ }\KeywordTok{head}\NormalTok{(}\DecValTok{5}\NormalTok{)}
\end{Highlighting}
\end{Shaded}

Create measure of average income by event type

\begin{Shaded}
\begin{Highlighting}[]
\NormalTok{event_berk_temp <-}\StringTok{ }\NormalTok{event_berk_temp }\OperatorTok\StringTok{ }
\StringTok{  }\KeywordTok{group_by}\NormalTok{(event_type) }\OperatorTok\StringTok{ }\CommentTok{# grouping by event type}
\StringTok{  }\KeywordTok{mutate}\NormalTok{(}\DataTypeTok{avg_inc_type =} \KeywordTok{mean}\NormalTok{(med_inc, }\DataTypeTok{na.rm=}\OtherTok{TRUE}\NormalTok{)) }\CommentTok{# create avg. inc measure}
  
\NormalTok{event_berk_temp }\OperatorTok\StringTok{ }\KeywordTok{head}\NormalTok{(}\DecValTok{5}\NormalTok{)}
\end{Highlighting}
\end{Shaded}

\end{frame}

\begin{frame}[fragile]{Attach aggregate measures to your data frame}

Task: Create a measure that identifies whether \texttt{med\_inc}
associated with the event is higher/lower than average income for all
events of that type

Steps:

\begin{enumerate}
\def\labelenumi{\arabic{enumi}.}
\tightlist
\item
  Create measure of average income for each event type {[}already
  done{]}
\item
  Create measure that compares income to average income for event type
\end{enumerate}

\begin{Shaded}
\begin{Highlighting}[]
\CommentTok{# average income at recruiting events across all universities}
\NormalTok{event_berk_tempv2 <-}\StringTok{ }\NormalTok{event_berk_temp }\OperatorTok\StringTok{ }
\StringTok{  }\KeywordTok{mutate}\NormalTok{(}\DataTypeTok{gt_avg_inc_type =}\NormalTok{ med_inc }\OperatorTok{>}\StringTok{ }\NormalTok{avg_inc_type) }\OperatorTok\StringTok{ }
\StringTok{  }\KeywordTok{select}\NormalTok{(}\OperatorTok{-}\NormalTok{(avg_inc)) }\CommentTok{# drop avg_inc (optional)}
\NormalTok{event_berk_tempv2 }\CommentTok{# note how med_ic = NA are treated}
\end{Highlighting}
\end{Shaded}

create integer indicator rather than logical

\begin{Shaded}
\begin{Highlighting}[]
\NormalTok{event_berk_tempv2 <-}\StringTok{ }\NormalTok{event_berk_tempv2 }\OperatorTok\StringTok{ }
\StringTok{  }\KeywordTok{mutate}\NormalTok{(}\DataTypeTok{gt_avg_inc_type =} \KeywordTok{as.integer}\NormalTok{(med_inc }\OperatorTok{>}\StringTok{ }\NormalTok{avg_inc_type)) }
\NormalTok{event_berk_tempv2  }\OperatorTok\StringTok{ }\KeywordTok{head}\NormalTok{(}\DecValTok{4}\NormalTok{)}
\CommentTok{#> # A tibble: 4 x 5}
\CommentTok{#> # Groups:   event_type [3]}
\CommentTok{#>   event_date event_type med_inc avg_inc_type gt_avg_inc_type}
\CommentTok{#>   <date>     <fct>        <dbl>        <dbl>           <int>}
\CommentTok{#> 1 2017-04-11 other       62022.       70922.               0}
\CommentTok{#> 2 2017-04-14 public hs  125056.       93956.               1}
\CommentTok{#> 3 2017-04-23 private hs  78897        88695.               0}
\CommentTok{#> 4 2017-04-23 other       55127        70922.               0}
\end{Highlighting}
\end{Shaded}

\end{frame}

\begin{frame}[fragile]{Student exercise}

Task: is \texttt{pct\_white\_zip} at a particular event higher or lower
than the average pct\_white\_zip for that \texttt{event\_type}?

\begin{itemize}
\tightlist
\item
  Note: all events attached to a particular zip\_code
\item
  \texttt{pct\_white\_zip}: pct of people in that zip\_code who identify
  as white
\end{itemize}

Steps in task:

\begin{itemize}
\tightlist
\item
  Create measure of average pct white for each event\_type
\item
  Compare whether pct\_white\_zup is higher or lower than this average
\end{itemize}

\end{frame}

\end{document}
