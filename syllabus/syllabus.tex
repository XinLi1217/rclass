\documentclass[11pt,]{article}
\usepackage[margin=1in]{geometry}
\newcommand*{\authorfont}{\fontfamily{phv}\selectfont}
\usepackage[]{mathpazo}
\usepackage{abstract}
\renewcommand{\abstractname}{}    % clear the title
\renewcommand{\absnamepos}{empty} % originally center
\newcommand{\blankline}{\quad\pagebreak[2]}

\providecommand{\tightlist}{%
  \setlength{\itemsep}{0pt}\setlength{\parskip}{0pt}} 
\usepackage{longtable,booktabs}

\usepackage{parskip}
\usepackage{titlesec}
\titlespacing\section{0pt}{12pt plus 4pt minus 2pt}{6pt plus 2pt minus 2pt}
\titlespacing\subsection{0pt}{12pt plus 4pt minus 2pt}{6pt plus 2pt minus 2pt}

\titleformat*{\subsubsection}{\normalsize\itshape}

\usepackage{titling}
\setlength{\droptitle}{-.20cm}

%\setlength{\parindent}{0pt}
%\setlength{\parskip}{6pt plus 2pt minus 1pt}
%\setlength{\emergencystretch}{3em}  % prevent overfull lines 

\usepackage[T1]{fontenc}
\usepackage[utf8]{inputenc}

\usepackage{fancyhdr}
\pagestyle{fancy}
\usepackage{lastpage}
\renewcommand{\headrulewidth}{0.3pt}
\renewcommand{\footrulewidth}{0.0pt} 
\lhead{}
\chead{}
\rhead{\footnotesize EDUC 263: Introduction to Data Management Using R -- Fall 2018}
\lfoot{}
\cfoot{\small \thepage/\pageref*{LastPage}}
\rfoot{}

\fancypagestyle{firststyle}
{
\renewcommand{\headrulewidth}{0pt}%
   \fancyhf{}
   \fancyfoot[C]{\small \thepage/\pageref*{LastPage}}
}

%\def\labelitemi{--}
%\usepackage{enumitem}
%\setitemize[0]{leftmargin=25pt}
%\setenumerate[0]{leftmargin=25pt}




\makeatletter
\@ifpackageloaded{hyperref}{}{%
\ifxetex
  \usepackage[setpagesize=false, % page size defined by xetex
              unicode=false, % unicode breaks when used with xetex
              xetex]{hyperref}
\else
  \usepackage[unicode=true]{hyperref}
\fi
}
\@ifpackageloaded{color}{
    \PassOptionsToPackage{usenames,dvipsnames}{color}
}{%
    \usepackage[usenames,dvipsnames]{color}
}
\makeatother
\hypersetup{breaklinks=true,
            bookmarks=true,
            pdfauthor={ ()},
             pdfkeywords = {},  
            pdftitle={EDUC 263: Introduction to Data Management Using R},
            colorlinks=true,
            citecolor=blue,
            urlcolor=blue,
            linkcolor=magenta,
            pdfborder={0 0 0}}
\urlstyle{same}  % don't use monospace font for urls


\setcounter{secnumdepth}{0}





\usepackage{setspace}

\title{EDUC 263: Introduction to Data Management Using R}
\author{Ozan Jaquette}
\date{Fall 2018}


\begin{document}  

		\maketitle
		
	
		\thispagestyle{firststyle}

%	\thispagestyle{empty}


	\noindent \begin{tabular*}{\textwidth}{ @{\extracolsep{\fill}} lr @{\extracolsep{\fill}}}


Instructor: \texttt{Ozan Jaquette} & Teaching Assistant: \texttt{Patricia Martín} \\
Pronouns: \texttt{he/him/his} & Pronouns: \texttt{she/her/hers} \\
E-mail: \texttt{\href{mailto:ozanj@ucla.edu}{\nolinkurl{ozanj@ucla.edu}}} & E-mail: \texttt{\href{mailto:pmarti@g.ucla.edu}{\nolinkurl{pmarti@g.ucla.edu}}} \\ 
Office Hours: Tues 3-4PM; and by appt  & Office Hours: \texttt{Wed 12-1PM; Thur 1:30-2:30PM} \\
Office: Moore Hall 3038 & Location: \texttt{Moore Hall 3120 (computer lab)} \\
\\ *
Class Room: Moore Hall 2120 & Class Hours: Fridays 12 - 4 pm\\
Class Website: \href{http://ozanj.github.io/rclass/}{\tt ozanj.github.io/rclass/} & Class Discussion forum: \href{http://piazza.com/class/jlo6477nzqo2j0}{\tt piazza.com/class/jlo6477nzqo2j0}\\
	&  \\
	\hline
	\end{tabular*}
	
\vspace{2mm}
	


\section{Course Description}\label{course-description}

This course has two foundational goals: (1) to develop core skills in
``data management,'' which are important regardless of which programming
language you use, and (2) to learn the fundamentals of the R programming
language.

Data management consists of acquiring, investigating, cleaning,
combining, and manipulating data. Most statistics courses teach you how
to analyze data that are ready for analysis. In real research projects,
cleaning the data and creating analysis datasets is often more time
consuming than conducting analyses. This course teaches the fundamental
data management and data manipulation skills necessary for creating
analysis datasets.

The course will be taught in R, a free, open-source programming
language. R has become the most popular language for statistical
analysis, surpassing SPSS, Stata, and SAS. What differentiates R from
these other languages is the thousands of open-source ``libraries''
created by R users. R is one of the most popular languages for ``data
science,'' because R libraries have been created for web-scraping,
mapping, network analysis, etc. By learning R you can be confident that
you know a programming language that can run any modeling technique you
might need and has amazing capabilities for data collection and data
visualization. By learning fundamentals of R in this course, you will be
``one step away'' from web-scraping, network analysis, interactive maps,
quantitative text analysis, or whatever other data science application
you are interested in.

Students will become proficient in data manipulation tasks through
weekly ``problem sets'' that you complete in groups of three. These
problem sets will account for 80\% of your grade for the course. Each
week class will begin with one group will leading a discussion of
challenges they encountered while completing the problem set. The rest
of class time will be devoted to learning new material. The instructor
will provide students with lecture notes, and also data and code used
during lecture. Therefore, student can follow along by running code from
their own computers.

\section{Course Learning Goals}\label{course-learning-goals}

\begin{enumerate}
\def\labelenumi{\arabic{enumi}.}
\tightlist
\item
  Understand fundamental concepts of object oriented programming

  \begin{itemize}
  \tightlist
  \item
    What are the basic object types and how do they apply to statistical
    analysis
  \item
    What are object attributes and how do they apply to statistical
    analysis
  \end{itemize}
\item
  Become familiar with Base R approach to data manipulation and
  Tidyverse approach to data manipulation
\item
  Investigate data patterns

  \begin{itemize}
  \tightlist
  \item
    Sort datasets in ways that generate insights about data structure
  \item
    Select specific observations and specific variables in order to
    identify data structure and to examine whether variables are created
    correctly
  \item
    Create summary statistics of particular variables to diagnose errors
    in data
  \end{itemize}
\item
  Create variables

  \begin{itemize}
  \tightlist
  \item
    Create variables that require calculations across columns
  \item
    Create variables that require processing across rows
  \end{itemize}
\item
  Combine multiple datasets

  \begin{itemize}
  \tightlist
  \item
    Join (merge) datasets
  \item
    Append (stack) datasets
  \end{itemize}
\item
  Manipulate the organizational structure of datasets

  \begin{itemize}
  \tightlist
  \item
    summarize and collapse by group
  \item
    Tidy untidy data
  \end{itemize}
\item
  Automate iterative tasks

  \begin{itemize}
  \tightlist
  \item
    Write your own functions
  \item
    Write loops
  \end{itemize}
\item
  Learn habits of mind and practical strategies for cleaning dirty data
  and avoiding errors when creating analysis variables variables
\end{enumerate}

\section{Prerequisite Requirements}\label{prerequisite-requirements}

\begin{enumerate}
\def\labelenumi{\arabic{enumi}.}
\tightlist
\item
  Students must have taken at least a one-semester introductory
  statistics course.\\
\item
  Students should have some very basic experience using statistical
  programming software (e.g., SPSS, Stata, R, SAS)
\item
  {[}General computer skills{]} Students should be able to download
  files from the internet, rename these files, save them to a folder of
  your choosing, and open this folder.

  \begin{itemize}
  \tightlist
  \item
    During this course we will often be downloading datasets, opening
    .Rmd files and .R scripts, changing directories to the folder where
    we stored the data, and then opening the dataset we just downloaded.
    Therefore, it is important that students feel comfortable doing
    these tasks.
  \end{itemize}
\end{enumerate}

\section{Course Readings}\label{course-readings}

Required text

\begin{itemize}
\tightlist
\item
  Grolemund, G., \& Wickham, H. (2018). \emph{R for data science}.
  Retrieved from \url{http://r4ds.had.co.nz/} {[}FREE!{]}
\end{itemize}

\section{Required Software and
Hardware}\label{required-software-and-hardware}

\subsection{Software {[}FREE!{]}}\label{software-free}

PATRICIA - CAN YOU ADD TEXT HERE; PERHAPS PUT A LINK TO THE TO DO LIST?

Please install the following software on your laptop

\begin{itemize}
\tightlist
\item
  R
\item
  RStudio
\item
  MikTeK
\end{itemize}

\subsection{Hardware}\label{hardware}

\begin{itemize}
\tightlist
\item
  Please bring in laptop with above software installed each week
\end{itemize}

\section{Course Website and
Resources}\label{course-website-and-resources}

PATRICIA - CAN YOU ADD TEXT HERE; ABOUT PIAZZA, ABOUT PIAZZA DISCUSSION
FORUMS; ABOUT THE WEBSITE

\section{Communication with Instructor and Teaching
Assistant}\label{communication-with-instructor-and-teaching-assistant}

Use Piazza discussion forums for all questions related to course
content. All students can then benefit from my response. I will aim to
respond within 24 hours of your post Monday through Friday and 48 hours
on Saturday and Sunday. Email me directly if you have a question
regarding any personal issue

I encourage students to answer questions your classmates post on CCLE
discussion forums. Writing out explanations to student questions will
improve your own knowledge and will benefit your classmates.

\section{Assignments \& Grading}\label{assignments-grading}

Your final grade will be based on the following components:

\begin{itemize}
\tightlist
\item
  Weekly problem sets (80 percent of total grade)
\item
  Your homework group leads a discussion about how you completed the
  problem set (10 percent of total grade)
\item
  Attendance and participation (10 percent of total grade)
\end{itemize}

\subsection{Weekly problem sets}\label{weekly-problem-sets}

Students will complete 10 problem sets. Problem sets are due by 12PM
each Friday (right before the class meeting). Late submissions will not
receive points because we will discuss solutions during class. The
lowest grade will be dropped from the calculation of your final grade.

In general, each problem set will give your practice using the skills
and concepts introduced during the previous lecture. For example, after
the lecture on joining (merging) datasets, the problem set for that week
will require that students complete several different tasks involving
merging data. Additionally, the weekly problem sets will require you to
use data manipulation skills you learned in previous weeks.

Students will work on problem sets in groups of three people (groups
assigned in week 2; same group throughout the semester). However, each
student will submit their own assignment. You are encouraged to share
ideas and get help from your group. However, it is important that you
understand how to do the problem set on your own, rather than copying
the solution developed by group members. If I find compelling evidence
that a student merely copied solutions from a classmate, I will consider
this a violation of academic integrity and that student will receive a
zero for the homework assignment.

A general strategy I recommend for completing the problem sets is as
follows: (1) after lecture, do the reading associated with that lecture;
(2) try doing the problem set on your own; (3) meet with your group to
work through the problem set, with a particular focus on areas group
members find challenging.

\subsection{Group led discussion about problem set (10 percent of total
grade)}\label{group-led-discussion-about-problem-set-10-percent-of-total-grade}

Each week a student group (groups of three) will lead a discussion on
the problem set. Groups have a lot of autonomy in how they want to
approach this discussion (e.g., can have slides, handouts, or neither).
However, my preference is that this is more of an open discussion than a
presentation. In terms of topics to cover, the group can invite class
members to share what challenges/problems they encountered while
completing the homework, how they were able to overcome these problems,
alternative ways to overcome data manipulation challenges, and what
concepts or tasks remain confusing.

For now, we'll allot 40 minutes (at beginning of class) for these
discussions. But this allotted time may increase to 50 minutes or
decrease to 30 minutes.

\subsection{Attendance and Participation (10 percent of total
grade)}\label{attendance-and-participation-10-percent-of-total-grade}

Students are required to attend the weekly class meetings. Each
unexcused absence results in a loss of 20\% from your
attendance/participation grade. Three or more unexcused absences will
result in a failing grade for the course.

An excused absence is a professional opportunity that you discuss with
me beforehand or a medical, or family emergency. Excused absences will
not result in a loss of attendance points. However, you will be
responsible for all material covered in that class and you will be
expected to turn in homework assignments on time.

Students are expected to participate in the weekly class meetings by
being attentive, by asking questions, by answering questions posed by
classmates and by the professor. In addition to participation during
class meetings, students can receive strong participation grades by
asking questions and answering questions on Piazza.

\section{Course Policies}\label{course-policies}

\subsection{Classroom environment}\label{classroom-environment}

We all have a responsibility to ensure that every member of the class
feels valued, safe, and included.

With respect to the course material, learning programming and the
essential skills of data manipulation is hard! This stuff feels
overwhelming to me all the time. So it is important that we all create
an environment where students feel comfortable asking questions and
talking about what they did not understand.

With respect to creatin an inclusive environment, be mindful that what
you say affects other people. So express your thoughts in a way that
doesn't make people feel excluded and does not disparaging
generalizations about a group.

As an instructor, I am responsible for setting an example through my own
conduct. I hope to create an environment where students feel comfortable
voicing concerns about the classroom environment. I will endeavor to
present materials that are respectful of diversity: race, color,
ethnicity, gender, age, disability, religious beliefs, political
preference, sexual orientation, gender identity, citizenship, or
national origin among other personal characteristics. The diversity of
student experiences and perspectives is essential to the deepening of
knowledge in a course. Any suggestions that you have about other ways to
include the value of diversity in this course are welcome. During the
quarter, I will also create forums where students can voice concerns
anonymously.

\subsection{Online
Collaboration/Netiquette}\label{online-collaborationnetiquette}

You will communicate with instructors and peers virtually through a
variety of tools such as discussion forums, email, and web conferencing.
The following guidelines will enable everyone in the course to
participate and collaborate in a productive, safe environment.

\begin{itemize}
\tightlist
\item
  Be professional, courteous, and respectful as you would in a physical
  classroom.
\item
  Online communication lacks the nonverbal cues that provide much of the
  meaning and nuances in face-to-face conversations. Choose your words
  carefully, phrase your sentences clearly, and stay on topic.
\item
  It is expected that students may disagree with the research presented
  or the opinions of their fellow classmates. To disagree is fine but to
  disparage others' views is unacceptable. All comments should be kept
  civil and thoughtful.
\end{itemize}

\subsection{Academic accomodations}\label{academic-accomodations}

Students needing academic accommodations based on a disability should
contact the Center for Accessible Education (CAE) at (310)825-1501 or in
person at Murphy Hall A255. When possible, students should contact the
CAE within the first two weeks of the term as reasonable notice is
needed to coordinate accommodations. For more information visit
www.cae.ucla.edu.

\subsection{Academic Honesty:}\label{academic-honesty}

UCLA is a community of scholars. In this community, all members
including faculty, staff and students alike are responsible for
maintaining standards of academic honesty. As a student and member of
the University community, you are here to get an education and are,
therefore, expected to demonstrate integrity in your academic endeavors.
You are evaluated on your own merits. Cheating, plagiarism,
collaborative work, multiple submissions without the permission of the
professor, or other kinds of academic dishonesty are considered
unacceptable behavior and will result in formal disciplinary proceedings
usually resulting in suspension or dismissal.

\section{Course Schedule and Required
Reading}\label{course-schedule-and-required-reading}

All reading is required reading. I have worked hard to keep reading load
light, focusing only on essentials, because weekly problem sets will be
time consuming.

In the below schedule, I lecture on a topic, and then you do the reading
about that topic and are required to complete a problem set about that
topic. However, if you would prefer to the reading about a topic
\textbf{prior} to me lecturing about that topic, feel free to do so.

\subsection{Lecture 1, 09/28: Course introduction; objects in
R}\label{lecture-1-0928-course-introduction-objects-in-r}

\begin{itemize}
\tightlist
\item
  Reading (after class): Grolemund and Wickham (GW) 1; GW 2; GW 4; GW
  20.1 - 20.3
\end{itemize}

\subsection{Lecture 2, 10/05: Investigating data
patterns}\label{lecture-2-1005-investigating-data-patterns}

\begin{itemize}
\tightlist
\item
  Problem set due (before class): Yes
\item
  Reading (after class): GW 5.1 - 5.4
\end{itemize}

\subsection{Lecture 3, 10/12: Variable creation, attributes, factors,
and
pipes}\label{lecture-3-1012-variable-creation-attributes-factors-and-pipes}

\begin{itemize}
\tightlist
\item
  Problem set due (before class): Yes
\item
  Reading (after class): GW 5.5; GW 15.1 - 15.2; GW 20.6 - 20.7
\end{itemize}

\subsection{Lecture 4, 10/19: Processing across
rows}\label{lecture-4-1019-processing-across-rows}

\begin{itemize}
\tightlist
\item
  Problem set due (before class): Yes
\item
  Reading (after class): GW 5.6 - 5.7
\end{itemize}

\subsection{Lecture 5, 10/26: Survey data and exploratory data analysis
(for data
quality)}\label{lecture-5-1026-survey-data-and-exploratory-data-analysis-for-data-quality}

\begin{itemize}
\tightlist
\item
  Problem set due (before class): Yes
\item
  Reading (after class): GW 10 (note: this chapter is about ``tibbles'',
  not survey data/EDA)
\end{itemize}

\subsection{Lecture 6, 11/02: Tidy data}\label{lecture-6-1102-tidy-data}

\begin{itemize}
\tightlist
\item
  Problem set due (before class): Yes
\item
  Reading (after class): GW 12
\end{itemize}

\subsection{Lecture 7, 11/09: Joining multiple
datasets}\label{lecture-7-1109-joining-multiple-datasets}

\begin{itemize}
\tightlist
\item
  Problem set due (before class): Yes
\item
  Reading (after class): GW 13
\end{itemize}

\subsection{Lecture 8, 11/16: Acquiring
data}\label{lecture-8-1116-acquiring-data}

\begin{itemize}
\tightlist
\item
  Problem set due (before class): Yes
\item
  Reading (after class): GW 11
\end{itemize}

\subsection{Thanksgiving, 11/23: No
class}\label{thanksgiving-1123-no-class}

\subsection{Lecture 9, 11/30: Writing
functions}\label{lecture-9-1130-writing-functions}

\begin{itemize}
\tightlist
\item
  Problem set due (before class): Yes
\item
  Reading (after class): GW 19
\end{itemize}

\subsection{Lecture 10, 12/07: Accessing object elements and
looping}\label{lecture-10-1207-accessing-object-elements-and-looping}

\begin{itemize}
\tightlist
\item
  Problem set due (before class): Yes
\item
  Reading (after class): GW 20.4 - 20.5; 21.1 - 21.3
\end{itemize}

\subsection{Finals Week, 12/14: No
class}\label{finals-week-1214-no-class}

\begin{itemize}
\tightlist
\item
  Problem set due: Yes
\end{itemize}




\end{document}

\makeatletter
\def\@maketitle{%
  \newpage
%  \null
%  \vskip 2em%
%  \begin{center}%
  \let \footnote \thanks
    {\fontsize{18}{20}\selectfont\raggedright  \setlength{\parindent}{0pt} \@title \par}%
}
%\fi
\makeatother