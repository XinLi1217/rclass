\documentclass[10pt,ignorenonframetext,]{beamer}
\setbeamertemplate{caption}[numbered]
\setbeamertemplate{caption label separator}{: }
\setbeamercolor{caption name}{fg=normal text.fg}
\beamertemplatenavigationsymbolsempty
\usepackage{lmodern}
\usepackage{amssymb,amsmath}
\usepackage{ifxetex,ifluatex}
\usepackage{fixltx2e} % provides \textsubscript
\ifnum 0\ifxetex 1\fi\ifluatex 1\fi=0 % if pdftex
  \usepackage[T1]{fontenc}
  \usepackage[utf8]{inputenc}
\else % if luatex or xelatex
  \ifxetex
    \usepackage{mathspec}
  \else
    \usepackage{fontspec}
  \fi
  \defaultfontfeatures{Ligatures=TeX,Scale=MatchLowercase}
\fi
% use upquote if available, for straight quotes in verbatim environments
\IfFileExists{upquote.sty}{\usepackage{upquote}}{}
% use microtype if available
\IfFileExists{microtype.sty}{%
\usepackage{microtype}
\UseMicrotypeSet[protrusion]{basicmath} % disable protrusion for tt fonts
}{}
\newif\ifbibliography
\hypersetup{
            pdftitle={Managing and Manipulating Data Using R},
            pdfauthor={Ozan Jaquette},
            pdfborder={0 0 0},
            breaklinks=true}
\urlstyle{same}  % don't use monospace font for urls
\usepackage{color}
\usepackage{fancyvrb}
\newcommand{\VerbBar}{|}
\newcommand{\VERB}{\Verb[commandchars=\\\{\}]}
\DefineVerbatimEnvironment{Highlighting}{Verbatim}{commandchars=\\\{\}}
% Add ',fontsize=\small' for more characters per line
\usepackage{framed}
\definecolor{shadecolor}{RGB}{248,248,248}
\newenvironment{Shaded}{\begin{snugshade}}{\end{snugshade}}
\newcommand{\KeywordTok}[1]{\textcolor[rgb]{0.13,0.29,0.53}{\textbf{#1}}}
\newcommand{\DataTypeTok}[1]{\textcolor[rgb]{0.13,0.29,0.53}{#1}}
\newcommand{\DecValTok}[1]{\textcolor[rgb]{0.00,0.00,0.81}{#1}}
\newcommand{\BaseNTok}[1]{\textcolor[rgb]{0.00,0.00,0.81}{#1}}
\newcommand{\FloatTok}[1]{\textcolor[rgb]{0.00,0.00,0.81}{#1}}
\newcommand{\ConstantTok}[1]{\textcolor[rgb]{0.00,0.00,0.00}{#1}}
\newcommand{\CharTok}[1]{\textcolor[rgb]{0.31,0.60,0.02}{#1}}
\newcommand{\SpecialCharTok}[1]{\textcolor[rgb]{0.00,0.00,0.00}{#1}}
\newcommand{\StringTok}[1]{\textcolor[rgb]{0.31,0.60,0.02}{#1}}
\newcommand{\VerbatimStringTok}[1]{\textcolor[rgb]{0.31,0.60,0.02}{#1}}
\newcommand{\SpecialStringTok}[1]{\textcolor[rgb]{0.31,0.60,0.02}{#1}}
\newcommand{\ImportTok}[1]{#1}
\newcommand{\CommentTok}[1]{\textcolor[rgb]{0.56,0.35,0.01}{\textit{#1}}}
\newcommand{\DocumentationTok}[1]{\textcolor[rgb]{0.56,0.35,0.01}{\textbf{\textit{#1}}}}
\newcommand{\AnnotationTok}[1]{\textcolor[rgb]{0.56,0.35,0.01}{\textbf{\textit{#1}}}}
\newcommand{\CommentVarTok}[1]{\textcolor[rgb]{0.56,0.35,0.01}{\textbf{\textit{#1}}}}
\newcommand{\OtherTok}[1]{\textcolor[rgb]{0.56,0.35,0.01}{#1}}
\newcommand{\FunctionTok}[1]{\textcolor[rgb]{0.00,0.00,0.00}{#1}}
\newcommand{\VariableTok}[1]{\textcolor[rgb]{0.00,0.00,0.00}{#1}}
\newcommand{\ControlFlowTok}[1]{\textcolor[rgb]{0.13,0.29,0.53}{\textbf{#1}}}
\newcommand{\OperatorTok}[1]{\textcolor[rgb]{0.81,0.36,0.00}{\textbf{#1}}}
\newcommand{\BuiltInTok}[1]{#1}
\newcommand{\ExtensionTok}[1]{#1}
\newcommand{\PreprocessorTok}[1]{\textcolor[rgb]{0.56,0.35,0.01}{\textit{#1}}}
\newcommand{\AttributeTok}[1]{\textcolor[rgb]{0.77,0.63,0.00}{#1}}
\newcommand{\RegionMarkerTok}[1]{#1}
\newcommand{\InformationTok}[1]{\textcolor[rgb]{0.56,0.35,0.01}{\textbf{\textit{#1}}}}
\newcommand{\WarningTok}[1]{\textcolor[rgb]{0.56,0.35,0.01}{\textbf{\textit{#1}}}}
\newcommand{\AlertTok}[1]{\textcolor[rgb]{0.94,0.16,0.16}{#1}}
\newcommand{\ErrorTok}[1]{\textcolor[rgb]{0.64,0.00,0.00}{\textbf{#1}}}
\newcommand{\NormalTok}[1]{#1}
\usepackage{graphicx,grffile}
\makeatletter
\def\maxwidth{\ifdim\Gin@nat@width>\linewidth\linewidth\else\Gin@nat@width\fi}
\def\maxheight{\ifdim\Gin@nat@height>\textheight0.8\textheight\else\Gin@nat@height\fi}
\makeatother
% Scale images if necessary, so that they will not overflow the page
% margins by default, and it is still possible to overwrite the defaults
% using explicit options in \includegraphics[width, height, ...]{}
\setkeys{Gin}{width=\maxwidth,height=\maxheight,keepaspectratio}

% Prevent slide breaks in the middle of a paragraph:
\widowpenalties 1 10000
\raggedbottom

\AtBeginPart{
  \let\insertpartnumber\relax
  \let\partname\relax
  \frame{\partpage}
}
\AtBeginSection{
  \ifbibliography
  \else
    \let\insertsectionnumber\relax
    \let\sectionname\relax
    \frame{\sectionpage}
  \fi
}
\AtBeginSubsection{
  \let\insertsubsectionnumber\relax
  \let\subsectionname\relax
  \frame{\subsectionpage}
}

\setlength{\parindent}{0pt}
\setlength{\parskip}{6pt plus 2pt minus 1pt}
\setlength{\emergencystretch}{3em}  % prevent overfull lines
\providecommand{\tightlist}{%
  \setlength{\itemsep}{0pt}\setlength{\parskip}{0pt}}
\setcounter{secnumdepth}{0}

%packages
\usepackage{graphicx}
\usepackage{rotating}
\usepackage{hyperref}

\usepackage{tikz} % used for text highlighting, amongst others
%title slide stuff
%\institute{Department of Education}
%\title{Managing and Manipulating Data Using R}

%
\setbeamertemplate{navigation symbols}{} % get rid of navigation icons:

%\setbeamertemplate{frametitle}{\thesection \hspace{0.2cm} \insertframetitle}
\setbeamertemplate{section in toc}[sections numbered]
\setbeamertemplate{subsection in toc}[subsections numbered]

%define colors
%\definecolor{uva_orange}{RGB}{216,141,42} % UVa orange (Rotunda orange)
\definecolor{mygray}{rgb}{0.95, 0.95, 0.95} % for highlighted text
	% grey is equal parts red, green, blue. higher values >> lighter grey
	%\definecolor{lightgraybo}{rgb}{0.83, 0.83, 0.83}

% new commands

%highlight text with very light grey
\newcommand*{\hlg}[1]{%
	\tikz[baseline=(X.base)] \node[rectangle, fill=mygray] (X) {#1};%
}
%, inner sep=0.3mm
%highlight text with very light grey and use font associated with code
\newcommand*{\hlgc}[1]{\texttt{\hlg{#1}}}

\title{Managing and Manipulating Data Using R}
\subtitle{Introduction}
\author{Ozan Jaquette}
\date{}

\begin{document}
\frame{\titlepage}

\begin{frame}
\tableofcontents[hideallsubsections]
\end{frame}

\section{Course overview}\label{course-overview}

\section{Introduction to R}\label{introduction-to-r}

\begin{frame}[fragile]{R as a calculator}

\begin{Shaded}
\begin{Highlighting}[]
\DecValTok{5}
\CommentTok{#> [1] 5}
\DecValTok{5}\OperatorTok{+}\DecValTok{2}
\CommentTok{#> [1] 7}
\DecValTok{10}\OperatorTok{*}\DecValTok{3}
\CommentTok{#> [1] 30}
\end{Highlighting}
\end{Shaded}

\end{frame}

\begin{frame}[fragile]{Executing commands in R}

\begin{Shaded}
\begin{Highlighting}[]
\DecValTok{5}
\CommentTok{#> [1] 5}
\DecValTok{5}\OperatorTok{+}\DecValTok{2}
\CommentTok{#> [1] 7}
\DecValTok{10}\OperatorTok{*}\DecValTok{3}
\CommentTok{#> [1] 30}
\end{Highlighting}
\end{Shaded}

Three ways to execute commands in R

\begin{enumerate}
\def\labelenumi{\arabic{enumi}.}
\tightlist
\item
  Type/copy commands directly into the ``console''
\item
  `code chunks' in RMarkdown (.Rmd files)

  \begin{itemize}
  \tightlist
  \item
    Can execute one command at a time, one chunk at a time, or ``knit''
    the entire document
  \end{itemize}
\item
  R scripts (.R files)

  \begin{itemize}
  \tightlist
  \item
    This is just a text file full of R commands
  \item
    Can execute one command at a time, several commands at a time, or
    the entire script
  \end{itemize}
\end{enumerate}

\end{frame}

\begin{frame}[fragile]{Shortcuts you should learn for executing
commands}

\begin{Shaded}
\begin{Highlighting}[]
\DecValTok{5}\OperatorTok{+}\DecValTok{2}
\CommentTok{#> [1] 7}
\DecValTok{10}\OperatorTok{*}\DecValTok{3}
\CommentTok{#> [1] 30}
\end{Highlighting}
\end{Shaded}

Three ways to execute commands in R

\begin{enumerate}
\def\labelenumi{\arabic{enumi}.}
\tightlist
\item
  Type/copy commands directly into the ``console''
\item
  `code chunks' in RMarkdown (.Rmd files)

  \begin{itemize}
  \tightlist
  \item
    \textbf{Cmd/Ctrl + Enter}: execute highlighted line(s) within chunk
  \item
    \textbf{Cmd/Ctrl + Shift + k}: ``knit'' entire document
  \end{itemize}
\item
  R scripts (.R files)

  \begin{itemize}
  \tightlist
  \item
    \textbf{Cmd/Ctrl + Enter}: execute highlighted line(s)
  \item
    \textbf{Cmd/Ctrl + Shift + Enter} (without highlighting any lines):
    run entire script
  \end{itemize}
\end{enumerate}

\end{frame}

\begin{frame}[fragile]{Assignment}

\textbf{Assignment} means creating a variable -- or more generally, an
``object'' -- and assigning values to it

\begin{itemize}
\tightlist
\item
  \hlgc{<-} is the assignment operator

  \begin{itemize}
  \tightlist
  \item
    in other languages \hlgc{=} is the assignment operator
  \end{itemize}
\item
  good practice to put a space before and after assignment operator
\end{itemize}

\begin{Shaded}
\begin{Highlighting}[]
\CommentTok{# Create an object and assign value}
\NormalTok{a <-}\StringTok{ }\DecValTok{5}
\NormalTok{a}
\CommentTok{#> [1] 5}

\NormalTok{b <-}\StringTok{ "yay!"}
\NormalTok{b}
\CommentTok{#> [1] "yay!"}
\end{Highlighting}
\end{Shaded}

\end{frame}

\begin{frame}{Objects}

Most statistics software (e.g., SPSS, Stata) operates on datasets, which
consist of rows of observations and columns of variables

\begin{itemize}
\tightlist
\item
  Usually, these packages can open only one dataset at a time
\end{itemize}

R is an ``object-oriented'' programming language

\begin{itemize}
\tightlist
\item
  ``Objects are like boxes in which we can put things: data, functions,
  and even other objects.'' - Ben Skinner
\item
  There are several different ``types'' of objects in R

  \begin{itemize}
  \tightlist
  \item
    A dataset is just one type of object in R
  \item
    There is no limit to the number of objects R can hold (except
    memory)
  \item
    R ``functions'' do different things to different types of objects
  \end{itemize}
\end{itemize}

\end{frame}

\begin{frame}[fragile]{Vectors}

The fundamental object in R is the ``vector''

\begin{itemize}
\tightlist
\item
  A vector is a collection of values
\item
  The individual values within a vector are called ``elements''
\item
  The values in a vector can be numeric, character (e.g., ``Apple''), or
  any other type
\end{itemize}

Create a numeric vector that contains three elements

\begin{Shaded}
\begin{Highlighting}[]
\NormalTok{x <-}\StringTok{ }\KeywordTok{c}\NormalTok{(}\DecValTok{4}\NormalTok{, }\DecValTok{7}\NormalTok{, }\DecValTok{9}\NormalTok{)}
\NormalTok{x}
\CommentTok{#> [1] 4 7 9}
\CommentTok{# examine help file for c() function}
\end{Highlighting}
\end{Shaded}

Vector where the elements are characters

\begin{Shaded}
\begin{Highlighting}[]
\NormalTok{animals <-}\StringTok{ }\KeywordTok{c}\NormalTok{(}\StringTok{"lions"}\NormalTok{, }\StringTok{"tigers"}\NormalTok{, }\StringTok{"bears"}\NormalTok{, }\StringTok{"oh my"}\NormalTok{)}
\NormalTok{animals}
\CommentTok{#> [1] "lions"  "tigers" "bears"  "oh my"}
\end{Highlighting}
\end{Shaded}

\end{frame}

\begin{frame}{Formal classification of vectors in R}

More formally, there are two broad types of vectors

\begin{enumerate}
\def\labelenumi{\arabic{enumi}.}
\tightlist
\item
  \textbf{Atomic vectors}. There are six types:

  \begin{itemize}
  \tightlist
  \item
    \textbf{logical}, \textbf{integer}, \textbf{double},
    \textbf{character}, \textbf{complex}, and \textbf{raw}.

    \begin{itemize}
    \tightlist
    \item
      \textbf{Integer} and \textbf{double} vectors are collectively
      known as \textbf{numeric} vectors.
    \end{itemize}
  \end{itemize}
\item
  \textbf{Lists}, which are sometimes called recursive vectors because
  lists can contain other lists.
\end{enumerate}

Difference between atomic vectors and lists

\begin{itemize}
\tightlist
\item
  atomic vectors are \textbf{homogeneous}: all elements within atomic
  vector must be of the same type
\item
  lists can be \textbf{heterogeneous}: e.g., one element can be an
  integer and another element can be character
\end{itemize}

\end{frame}

\begin{frame}{Formal classification of vectors in R}

PUT PIC ON WEBSITE AND PROVIDE LINK
\includegraphics{data-structures-overview.png}

\end{frame}

\begin{frame}{Let's develop an intuitive understanding of vector types}

Technically, \textbf{lists} are a type of \textbf{vector}, but most
people think of \textbf{atomic vectors} and \textbf{lists} as
fundamentally different things

\medskip From now on, I'll use the term \textbf{vector} to refer to
atomic vectors

\medskip Rule for (atomic) vectors:

\begin{itemize}
\tightlist
\item
  all elements within a vector must have the same data ``type''
\item
  data types we will focus on in class:

  \begin{itemize}
  \tightlist
  \item
    numeric (integer and double)
  \item
    character
  \item
    logical
  \end{itemize}
\end{itemize}

\end{frame}

\begin{frame}[fragile]{``Length'' of a vector is the number of elements}

Use \texttt{length()} function to examine vector length

\begin{Shaded}
\begin{Highlighting}[]
\NormalTok{x}
\CommentTok{#> [1] 4 7 9}
\KeywordTok{length}\NormalTok{(x)}
\CommentTok{#> [1] 3}

\NormalTok{animals}
\CommentTok{#> [1] "lions"  "tigers" "bears"  "oh my"}
\KeywordTok{length}\NormalTok{(animals)}
\CommentTok{#> [1] 4}
\end{Highlighting}
\end{Shaded}

A single number {[}or string{]} is a vector of length=1

\begin{Shaded}
\begin{Highlighting}[]
\NormalTok{z <-}\StringTok{ }\DecValTok{5}
\KeywordTok{length}\NormalTok{(z)}
\CommentTok{#> [1] 1}
\KeywordTok{length}\NormalTok{(}\StringTok{"Tommy"}\NormalTok{)}
\CommentTok{#> [1] 1}
\end{Highlighting}
\end{Shaded}

\end{frame}

\begin{frame}[fragile]{Aside: Sequences}

A vector that contains a ``sequence'' of numbers (e.g., 1, 2, 3, 4) can
be created with the notation \texttt{start:end}

\begin{Shaded}
\begin{Highlighting}[]
\DecValTok{0}\OperatorTok{:}\DecValTok{4}
\CommentTok{#> [1] 0 1 2 3 4}
\DecValTok{99}\OperatorTok{:}\DecValTok{104}
\CommentTok{#> [1]  99 100 101 102 103 104}
\NormalTok{w <-}\StringTok{ }\KeywordTok{c}\NormalTok{(}\DecValTok{10}\OperatorTok{:}\DecValTok{15}\NormalTok{)}
\NormalTok{w}
\CommentTok{#> [1] 10 11 12 13 14 15}
\KeywordTok{length}\NormalTok{(w)}
\CommentTok{#> [1] 6}
\end{Highlighting}
\end{Shaded}

\end{frame}

\begin{frame}[fragile]{Data type of a vector}

Three ``types'' of vectors, where type refers to the elements within the
vector

\begin{itemize}
\tightlist
\item
  numeric: can be ``integer'' (e.g., 5) or ``double'' (e.g., 5.5)
\item
  character (e.g., ``ozan'')
\item
  logical: TRUE or FALSE; more on this later
\end{itemize}

Use \texttt{typeof()} function to examine vector type

\begin{Shaded}
\begin{Highlighting}[]
\NormalTok{x}
\CommentTok{#> [1] 4 7 9}
\KeywordTok{typeof}\NormalTok{(x)}
\CommentTok{#> [1] "double"}

\NormalTok{p <-}\StringTok{ }\KeywordTok{c}\NormalTok{(}\FloatTok{1.5}\NormalTok{, }\FloatTok{1.6}\NormalTok{)}
\NormalTok{p}
\CommentTok{#> [1] 1.5 1.6}
\KeywordTok{typeof}\NormalTok{(p)}
\CommentTok{#> [1] "double"}

\NormalTok{animals}
\CommentTok{#> [1] "lions"  "tigers" "bears"  "oh my"}
\KeywordTok{typeof}\NormalTok{(animals)}
\CommentTok{#> [1] "character"}
\end{Highlighting}
\end{Shaded}

\end{frame}

\begin{frame}[fragile]{Data type of a vector, numeric}

Numeric vectors can be ``integer'' (e.g., 5) or ``double'' (e.g., 5.5)

\begin{Shaded}
\begin{Highlighting}[]
\KeywordTok{typeof}\NormalTok{(}\FloatTok{1.5}\NormalTok{)}
\CommentTok{#> [1] "double"}
\end{Highlighting}
\end{Shaded}

R stores numbers as doubles by default.

\begin{Shaded}
\begin{Highlighting}[]
\NormalTok{x}
\CommentTok{#> [1] 4 7 9}
\KeywordTok{typeof}\NormalTok{(x)}
\CommentTok{#> [1] "double"}
\end{Highlighting}
\end{Shaded}

To make an integer, place an \hlgc{L} after the number:

\begin{Shaded}
\begin{Highlighting}[]
\KeywordTok{typeof}\NormalTok{(}\DecValTok{5}\NormalTok{)}
\CommentTok{#> [1] "double"}
\KeywordTok{typeof}\NormalTok{(5L)}
\CommentTok{#> [1] "integer"}
\end{Highlighting}
\end{Shaded}

\end{frame}

\begin{frame}[fragile]{Vector math}

Most mathematical operations operate on each element of the vector

\begin{itemize}
\tightlist
\item
  e.g., add a single value to a vector and that value added to each
  element of the vector
\end{itemize}

\begin{Shaded}
\begin{Highlighting}[]
\DecValTok{1}\OperatorTok{:}\DecValTok{3}
\CommentTok{#> [1] 1 2 3}
\DecValTok{1}\OperatorTok{:}\DecValTok{3}\OperatorTok{+}\NormalTok{.}\DecValTok{5}
\CommentTok{#> [1] 1.5 2.5 3.5}
\NormalTok{(}\DecValTok{1}\OperatorTok{:}\DecValTok{3}\NormalTok{)}\OperatorTok{*}\DecValTok{2}
\CommentTok{#> [1] 2 4 6}
\end{Highlighting}
\end{Shaded}

Mathematical operations involving two vectors have the same length

\begin{itemize}
\tightlist
\item
  DESCRIBE IN WORDS
\end{itemize}

\begin{Shaded}
\begin{Highlighting}[]
\KeywordTok{c}\NormalTok{(}\DecValTok{1}\NormalTok{,}\DecValTok{1}\NormalTok{,}\DecValTok{1}\NormalTok{)}\OperatorTok{+}\KeywordTok{c}\NormalTok{(}\DecValTok{1}\NormalTok{,}\DecValTok{0}\NormalTok{,}\DecValTok{2}\NormalTok{)}
\CommentTok{#> [1] 2 1 3}
\KeywordTok{c}\NormalTok{(}\DecValTok{1}\NormalTok{,}\DecValTok{1}\NormalTok{,}\DecValTok{1}\NormalTok{)}\OperatorTok{*}\KeywordTok{c}\NormalTok{(}\DecValTok{1}\NormalTok{,}\DecValTok{0}\NormalTok{,}\DecValTok{2}\NormalTok{)}
\CommentTok{#> [1] 1 0 2}
\end{Highlighting}
\end{Shaded}

\end{frame}

\begin{frame}[fragile]{All elements in (atomic) vector must have same
data type.}

``When you try and create a vector containing multiple types with
\hlgc{c()}: the most complex type always wins'' - Wickham

\begin{Shaded}
\begin{Highlighting}[]
\DecValTok{1}\OperatorTok{:}\DecValTok{3}
\CommentTok{#> [1] 1 2 3}
\KeywordTok{typeof}\NormalTok{(}\DecValTok{1}\OperatorTok{:}\DecValTok{3}\NormalTok{)}
\CommentTok{#> [1] "integer"}

\NormalTok{mix <-}\StringTok{ }\KeywordTok{c}\NormalTok{(}\DecValTok{1}\OperatorTok{:}\DecValTok{3}\NormalTok{, }\StringTok{"hi!"}\NormalTok{)}
\NormalTok{mix}
\CommentTok{#> [1] "1"   "2"   "3"   "hi!"}
\KeywordTok{typeof}\NormalTok{(mix)}
\CommentTok{#> [1] "character"}
\end{Highlighting}
\end{Shaded}

\end{frame}

\begin{frame}[fragile]{Data type of a vector, logical}

Logical vectors can take three possible values: \hlgc{TRUE},
\hlgc{FALSE}, \hlgc{NA}

\begin{itemize}
\tightlist
\item
  \hlgc{TRUE}, \hlgc{FALSE}, \hlgc{NA} are special keywords, different
  from the character strings \hlgc{"TRUE"}, \hlgc{"FALSE"}, \hlgc{"NA"}
\item
  Don't worry about \hlgc{"NA"} for now
\end{itemize}

\begin{Shaded}
\begin{Highlighting}[]
\KeywordTok{typeof}\NormalTok{(}\OtherTok{TRUE}\NormalTok{)}
\CommentTok{#> [1] "logical"}
\KeywordTok{typeof}\NormalTok{(}\StringTok{"TRUE"}\NormalTok{)}
\CommentTok{#> [1] "character"}

\KeywordTok{typeof}\NormalTok{(}\KeywordTok{c}\NormalTok{(}\OtherTok{TRUE}\NormalTok{,}\OtherTok{FALSE}\NormalTok{,}\OtherTok{NA}\NormalTok{))}
\CommentTok{#> [1] "logical"}
\KeywordTok{typeof}\NormalTok{(}\KeywordTok{c}\NormalTok{(}\OtherTok{TRUE}\NormalTok{,}\OtherTok{FALSE}\NormalTok{,}\OtherTok{NA}\NormalTok{,}\StringTok{"FALSE"}\NormalTok{))}
\CommentTok{#> [1] "character"}
\end{Highlighting}
\end{Shaded}

We'll learn more about logical vectors later

\end{frame}

\begin{frame}{Lists}

What is a \textbf{list}?

\begin{itemize}
\tightlist
\item
  Like (atomic) vectors, a list is an object that contains
  \textbf{elements}
\item
  Unlike vectors, data types can differ across elements within a list
\item
  An element within a list can be another list

  \begin{itemize}
  \tightlist
  \item
    this characteristic makes lists more complicated than vectors
  \item
    suitable for representing hierarchical data
  \end{itemize}
\end{itemize}

Lists are more complicated than vectors; today we'll just provide a
basic introduction

\end{frame}

\begin{frame}[fragile]{Create lists using \hlgc{list()} function}

Review: a vector

\begin{Shaded}
\begin{Highlighting}[]
\NormalTok{a <-}\StringTok{ }\KeywordTok{c}\NormalTok{(}\DecValTok{1}\NormalTok{,}\DecValTok{2}\NormalTok{,}\DecValTok{3}\NormalTok{)}
\KeywordTok{typeof}\NormalTok{(a)}
\CommentTok{#> [1] "double"}
\KeywordTok{length}\NormalTok{(a)}
\CommentTok{#> [1] 3}
\end{Highlighting}
\end{Shaded}

A list

\begin{Shaded}
\begin{Highlighting}[]
\NormalTok{b <-}\StringTok{ }\KeywordTok{list}\NormalTok{(}\DecValTok{1}\NormalTok{,}\DecValTok{2}\NormalTok{,}\DecValTok{3}\NormalTok{)}
\KeywordTok{typeof}\NormalTok{(b)}
\CommentTok{#> [1] "list"}
\KeywordTok{length}\NormalTok{(b)}
\CommentTok{#> [1] 3}
\NormalTok{b }\CommentTok{# print list is awkward}
\CommentTok{#> [[1]]}
\CommentTok{#> [1] 1}
\CommentTok{#> }
\CommentTok{#> [[2]]}
\CommentTok{#> [1] 2}
\CommentTok{#> }
\CommentTok{#> [[3]]}
\CommentTok{#> [1] 3}
\end{Highlighting}
\end{Shaded}

\end{frame}

\begin{frame}[fragile]{Investigate structure of lists using \hlgc{str()}
function}

\begin{Shaded}
\begin{Highlighting}[]
\NormalTok{b <-}\StringTok{ }\KeywordTok{list}\NormalTok{(}\DecValTok{1}\NormalTok{,}\DecValTok{2}\NormalTok{,}\DecValTok{3}\NormalTok{)}
\KeywordTok{typeof}\NormalTok{(b)}
\CommentTok{#> [1] "list"}
\KeywordTok{length}\NormalTok{(b)}
\CommentTok{#> [1] 3}
\KeywordTok{str}\NormalTok{(b)}
\CommentTok{#> List of 3}
\CommentTok{#>  $ : num 1}
\CommentTok{#>  $ : num 2}
\CommentTok{#>  $ : num 3}
\end{Highlighting}
\end{Shaded}

Can also apply \hlgc{str()} to vectors

\begin{Shaded}
\begin{Highlighting}[]
\NormalTok{a}
\CommentTok{#> [1] 1 2 3}
\KeywordTok{str}\NormalTok{(a)}
\CommentTok{#>  num [1:3] 1 2 3}
\end{Highlighting}
\end{Shaded}

\end{frame}

\begin{frame}[fragile]{Elements within list can have different data
types}

\begin{Shaded}
\begin{Highlighting}[]
\NormalTok{b <-}\StringTok{ }\KeywordTok{list}\NormalTok{(}\DecValTok{1}\NormalTok{,}\DecValTok{2}\NormalTok{,}\StringTok{"apple"}\NormalTok{)}
\KeywordTok{typeof}\NormalTok{(b)}
\CommentTok{#> [1] "list"}
\KeywordTok{str}\NormalTok{(b)}
\CommentTok{#> List of 3}
\CommentTok{#>  $ : num 1}
\CommentTok{#>  $ : num 2}
\CommentTok{#>  $ : chr "apple"}
\end{Highlighting}
\end{Shaded}

Vector

\begin{Shaded}
\begin{Highlighting}[]
\NormalTok{a <-}\StringTok{ }\KeywordTok{c}\NormalTok{(}\DecValTok{1}\NormalTok{,}\DecValTok{2}\NormalTok{,}\StringTok{"apple"}\NormalTok{)}
\KeywordTok{typeof}\NormalTok{(a)}
\CommentTok{#> [1] "character"}
\KeywordTok{str}\NormalTok{(a)}
\CommentTok{#>  chr [1:3] "1" "2" "apple"}
\end{Highlighting}
\end{Shaded}

\end{frame}

\begin{frame}[fragile]{Lists can contain other lists}

\begin{Shaded}
\begin{Highlighting}[]
\NormalTok{x1 <-}\StringTok{ }\KeywordTok{list}\NormalTok{(}\DecValTok{1}\NormalTok{, }\KeywordTok{list}\NormalTok{(}\StringTok{"apple"}\NormalTok{, }\StringTok{"orange"}\NormalTok{), }\KeywordTok{list}\NormalTok{(}\DecValTok{1}\NormalTok{, }\DecValTok{2}\NormalTok{, }\DecValTok{3}\NormalTok{))}

\KeywordTok{typeof}\NormalTok{(x1)}
\CommentTok{#> [1] "list"}

\KeywordTok{str}\NormalTok{(x1)}
\CommentTok{#> List of 3}
\CommentTok{#>  $ : num 1}
\CommentTok{#>  $ :List of 2}
\CommentTok{#>   ..$ : chr "apple"}
\CommentTok{#>   ..$ : chr "orange"}
\CommentTok{#>  $ :List of 3}
\CommentTok{#>   ..$ : num 1}
\CommentTok{#>   ..$ : num 2}
\CommentTok{#>   ..$ : num 3}
\end{Highlighting}
\end{Shaded}

\end{frame}

\begin{frame}[fragile]{You can name each element in the list}

\begin{Shaded}
\begin{Highlighting}[]
\NormalTok{x2 <-}\StringTok{ }\KeywordTok{list}\NormalTok{(}\DataTypeTok{a=}\DecValTok{1}\NormalTok{, }\DataTypeTok{b=}\KeywordTok{list}\NormalTok{(}\StringTok{"apple"}\NormalTok{, }\StringTok{"orange"}\NormalTok{), }\DataTypeTok{c=}\KeywordTok{list}\NormalTok{(}\DecValTok{1}\NormalTok{, }\DecValTok{2}\NormalTok{, }\DecValTok{3}\NormalTok{))}

\KeywordTok{str}\NormalTok{(x2)}
\CommentTok{#> List of 3}
\CommentTok{#>  $ a: num 1}
\CommentTok{#>  $ b:List of 2}
\CommentTok{#>   ..$ : chr "apple"}
\CommentTok{#>   ..$ : chr "orange"}
\CommentTok{#>  $ c:List of 3}
\CommentTok{#>   ..$ : num 1}
\CommentTok{#>   ..$ : num 2}
\CommentTok{#>   ..$ : num 3}
\end{Highlighting}
\end{Shaded}

\hlgc{names()} function shows names of elements in the list

\begin{Shaded}
\begin{Highlighting}[]
\KeywordTok{names}\NormalTok{(x2) }\CommentTok{# has names}
\CommentTok{#> [1] "a" "b" "c"}
\KeywordTok{names}\NormalTok{(x1) }\CommentTok{# no names}
\CommentTok{#> NULL}
\end{Highlighting}
\end{Shaded}

\end{frame}

\begin{frame}[fragile]{Access individual elements in a ``named'' list}

Syntax: \texttt{list\_name\$element\_name}

\begin{Shaded}
\begin{Highlighting}[]
\NormalTok{x2 <-}\StringTok{ }\KeywordTok{list}\NormalTok{(}\DataTypeTok{a=}\DecValTok{1}\NormalTok{, }\DataTypeTok{b=}\KeywordTok{list}\NormalTok{(}\StringTok{"apple"}\NormalTok{, }\StringTok{"orange"}\NormalTok{), }\DataTypeTok{c=}\KeywordTok{list}\NormalTok{(}\DecValTok{1}\NormalTok{, }\DecValTok{2}\NormalTok{, }\DecValTok{3}\NormalTok{))}
\KeywordTok{typeof}\NormalTok{(x2}\OperatorTok{$}\NormalTok{a)}
\CommentTok{#> [1] "double"}
\KeywordTok{length}\NormalTok{(x2}\OperatorTok{$}\NormalTok{a)}
\CommentTok{#> [1] 1}

\KeywordTok{typeof}\NormalTok{(x2}\OperatorTok{$}\NormalTok{b)}
\CommentTok{#> [1] "list"}
\KeywordTok{length}\NormalTok{(x2}\OperatorTok{$}\NormalTok{b)}
\CommentTok{#> [1] 2}

\KeywordTok{typeof}\NormalTok{(x2}\OperatorTok{$}\NormalTok{c)}
\CommentTok{#> [1] "list"}
\KeywordTok{length}\NormalTok{(x2}\OperatorTok{$}\NormalTok{c)}
\CommentTok{#> [1] 3}
\end{Highlighting}
\end{Shaded}

Note: length can differ across elements within a list

\end{frame}

\begin{frame}[fragile]{Compare structure of list to structure of element
within a list}

\begin{Shaded}
\begin{Highlighting}[]
\KeywordTok{str}\NormalTok{(x2)}
\CommentTok{#> List of 3}
\CommentTok{#>  $ a: num 1}
\CommentTok{#>  $ b:List of 2}
\CommentTok{#>   ..$ : chr "apple"}
\CommentTok{#>   ..$ : chr "orange"}
\CommentTok{#>  $ c:List of 3}
\CommentTok{#>   ..$ : num 1}
\CommentTok{#>   ..$ : num 2}
\CommentTok{#>   ..$ : num 3}

\KeywordTok{str}\NormalTok{(x2}\OperatorTok{$}\NormalTok{c)}
\CommentTok{#> List of 3}
\CommentTok{#>  $ : num 1}
\CommentTok{#>  $ : num 2}
\CommentTok{#>  $ : num 3}
\end{Highlighting}
\end{Shaded}

\end{frame}

\begin{frame}[fragile]{A dataset is just a list!}

\medskip A data frame is a list with the following characteristics:

\begin{itemize}
\tightlist
\item
  Data type can differ across elements (like all lists)
\item
  Each \textbf{element} in data frame must be a \textbf{vector}, not a
  \textbf{list}

  \begin{itemize}
  \tightlist
  \item
    Each element (column) is a variable
  \end{itemize}
\item
  Each \textbf{element} in a data frame must have the same length

  \begin{itemize}
  \tightlist
  \item
    The length of an element is the number of observations (rows)
  \item
    so each variable in data frame must have same number of observations
  \end{itemize}
\end{itemize}

\begin{Shaded}
\begin{Highlighting}[]
\KeywordTok{names}\NormalTok{(df)}
\CommentTok{#> [1] "mpg" "cyl" "hp"}
\KeywordTok{head}\NormalTok{(df, }\DataTypeTok{n=}\DecValTok{5}\NormalTok{) }\CommentTok{# print first 5 rows}
\CommentTok{#> # A tibble: 5 x 3}
\CommentTok{#>     mpg   cyl    hp}
\CommentTok{#> * <dbl> <dbl> <dbl>}
\CommentTok{#> 1  21       6   110}
\CommentTok{#> 2  21       6   110}
\CommentTok{#> 3  22.8     4    93}
\CommentTok{#> 4  21.4     6   110}
\CommentTok{#> 5  18.7     8   175}
\end{Highlighting}
\end{Shaded}

\end{frame}

\begin{frame}[fragile]{A data frame is a named list}

\begin{Shaded}
\begin{Highlighting}[]
\KeywordTok{typeof}\NormalTok{(df)}
\CommentTok{#> [1] "list"}
\KeywordTok{names}\NormalTok{(df)}
\CommentTok{#> [1] "mpg" "cyl" "hp"}
\KeywordTok{length}\NormalTok{(df) }\CommentTok{# length=number of variables}
\CommentTok{#> [1] 3}
\KeywordTok{str}\NormalTok{(df)}
\CommentTok{#> 'data.frame':    32 obs. of  3 variables:}
\CommentTok{#>  $ mpg: num  21 21 22.8 21.4 18.7 18.1 14.3 24.4 22.8 19.2 ...}
\CommentTok{#>  $ cyl: num  6 6 4 6 8 6 8 4 4 6 ...}
\CommentTok{#>  $ hp : num  110 110 93 110 175 105 245 62 95 123 ...}
\end{Highlighting}
\end{Shaded}

Like any named list, can examine the elements

\begin{Shaded}
\begin{Highlighting}[]
\KeywordTok{typeof}\NormalTok{(df}\OperatorTok{$}\NormalTok{mpg)}
\CommentTok{#> [1] "double"}
\KeywordTok{length}\NormalTok{(df}\OperatorTok{$}\NormalTok{mpg) }\CommentTok{# length=number of rows/obs}
\CommentTok{#> [1] 32}
\KeywordTok{str}\NormalTok{(df}\OperatorTok{$}\NormalTok{mpg)}
\CommentTok{#>  num [1:32] 21 21 22.8 21.4 18.7 18.1 14.3 24.4 22.8 19.2 ...}
\end{Highlighting}
\end{Shaded}

\end{frame}

\begin{frame}{Main takeaways on types of objects -- vectors and lists}

Basic data stuctures

\begin{enumerate}
\def\labelenumi{\arabic{enumi}.}
\tightlist
\item
  \textbf{(Atomic) vectors}: \textbf{logical}, \textbf{integer},
  \textbf{double}, \textbf{character}.

  \begin{itemize}
  \tightlist
  \item
    each element in vector must have same data type
  \end{itemize}
\item
  \textbf{Lists}:

  \begin{itemize}
  \tightlist
  \item
    Data type can differ across elements
  \end{itemize}
\end{enumerate}

Takeaways

\begin{itemize}
\tightlist
\item
  These concepts are difficult; ok to feel confused
\item
  I will reinforce these concepts throughout the course
\item
  Good practice: run simple diagnostics on any new object

  \begin{itemize}
  \tightlist
  \item
    \hlgc{length()} : how many \textbf{elements} in the object
  \item
    \hlgc{typeof()} : what \textbf{type} of data is the object
  \item
    \hlgc{str()} : hierarchical structure of the object
  \end{itemize}
\end{itemize}

\end{frame}

\begin{frame}{Main takeaways on types of objects -- vectors and lists}

Basic data stuctures

\begin{enumerate}
\def\labelenumi{\arabic{enumi}.}
\tightlist
\item
  \textbf{(Atomic) vectors}: \textbf{logical}, \textbf{integer},
  \textbf{double}, \textbf{character}.

  \begin{itemize}
  \tightlist
  \item
    each element in vector must have same data type
  \end{itemize}
\item
  \textbf{Lists}:

  \begin{itemize}
  \tightlist
  \item
    Data type can differ across elements
  \end{itemize}
\end{enumerate}

Takeaways

\begin{itemize}
\tightlist
\item
  These data structures (vectors, lists) and data types (e.g.,
  character, numeric, logical) are the basic building blocks of all
  object oriented programming languages
\item
  Application to statistical analysis

  \begin{itemize}
  \tightlist
  \item
    Datasets are just lists
  \item
    The individual elements -- columns/variables -- within a dataset are
    just vectors
  \end{itemize}
\item
  These structures and data types are foundational for all ``data
  science'' applications, e.g.,:

  \begin{itemize}
  \tightlist
  \item
    maapping, webscraping, network analysis, etc.
  \end{itemize}
\end{itemize}

\end{frame}

\end{document}
